% java -cp /home/lukas/IdeaProjects/AAProject01/src/:/home/lukas/packages/algs4.jar:/home/lukas/packages/stdlib.jar:/home/lukas/packages/acm.jar:/home/lukas/packages/turtle.jar TurtleGraphics
% Use this template to write your solutions to COS 423 problem sets
% pdflatex filename.tex
\documentclass[12pt]{article}
\usepackage[utf8]{inputenc}
\usepackage{amsmath, amsfonts, amsthm, amssymb, algorithm, graphicx, mathtools,xfrac}
\usepackage[noend]{algpseudocode}
\usepackage{fancyhdr, lastpage}
\usepackage[vmargin=1.20in,hmargin=1.25in,centering,letterpaper]{geometry}
\setlength{\headsep}{.50in}
\setlength{\headheight}{15pt}

% Landau notation
\DeclareMathOperator{\BigOm}{\mathcal{O}}
\newcommand{\BigOh}[1]{\BigOm\left({#1}\right)}
\DeclareMathOperator{\BigTm}{\Theta}
\newcommand{\BigTheta}[1]{\BigTm\left({#1}\right)}
\DeclareMathOperator{\BigWm}{\Omega}
\newcommand{\BigOmega}[1]{\BigWm\left({#1}\right)}
\DeclareMathOperator{\LittleOm}{\mathrm{o}}
\newcommand{\LittleOh}[1]{\LittleOm\left({#1}\right)}
\DeclareMathOperator{\LittleWm}{\omega}
\newcommand{\LittleOmega}[1]{\LittleWm\left({#1}\right)}

% argmin and argmax
\newcommand{\argmin}{\operatornamewithlimits{argmin}}
\newcommand{\argmax}{\operatornamewithlimits{argmax}}

\newcommand{\calP}{\mathcal{P}}
\newcommand{\Z}{\mathbb{Z}}
\newcommand{\R}{\mathbb{R}}
\newcommand{\Exp}{\mathbb{E}}
\newcommand{\Q}{\mathbb{Q}}
\newcommand{\sign}{\mathrm{sign\ }}
\newcommand{\abs}{\mathrm{abs\ }}
\newcommand{\eps}{\varepsilon}
\newcommand{\zo}{\{0, 1\}}
\newcommand{\SAT}{\mathit{SAT}}
\renewcommand{\P}{\mathbf{P}}
\newcommand{\NP}{\mathbf{NP}}
\newcommand{\coNP}{\co{NP}}
\newcommand{\co}[1]{\mathbf{co#1}}
\renewcommand{\Pr}{\mathop{\mathrm{Pr}}}

% theorems, lemmas, invariants, etc.
\newtheorem{theorem}{Theorem}
\newtheorem{lemma}[theorem]{Lemma}
\newtheorem{invariant}[theorem]{Invariant}
\newtheorem{corollary}[theorem]{Corollary}
\newtheorem{definition}[theorem]{Definition}
\newtheorem{property}[theorem]{Property}
\newtheorem{proposition}[theorem]{Proposition}

% piecewise functions
\newenvironment{piecewise}{\left\{\begin{array}{l@{,\ }l}}
{\end{array}\right.}

% paired delimiters
\DeclarePairedDelimiter{\ceil}{\lceil}{\rceil}
\DeclarePairedDelimiter{\floor}{\lfloor}{\rfloor}
\DeclarePairedDelimiter{\len}{|}{|}
\DeclarePairedDelimiter{\set}{\{}{\}}

\makeatletter
\@addtoreset{equation}{section}
\makeatother
\renewcommand{\theequation}{\arabic{section}.\arabic{equation}}

% algorithms
\algnewcommand\algorithmicinput{\textbf{INPUT:}}
\algnewcommand\INPUT{\item[\algorithmicinput]}
\algnewcommand\algorithmicoutput{\textbf{OUTPUT:}}
\algnewcommand\OUTPUT{\item[\algorithmicoutput]}


% Formating Macros

\pagestyle{fancy}
\lhead{\sc \hmwkClass $\;\;\cdot \;\;$ \hmwkSemester $\;\;\cdot\;\;$
Problem \hmwkAssignmentNum.\hmwkProblemNum}
%\chead{\sc Problem \hmwkAssignmentNum.\hmwkProblemNum}
%\chead{}
\rhead{\em \hmwkAuthorName\ $($\hmwkAuthorID$)$}
\cfoot{}
\lfoot{}
\rfoot{\sc Page\ \thepage\ of\ \protect\pageref{LastPage}}
\renewcommand\headrulewidth{0.4pt}
\renewcommand\footrulewidth{0.4pt}

\fancypagestyle{fancycollab}
{
    \lfoot{\em Collaborators: \hmwkCollaborators}
}

\fancypagestyle{problemstatement}
{
    \rhead{}
    \lfoot{}
}

%%%%%% Begin document with header and title %%%%%%%%%%%%%%%%%%%%%%%%%

\begin{document}

%%%%%% Header Information %%%%%%%%%%%%%%%%%%%%%%%%%%%%%%%%%%%%%%%%%%%

%%% Shouldn't need to change these
\newcommand{\hmwkClass}{COS 255}
\newcommand{\hmwkSemester}{Spring 2016}

%%% Your name, in standard First Last format
\newcommand{\hmwkAuthorName}{Annamalis \& Lukas}
%%% Your NetID
\newcommand{\hmwkAuthorID}{asharp \& lleung}

%%% The problem set number (just the number)
\newcommand{\hmwkAssignmentNum}{0}

%%% The problem number (just the number)
\newcommand{\hmwkProblemNum}{1}

%%% A list of your collaborators' NetIDs, separated by ", ".
%%% You can use a new line ("\\") in the middle to prevent a long
%%% list from overflowing.
\newcommand{\hmwkCollaborators}{asharp}
%%% Sets the collaborator list to appear on the first page
\thispagestyle{fancycollab}

%%%%%%% begin Solution %%%%%%%%%%%%%%%%%%%%%%%%%%%%%%%%%%%%%%%%%%%%%

\noindent
{\em This document contains an explaination of algorithmic design and solutions to two problems.
 First will be the statement of the question followed by the mathematical formulation, the
 Brute-Force solution, our solution, proof of correctness for our solution and any slight
 improvements we could do.}

%%%%%%% begin Turtle Problem %%%%%%%%%%%%%%%%%%%%%%%%%%%%%%%%%%%%%%

\section{Turtle Graphics Problem}
Our program will allow the user to program a virtual turtle to move around the graphics window, drawing
pictures as it goes. The  virtual  turtle  in  this application can move forward, turn left, and turn
right. Each of these operations takes a parameter that indicates either how far the turtle should move
or how many degrees it should turn. The turtle also comes equipped with a virtual pen in its belly
that it can use to draw a line on the graphics window.

%%%%%%% end Problem %%%%%%%%%%%%%%%%%%%%%%%%%%%%%%%%%%%%%%%%%%%%%%%%

\newpage

%%%%%%% Mathematical Formulation %%%%%%%%%%%%%%%%%%%%%%%%%%%%%%%%%%%%%%%%%%%%%%%

\subsection{Mathematical Formulation}
In the TurtleTokenizer algorithms we will be taking in a String of length $N$ and removing all
white-spaces so that the resulting string is of length $N^{*}$ were $N^{*}\ \leq\ N$. We say
that this string can be composed as $n_0n_1...n_k\ =\ N^{*}$ where $k$ is the number of tokens.
Let $M$ then represent the longest $n_i$. $(Note\ that\ each\ token\ does\ not\ need\ to\ be\
valid)$

%%%%%%% Algorithm %%%%%%%%%%%%%%%%%%%%%%%%%%%%%%%%%%%%%%%%%%%%%%%

\subsection{Solution}
We approach the problem by first determining how to split up tokens.  We do this using our
Tokenizer algorithm which will take a string and, upon request give the next token (action)
that will be executed by TurtleGraphics. \\
\indent Within the Tokenizer, we begin by initializing our tokens to be a string with no spaces and
all letters being uppercase as well as keeping track of our current place within the string
at all times. \textit{(Note: The string is also checked for miss-matching backets, so that this
is a valid input in that sense)}

\begin{algorithm}[H]
\caption{Tokenizer I}
\begin{algorithmic}
    \Procedure{Initialize}{$command$}
        \State $command \gets$ remove white-spaces and make all uppercase
        \State $command$.\Call{assertValidity}{ } // check to make sure that \{\}'s match up
        \State (String) $tokens \gets$ $commands$
        \State $currentIndex \gets$ 0
    \EndProcedure
\end{algorithmic}
\end{algorithm}

Whenever the user calls nextToken, a stringbuilder is initialized which will
be returned at the end. We then have a defense for whether or not there is an invalid command
entered \textit{(i.e. 10 by itself)} by using the fact that a \textbf{valid} token must begin
with one of the characters \{F,R,L,X,D,U,'\{'\}. If it is not one of these, Tokenizer will
skip over the character and continue. If one of the letters is found, it will append the letter
to the stringbuilder and check for proceeding numbers. If a number is found it will be appended.
The current token is complete, and the string is returned. If the next char is '\{', Tokenizer keeps
track of how many open and closed brackets are seen and builds the next token from the characters within
the braces if there are an equal number of open and closed brackets. The reason for the counter
is to ensure that all nested brackets are also accounted for. Once this (possibly mega) token has
been built, it is returned.

\begin{algorithm}[H]
\caption{Tokenizer II}
\begin{algorithmic}
    \Procedure{nextToken}{ }
        \State $token \gets$ String
        \While {$currentIndex$ \textless tokens.\Call{length}{ }}
            \If{tokens.\Call{charAt}{currentIndex} $\in \{F,R,L,X,D,U\}$ }
                \State $token.\Call{append}{letter}$
                \While{nextLetterIsANumber}
                     \State $token.\Call{append}{number}$
                \EndWhile
                \State \Call{return}{$token$}
            \ElsIf{tokens.\Call{charAt}{currentIndex} == "\{"}
                \State $numOpen = 1$
                \While{thereAreMoreChars \&\& numOpen \textgreater\ 0}
                    \State $token.\Call{append}{char}$
                    \If{nextchar is \{}
                        \State $numOpen++$
                    \ElsIf{nextChar is \}}
                        \State $numOpen--$
                        \If{numOpen == 0}
                            \State $token.\Call{append}{\}}$
                        \EndIf
                    \EndIf
                \EndWhile
                \State \Call{return}{$token$}
            \Else
                \State $currentIndex++$
            \EndIf
        \EndWhile
        \State \Call{return}{$token$}
    \EndProcedure
\end{algorithmic}
\end{algorithm}

TurtleGraphics takes the given token and uses the $execute$ method to scan it for one of the
command characters \{F, R, L, X, D, U\}. If the command is F, R, or L, TurtleGraphics moves
the Turtle the specified amount or default distance if proceeding number is given. If the
token contains U or D, the Turtle raises or lowers its pen to draw. If a token containing
X is given, the Turtle will be executing a token, separated by \{\}, repeated a specified
number of times. $execute$ gets the number proceeding X and the token contained within the
brackets. Then the $execute$ is called the specified number of times on the bracketed token.

\begin{algorithm}[H]
\caption{Turtle Graphics I}
\begin{algorithmic}
    \Procedure{execute}{$command$}
        \State $tokenizer \gets$ TurtleTokenizer($string$)
        \While{tokenizer.\Call{hasMoreTokens}{ }}
            \If{nextToken = 'F'}
                \If{nextToken only 'F'}
                    \State turtle.\Call{forward}{50}
                \Else
                    \State turtle.\Call{forward}{trailingNumbers}
                \EndIf
            \ElsIf{nextToken = 'L'}
                \If{nextToken only 'L'}
                    \State turtle.\Call{left}{90}
                \Else
                    \State turtle.\Call{left}{trailingNumbers}
                \EndIf
            \ElsIf{nextToken = 'R'}
                \If{nextToken only 'R'}
                    \State turtle.\Call{right}{90}
                \Else
                    \State turtle.\Call{right}{trailingNumbers}
                \EndIf
            \ElsIf{nextToken = 'U'}
                \State turtle.\Call{penUp}{ }
            \ElsIf{nextToken = 'D'}
                \State turtle.\Call{penDown}{ }
            \ElsIf{nextToken = 'X'}
                \State $numRepeats \gets$ trailingNumbers
                \State $repeatable \gets$ nextToken
                \For{numRepeats}
                    \State \Call{execute}{repeatable}
                \EndFor
            \EndIf
        \EndWhile
    \EndProcedure
\end{algorithmic}
\end{algorithm}

ReplaceAction begins by getting the replacement field string from the user interface.
For the length of the string, our program looks through the first half up to the -
symbol of the -\textgreater and appends each character to a StringBuilder original.
 Next the program builds the replacement commend from the latter half of the replacement field.
 ReplaceAction then calls the helper method replaceAll to replace all instances of the original
 command and return the new token to the user interface.

\begin{algorithm}[H]
\caption{Turtle Graphics II}
\begin{algorithmic}
    \Procedure{replaceAction}{ }
        \State $replacement \gets$ ui.\Call{getReplacementField}{ }
        \State $replacement \gets$ remove white-spaces, make all uppercase and assert validity
        \State hasArrow = false
        \For{replacement.\Call{length}{ }}
            \If{there is a "-\textgreater"}
                \State hasArrow = true
            \EndIf
        \EndFor
        \If{!hasArrow}
            \State{Throw Exception}
        \EndIf
        \State $original \gets$ StringBuilder
        \For{replacement.\Call{length}{ }}
            \If{$current\_char =\ '-'$}
                \State \Call{break}{ }
            \Else
                \State $original.\Call{append}{current\_char}$
            \EndIf
        \EndFor
        \State $replaceWith \gets$ StringBuilder
        \For{from (original.\Call{length}{ } + 1) to replacement.\Call{length}{ }}
            \State $replaceWith.\Call{append}{current\_char}$
        \EndFor
        \State $oldProgramText \gets$ ui.\Call{getProgramText}{ }
        \State $newProgramText \gets$ \Call{replaceAll}{$replacement,\ original,\ replaceWith$}
        \State ui.\Call{setProgramText}{$newProgramText$}
    \EndProcedure
\end{algorithmic}
\end{algorithm}

ReplaceAll scans the original token given for the original command to be replaced with the
replacement command. Scanning each character of the token, ReplaceAll checks for the first
character of the original command. If the character is not the same as the original command,
it is appended to the new token. If the same character is seen, ReplaceAll looks ahead to
check if the commands are the same. If they are, the original command is replaced by the
replacement comment. If not, ReplaceAll continues appending characters. Once the end of
token is reached, the newly built token with all original commands replaced is returned.

\begin{algorithm}[H]
\caption{Turtle Graphics III}
\begin{algorithmic}
    \Procedure{replaceAll}{$token, original, replacement$}
        \State $result \gets$ StringBuilder
        \State currentIndex = 0
        \While{$currentIndex\ \textless\ token.\Call{length}{ }$}
            \If{$currentChar\ =\ original.firstChar$}
                \For{$original.\Call{length}{ }$}
                    \If{chars are not equal}
                        \State \Call{break}{}
                    \EndIf
                \EndFor
                \If{Found original}
                    \State $result.\Call{append}{replacement}$
                \Else
                    \State $result.\Call{append}{stringFrom\_currentIndex}$
                \EndIf
                \State update $currentIndex$
            \Else
                \State result.\Call{append}{$currentChar$}
                \State $currentIndex++$
            \EndIf
        \EndWhile
        \State \Call{return}{$result$}
    \EndProcedure
\end{algorithmic}
\end{algorithm}


%%%%%%% Correctness %%%%%%%%%%%%%%%%%%%%%%%%%%%%%%%%%%%%%%%%%%%%%%%

\subsection{Correctness}

\begin{proposition}
The TurtleTokenizer algorithm either outputs the next valid token while ignoring invalid tokens.
\end{proposition}

\begin{proof}
Our Tokenizer algorithm will "clean" the inputed string so that if it has invalid brackets, an
exception is thrown and all of the spaces are removed to allow for accurate pointers to keep
track of the current position. This cleaned version is stored as well as the pointer as instance
variables allowing for the user to easily ask for the $nextToken()$, returning the next valid
token. The validity of the token is assertained by the if-statements, if not a valid letter or
not a '\{', the program will skip this portion of the string. The program will also determine if
a multiplier is followed by a '\{', and if not will throw the appropriate exception.
\end{proof}

\begin{proposition}
The $execute()$ algorithm will interpret the user inputted string and carry out the
corresponding commands
\end{proposition}

\begin{proof}
This is more of a trivial proof as it is largely dependent on the success of the aforementioned
tokenizer algorithm, however when it gets a valid token (determined to be so by the tokenizer),
it will execute the command with the starting letter. If the token is a multiplyer, it will
parse the number of repetitions from it, and use a for-loop to execute the subsequent command
block, calling $execute$ recursively.
\end{proof}

\begin{proposition}
The $replaceAction()$ algorithm will replace all occurances of a given command with a user specified
command and throw an exception if the command called with and invalid replacement statement.
\end{proposition}

\begin{proof}
The $replaceAction()$ first checkes that the replacement statement is valid through linearly searching
for the "-\textgreater" symbol, if not found will end. Otherwise it will get the original command to
be replaced and the command to replace it. It will then linearly search through the tokens again and
look for every instance where the original is. If found, the program will replace it. Otherwise it will
keep appending the original commands.
\end{proof}
%%%%%%% Analysis %%%%%%%%%%%%%%%%%%%%%%%%%%%%%%%%%%%%%%%%%%%%%%%

\subsection{Analysis}

\begin{proposition}
\label{numq}
The $nextToken()$ algorithm will have time and space complexity in respect to $M$, the length of
the longest token.
\end{proposition}

\begin{proof}
Since TurtleTokenizer is a linear process that moves through each index of the token, the average
time complexity is \textbf{O(M)} where $M$ is the length of the longest token.

\begin{center}
    Giving us an overarching \textbf{O(M)} complexity.
\end{center}

\end{proof}

\begin{proposition}
\label{numq}
The $execute$ algorithm will have time and space complexity in respect to $N^{\dagger}$, the longest
expanded string of commands
\end{proposition}

\begin{proof}
Given an input of length $N$, if it is expanded to its maximum length where all loops are expanded
and written out, it can be represented as $N^{\dagger}$, where $N^{\dagger}\ \geq\ N^{*}$, the
length of the cleaned-up strings. Since from this expansion, the $execute$ algorithm simply linearly
goes through the string and executes all valid tokens.

\begin{center}
    This gives us an overarching \textbf{O(N$^{\dagger}$)} complexity.
\end{center}

\end{proof}

\begin{proposition}
\label{numq}
The $replaceAction()$ algorithm will have time and space complexity of \textbf{O(N + (N$^{\ddagger}$+k+m))}
where:
\begin{itemize}
 \item \textbf{N:} ~ The length of the original string of tokens
 \item \textbf{N$^{\star}$:} ~ The length of the original string of tokens without spaces, $N^{*}\ \leq\ N$
 \item \textbf{n:} ~ The length of the token to be replaced
 \item \textbf{k:} ~ The number of occurences of the above token
 \item \textbf{m:} ~ The length of the tokens that will be doing the replacing
 \item \textbf{N$^{\ddagger}$:} ~ $N^{\star}\ -\ k*n$
\end{itemize}
\end{proposition}

\begin{proof}
This is the case since we will first go through the original to remove spaces, \textbf{O(N)}, making $N^{\star}$ the
linear search then we go throug and replace all of the occurances of the tokens to be replaced with the tokens to be
replacing. Note that if the tokens to be replaced are the longer ones, will not exceed \textbf{O(N$^{\star}$)}.
Otherwise the we have a string of length$N^{\ddagger}+k+m$ to build which is also a linear process, \textbf{O(N$^{\ddagger}$+k+m)}

\begin{center}
    Giving us an overarching \textbf{O(N + (N$^{\ddagger}$+k+m))} complexity.
\end{center}

\end{proof}


%%%%%%% Example %%%%%%%%%%%%%%%%%%%%%%%%%%%%%%%%%%%%%%%%%%%%%%%

\subsection{An Example}
We first note that only an example of the TurtleGraphics working is sufficient to
show that TurtleTokenizer works. For this example we entered the string:
 X3 \{F81 L120\}, this produces a triangle of side length 81. We then used the
 replace functionality with the command F81 -\textgreater\ F27 R60 F27 L120 F27 R60 F27
 to produce: X3 \{F27 R60 F27 L120 F27 R60 F27 L120\}. This produces a Koch snowflake
 of order 1.  We then employed the following replace commands to create a 4th order Koch
 snowflake:
 \begin{itemize}
   \item \textbf{2nd Order:} F27 -\textgreater\ F9 R60 F9 L120 F9 R60 F9
   \item \textbf{3rd Order:} F9 -\textgreater\ F3 R60 F3 L120 F3 R60 F3
   \item \textbf{4th Order:} F3 -\textgreater\ F1 R60 F1 L120 F1 R60 F1
 \end{itemize}
 \textbf{\underline{Resulting in:}} \\
 X3 \{F1 R60 F1 L120 F1 R60 F1 R60 F1 R60 F1 L120 F1 R60 F1 L120 F1 R60 F1
 L120 F1 R60 F1 R60 F1 R60 F1 L120 F1 R60 F1 R60 F1 R60 F1 L120 F1 R60 F1 R60 F1 R60 F1
 L120 F1 R60 F1 L120 F1 R60 F1 L120 F1 R60 F1 R60 F1 R60 F1 L120 F1 R60 F1 L120 F1 R60
 F1 L120 F1 R60 F1 R60 F1 R60 F1 L120 F1 R60 F1 L120 F1 R60 F1 L120 F1 R60 F1 R60 F1 R60
 F1 L120 F1 R60 F1 R60 F1 R60 F1 L120 F1 R60 F1 R60 F1 R60 F1 L120 F1 R60 F1 L120 F1 R60
 F1 L120 F1 R60 F1 R60 F1 R60 F1 L120 F1 R60 F1 R60 F1 R60 F1 L120 F1 R60 F1 R60 F1 R60
 F1 L120 F1 R60 F1 L120 F1 R60 F1 L120 F1 R60 F1 R60 F1 R60 F1 L120 F1 R60 F1 R60 F1 R60
 F1 L120 F1 R60 F1 R60 F1 R60 F1 L120 F1 R60 F1 L120 F1 R60 F1 L120 F1 R60 F1 R60 F1 R60
 F1 L120 F1 R60 F1 L120 F1 R60 F1 L120 F1 R60 F1 R60 F1 R60 F1 L120 F1 R60 F1 L120 F1 R60
 F1 L120 F1 R60 F1 R60 F1 R60 F1 L120 F1 R60 F1 R60 F1 R60 F1 L120 F1 R60 F1 R60 F1 R60
 F1 L120 F1 R60 F1 L120 F1 R60 F1 L120 F1 R60 F1 R60 F1 R60 F1 L120 F1 R60 F1 L120 F1 R60
 F1 L120 F1 R60 F1 R60 F1 R60 F1 L120 F1 R60 F1 L120 F1 R60 F1 L120 F1 R60 F1 R60 F1 R60
 F1 L120 F1 R60 F1 R60 F1 R60 F1 L120 F1 R60 F1 R60 F1 R60 F1 L120 F1 R60 F1 L120 F1 R60
 F1 L120 F1 R60 F1 R60 F1 R60 F1 L120 F1 R60 F1 L120 F1 R60 F1 L120 F1 R60 F1 R60 F1 R60
 F1 L120 F1 R60 F1 L120 F1 R60 F1 L120 F1 R60 F1 R60 F1 R60 F1 L120 F1 R60 F1 R60 F1 R60
 F1 L120 F1 R60 F1 R60 F1 R60 F1 L120 F1 R60 F1 L120 F1 R60 F1 L120 F1 R60 F1 R60 F1 R60
 F1 L120 F1 R60 F1 R60 F1 R60 F1 L120 F1 R60 F1 R60 F1 R60 F1 L120 F1 R60 F1 L120 F1 R60
 F1 L120 F1 R60 F1 R60 F1 R60 F1 L120 F1 R60 F1 R60 F1 R60 F1 L120 F1 R60 F1 R60 F1 R60 F1
 L120 F1 R60 F1 L120 F1 R60 F1 L120 F1 R60 F1 R60 F1 R60 F1 L120 F1 R60 F1 L120 F1 R60 F1
 L120 F1 R60 F1 R60 F1 R60 F1 L120 F1 R60 F1 L120 F1 R60 F1 L120 F1 R60 F1 R60 F1 R60 F1
 L120 F1 R60 F1 R60 F1 R60 F1 L120 F1 R60 F1 R60 F1 R60 F1 L120 F1 R60 F1 L120 F1 R60 F1
 L120 F1 R60 F1 R60 F1 R60 F1 L120 F1 R60 F1 L120\}
 \\ \\
 This is sufficient to show that the Turtle algorithms all work since when we press run
 on the ui, it will execute the given code correctly, meaning that the execute code is
 functioning correctly and therefore also the Tokenizer class.


%%%%%%% end Turtle Solution %%%%%%%%%%%%%%%%%%%%%%%%%%%%%%%%%%%%%%%%%
\newpage
%%%%%%% begin Anagrams Problem %%%%%%%%%%%%%%%%%%%%%%%%%%%%%%%%%%%%%%

\section{Anagrams Problem}
Two strings are anagrams of each other if they contain the same characters with the
same frequencies: “stops” and “psost” are anagrams, while “stops” and “stoop”
are not. (It’s not necessary that the permuted strings be English words.)

Your program will read a text file, with its name specified as a command line argument,
that contains one string of characters per line. For each line of the file, your program must
print all anagrams of the string in alphabetical order to standard output, with any repetitions
removed. The output from different lines should be separated by a blank line.


\bigskip


\bigskip

\noindent

%%%%%%% end Problem %%%%%%%%%%%%%%%%%%%%%%%%%%%%%%%%%%%%%%%%%%%%%%%%

\newpage

%%%%%%% Mathematical Formulation %%%%%%%%%%%%%%%%%%%%%%%%%%%%%%%%%%%%%%%%%%%%%%%

\subsection{Mathematical Formulation}
In this problem, we will be given a word of length $N$ consisting of $k$ unique letters.
We will find $M$ anagrams of this word, $M$ can be represented in terms of $N$ as:
  $\frac{N!}{n_0!\cdot n_1!\cdot ...\cdot n_k!}$ where $n_i$ = number of repetitions
  for each letter s.t. $n_0+n_1+...+n_k$ = {\em N} \\ $(Note\ that\ n_i\ may\ =\ 1)$

%%%%%%% Algorithm %%%%%%%%%%%%%%%%%%%%%%%%%%%%%%%%%%%%%%%%%%%%%%%

\subsection{Solution}

Our algorithm for finding all anagrams in a list of words is accomplished
through several steps. We look at one word at a time, checking its length, $N$.
If $N$ \textgreater\ 1, this is the trivial case, if not we begin to enumerate
the Anagrams.

\begin{algorithm}[H]
\caption{Initialization Step.}
\begin{algorithmic}
    \Procedure{Anagrams}{$fileName$}
        \While{$fileName$ contains at least 1 word}
            \State $word \gets$ $fileName$
            \If{$word$.length \textless 2}
                \State print (word)
            \Else
                \State $anagrams$ = Initialize $HashSet$
                \State \Call{findAnagrams}{$word$}
            \EndIf
        \EndWhile
    \EndProcedure
\end{algorithmic}
\end{algorithm}

Begin by sorting the word and enumerating the anagrams via a $swap$ method.
$Note: We\ do\ not\ enumerate\ these\ in\ lexiographical\ order,\ see\ below\ for\ details$.
Once all of the anagrams have been found, we move onto the next phase which would be to take
all of the anagrams and sort them so that they are in alphabetical order.

\begin{algorithm}[H]
\caption{Finding Anagrams}
\begin{algorithmic}
    \Procedure{findAnagrams}{$word$}
        \State sort and input \big($word$\big)  into $hashset$
        \For{$c_0 \in word$}
            \For{$c_1 \in word \setminus \{previous characters\}$}
                \If{$c_0\ equals\ c_1$}
                    \State CONTINUE
                \Else
                    \State \Call{swap}{$word, c_0, c_1$}
                \EndIf
            \EndFor
        \EndFor
        \State \Call{sort}{$anagrams$}
        \State \Call{print}{$anagrams$}
    \EndProcedure
\end{algorithmic}
\end{algorithm}

{\em Swap} makes recursive calls checking to see if the permutation of
the word has already been found via a hashset, inserting them if they
are not contained within the set.

\begin{algorithm}[H]
\caption{Swaping Characters}
\begin{algorithmic}
    \Procedure{swap}{$word, c_0, c_1$}
        \State $newWord \gets$ Switch characters $c_0$ and $c_1$ within $word$
        \If{$newWord \notin anagrams$}
            \State anagrams.\Call{insert}{$newWord$}
        \EndIf
        \State \Call{increment}{$c_0\ and\ c_1$}
        \For{$c_0 \in newWord$}
            \For{$c_1 \in newWord \setminus \{previous characters\}$}
                \If{$c_0\ equals\ c_1$}
                    \State CONTINUE
                \Else
                    \State \Call{swap}{$newWord, c_0, c_1$}
                \EndIf
            \EndFor
        \EndFor
    \EndProcedure
\end{algorithmic}
\end{algorithm}

%%%%%%% Correctness %%%%%%%%%%%%%%%%%%%%%%%%%%%%%%%%%%%%%%%%%%%%%%%

\subsection{Correctness}

\begin{proposition}
This algorithm with effectively find and print out all of the $M$ Anagrams of a word of length $N$ consisting
of $k$ unique letters, the frequency of each can be represented as $n_i$ s.t. $n_0+n_1+...+n_k$ = $N$.
\end{proposition}

\begin{proof}
We do this proof inductively. \\ \\
\textbf{\underline{Base Cases:}} \\
For $N$ = 1, (i.e. word is "a"), this is trivial, so look instead at $N$ = 2. Here we have 2 cases:
\begin{enumerate}
 \item \textbf{k = 2}, word = "ab"
 \item \textbf{k = 1}, word = "aa"
\end{enumerate}
In \textbf{case 1}, the algorithm would sort first to "\textbf{ab}" and store that, then swap the first letter, 'a', with the next
distinct letter, 'b', producing "\textbf{ba}". This would end the program as there are no other swaps to be made.
In \textbf{case 2}, the algorithm would sort first to "\textbf{aa}" and store that, then see that the two characters are the
same so it would end the program as there are no other swaps to be made.

The next step would be $N$ = 3 where we have the following cases:
\begin{enumerate}
  \item \textbf{k = 3}, word = "abc"
  \item \textbf{k = 2}, word = "aab"
  \item \textbf{k = 1}, word = "aaa"
\end{enumerate}
In \textbf{case 1}, the algorithm would sort first to "\textbf{abc}" and store that, then swap the first
letter with the next distinct letter, 'b', producing "\textbf{bac}". Now 'b' is an anchor and so we will look to
enumerate all anagrams of the remainder, putting 'b' in front. Note that the remainder is like that of
\textbf{case 1} when $N=2$ so the resulting product would be "\textbf{bca}" since "\textbf{bac}" was found already.
Now we return to  the original sorted string "\textbf{abc}" and swap 'a' with the next unique letter, 'c', producing
"\textbf{cba}". Now 'c' is the anchor and for the same reasoning as previous, we see that the string "\textbf{cab}"
is generated so we now return to the original sorted string and attempt to swap 'a' with the next unique letter.
Since there are no more, we anchor 'a' and now attempt to swap 'b' with its next unique letter, 'c'. This produces
"\textbf{acb}" and would end this portion of the algorithm as there are no other swaps to be made. Now that all of
the possible combinations have been found, the program will sort all of the found strings and print them out. \\
    \indent In \textbf{case 2}, the algorithm would sort first to "\textbf{aab}" and store that, then swap the first
letter with the next distinct letter, 'b', producing "\textbf{baa}". Now 'b' is an anchor and so we will look to
enumerate all anagrams of the remainder, putting a 'b' in front. Note that the remainder is like that of
\textbf{case 2} when $N=2$ so there are no more strings to enumerate so we return to the original sorted string,
"\textbf{aab}". Now the program will swap the next letter, 'a', with the next distinct letter, 'b'.  This produces
the string "\textbf{aba}".  Since now 'b' is the anchor and the second 'a' has no proceeding letters, the process
ends as all of the possible combinations have been found, the program will sort all of the found strings and
print them out. \\
    \indent In \textbf{case 3}, the algorithm would sort first to "\textbf{aaa}" and recognize this to be a trivial
case so it will print out the found string.  \\

\noindent \textbf{\underline{Assumption:}} \\
Let us have a word (when sorted) be $n_{1_1}$$n_{1_2}$...$n_{1_i}$$n_{2_1}$$n_{2_2}$...$n_{j_1}$...$n_{k_1}$$n_{k_2}$..$n_{k_p}$
s.t. $n_{l_1} + ... + n_{l_m}$ = $n_1$ and $n_1+...+n_k$ = $N$ for $i,j,k,l,m \in \mathbb{N}$. We assume that
we can find every anagram for this form of word. \\

\noindent \textbf{\underline{{\em k+1} Step:}} \\
Now we use the same logic as we did in the cases of the $N=3$ stage when we had "anchor" substrings so that way we
enumerate first all of the strings that begin with $n_{2_1}$, then $n_{2_2}$, ..., then finally $n_{{k+1}_1}$ and
$n_{1_1}$. Therefore the algorithm will progress anchoring also the next term, up until there are only have 3 terms not
anchored which we saw from the base cases will be fully enumerated. Therefore the algorithm continues swapping in
letters until all cases have been found.





\end{proof}

%%%%%%% Analysis %%%%%%%%%%%%%%%%%%%%%%%%%%%%%%%%%%%%%%%%%%%%%%%

\subsection{Analysis}

\begin{proposition}
\label{numq}
Our algorithm will have time and space complexity in respect to the number of anagrams $M$ that can be derived
from a word of length $N$ with $k$ unique letters such that $n_0+n_1+...+n_k$ = $N$.
(Recall: $M$ = $\frac{N!}{n_0!\cdot n_1!\cdot ...\cdot n_k!}$). We say that this
algorithm has: \begin{center}\textbf{O(M$\cdot$ log(M))} complexity.\end{center}
\end{proposition}

\begin{proof}
We begin this proof by stating that it is impossible to get better than O($M$) since every possible enumeration
must be explored and worst case is that there are no repeating letters so this would be $N$! outcomes so an $N$!
algorithm must be used. This being said, our algorithm will only enumerate each possible outcome once and due
to fact that the senario where two of the same letters are swapped which would lead to an anagram that has already been
found has been eliminated, our algorithm performs at the afformentioned $M$ complexity for finding the anagrams. After
we have found all of the anagrams, we now must put them in alphabetic order, which we do through linearly
copying them into an array, followed by sorting them, and finally printing them. Therefore we can list the
complexities as such:
\begin{enumerate}
    \item \textbf{Finding Anagrams:} O($M$) ~ Enumerating all possible $M$ anagrams
    \item \textbf{Copying:} O($M$)
    \item \textbf{Sorting:} O($M\cdot log(M)$) ~ Sorting all $M$ found anagrams
    \item \textbf{Printing:} O($M$)
\end{enumerate}

\begin{center}
    Giving us an overarching \textbf{O(M$\cdot$ log(M))} complexity.
\end{center}

\end{proof}




%%%%%%% Example %%%%%%%%%%%%%%%%%%%%%%%%%%%%%%%%%%%%%%%%%%%%%%%

\subsection{An Example}
Say that we input the simple word \textbf{jeep}. Before beginning tracing, let us first observe that this word has
$N$ = 4 letters, which implies that it has $M$ = $\frac{4!}{1!\cdot 2!\cdot 1!}$ = 12 anagrams. to solve, first the
algorithm would initialize our HashSet with the sorted form of the word \textbf{eejp}:

\begin{center}
\begin{tabular}{ |c|c| }
    \hline
    \multicolumn{2}{|c|}{HashSet} \\
    \hline
    \textbf{Stage-Found} & \textbf{Anagram} \\
    \hline
    $sorted$ & eejp \\
    \hline
\end{tabular}
\end{center}

Now the \textbf{FindAnagrams} portion beings, pointers $i_0$ and $j_0$ = 0 and 1 respectively
so that they are pointing to \textbf{e} and \textbf{e}. Since they are equal, $j$ increments
so that $i_0 = 0,\ j_0 = 2$ \\ \indent i.e. $i_0$ = first \textbf{e} and $j_0$ = \textbf{j} \\ These two
letters are not the same so we call $swap$ \\
\indent Within $S_1$ = $swap$($eejp$, 0, 2); we swap \textbf{e} and \textbf{j} to get \textbf{jeep}, since
this is not in the HashSet, we add it.

\begin{center}
\begin{tabular}{ |c|c| }
    \hline
    \multicolumn{2}{|c|}{HashSet} \\
    \hline
    \textbf{Stage-Found} & \textbf{Anagram} \\
    \hline
    $sorted$ & eejp \\
    \hline
    $S_1$ & jeep \\
    \hline
\end{tabular}
\end{center}

Now we will attempt to recursively call swap, setting $i_1$ = 1 and $j_1$ = 2. since our word now is \textbf{jeep},
we see that the letters are again both $i_1$ and $j_1$ points to \textbf{e} so increment $j_1$ = 3 now.  Our next
call of $swap$ therefore will be $S_2$ =  $swap$($jeep$, 1, 3); resulting in the word \textbf{jpee}, since not in the
HashSet, we add it.

\begin{center}
\begin{tabular}{ |c|c| }
    \hline
    \multicolumn{2}{|c|}{HashSet} \\
    \hline
    \textbf{Stage-Found} & \textbf{Anagram} \\
    \hline
    $sorted$ & eejp \\
    \hline
    $S_1$ & jeep \\
    \hline
    $S_2$ & jpee \\
    \hline
\end{tabular}
\end{center}

Now will attempt to recursively call swap again, setting $i_2$ and $j_2$ to 2, and 3 respectively, seeing again that
both $i_1$ and $j_1$ points to \textbf{e} but since we have reached the end of the for loop, $S_2$ call will end.

Returning to $S_1$ (recall that the word is \textbf{jeep}, $i_1$ = 1 and $j_1$ = 3). We continue in the loop
such that now  $i_1$ = 2, $j_1$ = 3. Therefore $i_2$ points to the second \textbf{e} and $j_0$ to \textbf{p}. Swapping
them through calling: $S_3$ = $swap$($jeep$, 2, 3); results in the word \textbf{jepe}, not in the HashMap so we add it. \\
\indent \textit{Note: the for loop will not be able to execute here so $S_3$ terminates here.}

\begin{center}
\begin{tabular}{ |c|c| }
    \hline
    \multicolumn{2}{|c|}{HashSet} \\
    \hline
    \textbf{Stage-Found} & \textbf{Anagram} \\
    \hline
    $sorted$ & eejp \\
    \hline
    $S_1$ & jeep \\
    \hline
    $S_2$ & jpee \\
    \hline
    $S_3$ & jepe \\
    \hline
\end{tabular}
\end{center}

Returning to $S_1$ (recall that the word is \textbf{jeep}, $i_1$ = 2 and $j_1$ = 3) we have ended the for loop now so
terminate $S_1$ returning to the original $sorted$ state.
\\ \indent \textit{\underline{Note}: we have successfully found all anagrams of \textbf{jeep} beginning with \textbf{j}!}
\\ Recall  the word now is \textbf{eejp} and $i_0 = 0,\ j_0 = 2$, continuing in the for loop we get,
$i_0 = 0,\ j_0 = 3$, so we call $S_4$ =  $swap$($eejp$, 0, 3); resulting in the word \textbf{peje}. From
here much like before, we shall find all anagrams of \textbf{jeep} beginning with \textbf{p} this time, finally followed
by all anagrams beginning with \textbf{e} so our HashSet will look like:

\begin{center}
\begin{tabular}{ |c|c| }
    \hline
    \multicolumn{2}{|c|}{HashSet} \\
    \hline
    \textbf{Stage-Found} & \textbf{Anagram} \\
    \hline
    $sorted$ & eejp \\
    \hline
    $S_1$ & jeep \\
    \hline
    $S_2$ & jpee \\
    \hline
    $S_3$ & jepe \\
    \hline
    $S_4$ & peje \\
    \hline
    $S_5$ & pjee \\
    \hline
    $S_6$ & peej \\
    \hline
    $S_7$ & ejep \\
    \hline
    $S_8$ & ejpe \\
    \hline
    $S_9$ & epje \\
    \hline
    $S_{10}$ & epej \\
    \hline
    $S_{11}$ & eepj \\
    \hline
\end{tabular}
\end{center}

Now that we have reached the end of the enumerating stage, we come to sorting the above HashSet and printing all of
them out.

%%%%%%% end Solution %%%%%%%%%%%%%%%%%%%%%%%%%%%%%%%%%%%%%%%%%%%%%%%

\end{document}