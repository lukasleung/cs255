% Use this template to write your solutions to COS 423 problem sets

\documentclass[12pt]{article}
\usepackage[utf8]{inputenc}
\usepackage{amsmath, amsfonts, amsthm, amssymb, algorithm, graphicx, mathtools, xfrac}
\usepackage[noend]{algpseudocode}
\usepackage{fancyhdr, lastpage}
\usepackage{booktabs}
\usepackage{multirow}
\usepackage[vmargin=1.20in,hmargin=1.25in,centering,letterpaper]{geometry}
\setlength{\headsep}{.50in}
\setlength{\headheight}{15pt}

% Landau notation
\DeclareMathOperator{\BigOm}{\mathcal{O}}
\newcommand{\BigOh}[1]{\BigOm\left({#1}\right)}
\DeclareMathOperator{\BigTm}{\Theta}
\newcommand{\BigTheta}[1]{\BigTm\left({#1}\right)}
\DeclareMathOperator{\BigWm}{\Omega}
\newcommand{\BigOmega}[1]{\BigWm\left({#1}\right)}
\DeclareMathOperator{\LittleOm}{\mathrm{o}}
\newcommand{\LittleOh}[1]{\LittleOm\left({#1}\right)}
\DeclareMathOperator{\LittleWm}{\omega}
\newcommand{\LittleOmega}[1]{\LittleWm\left({#1}\right)}

% argmin and argmax
\newcommand{\argmin}{\operatornamewithlimits{argmin}}
\newcommand{\argmax}{\operatornamewithlimits{argmax}}

\newcommand{\calP}{\mathcal{P}}
\newcommand{\Z}{\mathbb{Z}}
\newcommand{\R}{\mathbb{R}}
\newcommand{\Exp}{\mathbb{E}}
\newcommand{\Q}{\mathbb{Q}}
\newcommand{\sign}{\mathrm{sign\ }}
\newcommand{\abs}{\mathrm{abs\ }}
\newcommand{\eps}{\varepsilon}
\newcommand{\zo}{\{0, 1\}}
\newcommand{\SAT}{\mathit{SAT}}
\renewcommand{\P}{\mathbf{P}}
\newcommand{\NP}{\mathbf{NP}}
\newcommand{\coNP}{\co{NP}}
\newcommand{\co}[1]{\mathbf{co#1}}
\renewcommand{\Pr}{\mathop{\mathrm{Pr}}}

% theorems, lemmas, invariants, etc.
\newtheorem{theorem}{Theorem}
\newtheorem{lemma}[theorem]{Lemma}
\newtheorem{invariant}[theorem]{Invariant}
\newtheorem{corollary}[theorem]{Corollary}
\newtheorem{definition}[theorem]{Definition}
\newtheorem{property}[theorem]{Property}
\newtheorem{proposition}[theorem]{Proposition}

% piecewise functions
\newenvironment{piecewise}{\left \{\begin{array}{l@{,\ }l}}
{\end{array}\right.}

% paired delimiters
\DeclarePairedDelimiter{\ceil}{\lceil}{\rceil}
\DeclarePairedDelimiter{\floor}{\lfloor}{\rfloor}
\DeclarePairedDelimiter{\len}{|}{|}
\DeclarePairedDelimiter{\set}{\{}{\}}

\makeatletter
\@addtoreset{equation}{section}
\makeatother
\renewcommand{\theequation}{\arabic{section}.\arabic{equation}}

% algorithms
\algnewcommand\algorithmicinput{\textbf{INPUT:}}
\algnewcommand\INPUT{\item[\algorithmicinput]}
\algnewcommand\algorithmicoutput{\textbf{OUTPUT:}}
\algnewcommand\OUTPUT{\item[\algorithmicoutput]}


% Formating Macros

\pagestyle{fancy}
\lhead{\sc \hmwkClass\ $\; \;\cdot \; \;$ \hmwkSemester\ $\; \;\cdot \; \;$
Problem \hmwkAssignmentNum.\hmwkProblemNum}
%\chead{\sc Problem \hmwkAssignmentNum.\hmwkProblemNum}
%\chead{}
\rhead{\em \hmwkAuthorName\ $($\hmwkAuthorID$)$\/}
\cfoot{}
\lfoot{}
\rfoot{\sc Page\ \thepage\ of\ \protect\pageref{LastPage}}
\renewcommand\headrulewidth{0.4pt}
\renewcommand\footrulewidth{0.4pt}

\fancypagestyle{fancycollab}
{
    \lfoot{\textit{Collaborators: \hmwkCollaborators}}
}

\fancypagestyle{problemstatement}
{
    \rhead{}
    \lfoot{}
}

%%%%%% Begin document with header and title %%%%%%%%%%%%%%%%%%%%%%%%%

\begin{document}

%%%%%% Header Information %%%%%%%%%%%%%%%%%%%%%%%%%%%%%%%%%%%%%%%%%%%

%%% Shouldn't need to change these
\newcommand{\hmwkClass}{CSCI 255}
\newcommand{\hmwkSemester}{Spring 2016}

%%% Your name, in standard First Last format
\newcommand{\hmwkAuthorName}{Annamalis \& Lukas}
%%% Your NetID
\newcommand{\hmwkAuthorID}{asharp,lleung}

%%% The problem set number (just the number)
\newcommand{\hmwkAssignmentNum}{1}

%%% The problem number (just the number)
\newcommand{\hmwkProblemNum}{2}

%%% A list of your collaborators' NetIDs, separated by ", ".
%%% You can use a new line ("\\") in the middle to prevent a long
%%% list from overflowing.
\newcommand{\hmwkCollaborators}{}
%%% Sets the collaborator list to appear on the first page
\thispagestyle{fancycollab}

%%%%%%% begin Solution %%%%%%%%%%%%%%%%%%%%%%%%%%%%%%%%%%%%%%%%%%%%%

\noindent
{\em This document contains test reports of all fellow classmates and explanations
of motivations behind each test. Tests will be for correctness, memory usage and
timing when applicable.}

%%%%%%% begin Anagrams Testing %%%%%%%%%%%%%%%%%%%%%%%%%%%%%%%%%%%%%
\section{Anagrams}
\subsection{Correctness:}
To test for correctness we had seven test cases:
\begin{table}[H]
	\centering
	\begin{tabular}{cc}
		\toprule
		Case & String \\
		\midrule
		1 & \texttt{``a''} \\
		2 & \texttt{``abc''} \\
		3 & \texttt{``aaa''} \\
		4 & \texttt{``cabcab''} \\
		5 & \texttt{``annamalis''} \\
		6 & \texttt{``aaaaaaaabb''} \\
		7 & \texttt{``qwertyuiop''} \\
		\bottomrule
	\end{tabular}
\end{table}
\noindent \underline{The reasoning behind each case is as follows:} \\
\em{Note: For the last 3 cases, we know that the java HeapSpace = 2,097,152 bytes and that $10!$ = 3,628,800
    where as $9!$ = 362,880.}
\begin{enumerate}
\item Base case to see if the algorithm will print out just a single string.
\item Just a simple input meant to easily check for correctness and to see if the
algorithm is overkill when it comes to time complexity and/or space usage.
\item This is meant to test and see if the student's code can handle a string of all
the same character, yet still be an easy case.
\item Tests students ability to handle unsorted and longer strings with repeating letters.
\item We use ''annamalis'' so as to test the 9 character correctness aspect. If the students
were to use all of the space and store repeats, it would not time out in the system, but it
would show up in our space analysis section where as if they did not store the extra
strings, it would also show in the space analysis.
\item If students ignore all of the repetitions, they would pass this case since it would only be $\frac{10!}{8!\cdot 2!}$ = 45 strings being stored.
\item Expect any algorithm which requires storage of all of the numbers to fail this case.
\end{enumerate}


\underline{The results were:}
\begin{table}[H]
	\centering
	\resizebox{\textwidth}{!}{
	\begin{tabular}{cccccccc}
		\toprule
		Group name        & Case 1 & Case 2 & Case 3 & Case 4 & Case 5 & Case 6 & Case 7 \\
		\midrule
		Lyle \& Derek     & Passed & Passed & Passed & Passed & Passed & Passed & Passed? \\
		Viet \& Christ    & Passed & Passed & Passed & Passed & Passed & Passed & \textbf{Failed} \\
		Jon \& Trung      & Passed & Passed & Passed & Passed & Passed & Passed & Passed? \\
		Colin \& Rafael   & Passed & Passed & Passed & Passed & Passed & Passed & \textbf{Failed}  \\
		Jiri \& Ronak     & Passed & Passed & Passed & Passed & Passed & Passed & \textbf{Failed}  \\
		Josh \& Parker    & Passed & Passed & Passed & Passed & Passed & Passed & Passed? \\
		Brenda \& Brendan & \textbf{Failed} & Passed & Passed & Passed & Passed & Passed & \textbf{Failed} \\
		Lukas \& Sharp    & Passed & Passed & Passed & Passed & Passed & Passed & \textbf{Failed} \\
		\bottomrule
	\end{tabular}}
\end{table}
The reason why we put ''Passed?'' for case 7 is because our program was unable to generate a
complete list to check for correctness, however when compared to eachouther's output, we found
them all to be the same. For everyone else who failed this case, it was because they had an
Out of Memory error caused by exceeding the heap space allotted for the program.
% TIMING
\subsection{Timing:}
\underline{The results were (seconds):}
\begin{table}[H]
    \centering
    \resizebox{\textwidth}{!}{
    \begin{tabular}{cccccccc}
        \toprule
        Group name        & Case 1 & Case 2 & Case 3 & Case 4 & Case 5 & Case 6 & Case 7 \\
        \midrule
        Lyle \& Derek     & \textbf{0.002} & \textbf{0.008} & \textbf{0.007} & \textbf{0.016} & \textbf{0.956} & \textbf{0.014} & \textbf{87.732} \\
        Viet \& Christ    & 0.308 & 0.196 & 0.088 & 0.133 & 1.384 & 0.080 & ---- \\
        Jon \& Trung      & 0.164 & 0.063 & 0.089 & 0.106 & 225.2 & 5.987 & 119.6 \\
        Colin \& Rafael   & 0.128 & 0.074 & 0.098 & 0.085 & 1.136 & 0.081 & ---- \\
        Jiri \& Ronak     & 0.189 & 0.097 & 0.091 & 0.119 & 1.920 & 0.124 & ---- \\
        Josh \& Parker    & 0.060 & 0.057 & 0.055 & 0.075 & 1.225 & 0.088 & 108.1 \\
        Brenda \& Brendan & 0.111 & 0.084 & 0.083 & 0.109 & 1.171 & 0.118 & ---- \\
        Lukas \& Sharp    & 0.076 & 0.080 & 0.080 & 0.124 & 2.152 & 0.091 & ---- \\
        \bottomrule
    \end{tabular}}
\end{table}
For this Section, Lyle and Derek had the most optimized code for Timing due to the fact that
they printed straight out onto the command line. Apart from this everyone else had very similar
run time with the exception of Jon and Trung's code. This was mostly due to the fact that they
consistently did checks for different cases whenever they encountered a new enumeration of the
word i.e. if every character is the same.

\newpage
% MEMORY
\subsection{Memory:}
\underline{The results were (percent used of total memory):}
\begin{table}[H]
    \centering
    \resizebox{\textwidth}{!}{
    \begin{tabular}{cccccccc}
        \toprule
        Group name        & Case 1 & Case 2 & Case 3 & Case 4 & Case 5 & Case 6 & Case 7 \\
        \midrule
        Lyle \& Derek     & \textbf{3.4\%} & \textbf{3.5\%} & \textbf{3.4\%} & \textbf{4.0\%} & \textbf{16.9\%} & \textbf{4.0\%} & \textbf{19.2\%} \\
        Viet \& Christ    & 5.0\% & 5.1\% & 5.0\% & 5.6\% & 39.0\% & 5.6\% & ---- \\
        Jon \& Trung      & 5.1\% & 5.1\% & 5.1\% & 6.2\% & 23.2\% & 24.2\% & 25.3\% \\
        Colin \& Rafael   & 5.1\% & 5.0\% & 5.1\% & 5.6\% & 40.6\% & 5.6\% & ---- \\
        Jiri \& Ronak     & 5.9\% & 5.9\% & 5.9\% & 7.0\% & 78.8\% & 7.0\% & ---- \\
        Josh \& Parker    & 5.1\% & 5.1\% & 5.1\% & 5.6\% & 24.2\% & 5.6\% & 25.3\% \\
        Brenda \& Brendan & 5.5\% & 5.5\% & 5.5\% & 6.1\% & 38.9\% & 6.1\% & ---- \\
        Lukas \& Sharp    & 5.2\% & 5.2\% & 5.2\% & 5.6\% & 40.5\% & 5.8\% & ---- \\
        \bottomrule
    \end{tabular}}
\end{table}
For this Section, Lyle and Derek had the most optimized code again for Memory due to the fact that
they printed straight out onto the command line. Apart from this everyone else had very similar
space usage with the exception of Jon and Trung's code when it came to case 6. We had expected to see a similar drop
in memory usage for case 6, but because their code stored extra information for each word Jon and Trung's
algorithm did significantly worse than all other groups.
\subsection{Summary:}
\begin{table}[H]
	\centering
	\resizebox{\textwidth}{!}{
	\begin{tabular}{cccccccc}
		\toprule
		Group name        & Case 1 & Case 2 & Case 3 & Case 4 & Case 5 & Case 6 & Case 7 \\
		\midrule
		Lyle \& Derek     & Passed & Passed & Passed & Passed & Passed & Passed & Passed \\
		Viet \& Christ    & Passed & Passed & Passed & Passed & Passed & Passed & \textbf{Failed} \\
		Jon \& Trung      & Passed & Passed & Passed & Passed & \textbf{Failed} & \textbf{Failed} & Passed \\
		Colin \& Rafael   & Passed & Passed & Passed & Passed & Passed & Passed & \textbf{Failed}  \\
		Jiri \& Ronak     & Passed & Passed & Passed & Passed & Passed & Passed & \textbf{Failed}  \\
		Josh \& Parker    & Passed & Passed & Passed & Passed & Passed & Passed & Passed \\
		Brenda \& Brendan & \textbf{Failed} & Passed & Passed & Passed & Passed & Passed & \textbf{Failed} \\
		Lukas \& Sharp    & Passed & Passed & Passed & Passed & Passed & Passed & \textbf{Failed} \\
		\bottomrule
	\end{tabular}}
\end{table}
The only groups to pass all cases were Lyle \& Derek and Josh \& Parker.
\newpage



%%%%%%% begin Tokenizer Testing %%%%%%%%%%%%%%%%%%%%%%%%%%%%%%%%%%%%
\section{Turtle Tokenizer}
\subsection{Correctness:}
To test for correctness we had seven test cases:
\begin{table}[H]
	\centering
	\begin{tabular}{cc}
		\toprule
		Number & Token Input \\
		\midrule
		1 & \texttt{``f10 R120''} \\
		2 & \texttt{``g10''} \\
		3 & \texttt{``x2\{f60x30\{10R60L10f\}g16\}''} \\
		4 & \texttt{``f60x30\{10R60L10f\}g16''} \\
		5 & \texttt{``10R60L10f''} \\
		6 & \texttt{``yolo''} \\
		7 & \texttt{``x2\{f60x30\{10R60L10f\}g16''} \\
		\bottomrule
	\end{tabular}
\end{table}
\noindent \underline{The reasoning behind each case is as follows:}
\begin{enumerate}
\item Base case to see if the algorithm will correctly parse two basic commands with camel case
\item An invalid entry which has numbers still to see if the program will output anything
\item Tests if the student's code can correctly return a multiplyer command with its
corresponding braces, with nested braces.
\item Tests the student's code ability to return multiple commands with a multiplyer and an
invalid entry
\item Tests if the code identifies a missing letter at the beginning and the precense of a
stand alone valid entry
\item Tests to see if students ignore invalid single letters but can pick up on valid single
letter entries
\item Tests to see if the students code is able to catch an illegal entry (missmatched brackets)
and throw the correct exception
\end{enumerate}
\newpage
\underline{The results were:}
\begin{table}[H]
	\centering
	\resizebox{\textwidth}{!}{
	\begin{tabular}{cccccccc}
		\toprule
		Group name        & Case 1 & Case 2 & Case 3 & Case 4 & Case 5 & Case 6 & Case 7 \\
		\midrule
		Lyle \& Derek     & \textbf{Failed} & \textbf{Failed} & \textbf{Failed} & \textbf{Failed} & \textbf{Failed} & \textbf{Failed} & \textbf{Failed} \\
		Viet \& Christ    & Passed & \textbf{Failed} & Passed & \textbf{Failed} & Passed & Passed & \textbf{Failed} \\
		Jon \& Trung      & Passed & \textbf{Failed} & \textbf{Failed} & \textbf{Failed} & \textbf{Failed} & \textbf{Failed} & Passed \\
		Colin \& Rafael   & Passed & Passed & Passed & Passed & \textbf{Failed} & Passed & \textbf{Failed}  \\
		Jiri \& Ronak     & Passed & \textbf{Failed} & Passed & \textbf{Failed} & \textbf{Failed} & \textbf{Failed} & \textbf{Failed}  \\
		Josh \& Parker    & Passed & \textbf{Failed} & \textbf{Failed} & \textbf{Failed} & \textbf{Failed} & \textbf{Failed} & \textbf{Failed} \\
		Brenda \& Brendan & Passed & \textbf{Failed} & Passed & \textbf{Failed} & \textbf{Failed} & \textbf{Failed} & \textbf{Failed} \\
		Lukas \& Sharp    & Passed & Passed & Passed & Passed & Passed & Passed & Passed \\
		\bottomrule
	\end{tabular}}
\end{table}
Lyle \& Derek print everything out two times due to a print statement within their code; if
commented out they would pass cases 1 and 3. They fail other cases because they return invalid
tokens and do not throw the correct exception when mismatched brackets are inputed.
$\Diamond$ Viet \& Christ fail because they return invalid tokens and do not throw the correct exception
when mismatched brackets are inputed.
$\Diamond$ Jon \& Trung fail because they are unable to handle invalid inputs and go into infinite loops
$\Diamond$ Colin \& Rafael fail case 5 because they could not seperate two tokens and failed the last case
becasue they do not throw the correct exception when mismatched brackets are inputed.
$\Diamond$ Jiri \& Ronak fail because they return invalid token tokens and do not throw the corrext exception
when mismatched brackets are inputed.
$\Diamond$ Josh \& Parker fail because they have Invalid Character Input Exception and String Index Out of
Bounds exceptions when there are invalid inputs.
$\Diamond$ Brenda \& Brendan fail because they return invalid tokens and do not throw the correct exception
when mismatched brackets are inputed.

% TIMING and MEMORY
\subsection{Timing \& Memory:}
For this file, there was little to no difference in run time or memory usage as all had around
0.001 seconds to run each case and took ~2.6\% memory
\newpage

%%%%%%% begin Graphics Testing %%%%%%%%%%%%%%%%%%%%%%%%%%%%%%%%%%%%%
\section{Turtle Graphics}
\subsection{Correctness:}
To test for correctness we had seven test cases:
\begin{table}[H]
	\centering
	\begin{tabular}{cc}
		\toprule
		Number & String \\
		\midrule
		1 & \texttt{``F3 -> F10''} \\
		2 & \texttt{``F1 -> X3\{F1\{R2L3\}\}''} \\
		3 & \texttt{``F1 F2''} \\
		4 & \texttt{``F -> ABC''} \\
		5 & \texttt{``F -> X''} \\
		6 & \texttt{``L -> R, R -> L''} \\
		\bottomrule
	\end{tabular}
\end{table}
\noindent \underline{We began by running the program with the order-3 Snowflake:} \\
X3 \{F3 R60 F3 L120 F3 R60 F3 R60 F3 R60 F3 L120 F3
R60 F3 L120 F3 R60 F3 L120 F3 R60 F3 R60 F3 R60 F3
L120 F3 R60 F3 R60 F3 R60 F3 L120 F3 R60 F3 R60 F3
R60 F3 L120 F3 R60 F3 L120 F3 R60 F3 L120 F3 R60 F3
R60 F3 R60 F3 L120 F3 R60 F3 L120 F3 R60 F3 L120 F3
R60 F3 R60 F3 R60 F3 L120 F3 R60 F3 L120 F3 R60 F3
L120 F3 R60 F3 R60 F3 R60 F3 L120 F3 R60 F3 R60 F3
R60 F3 L120 F3 R60 F3 R60 F3 R60 F3 L120 F3 R60 F3
L120 F3 R60 F3 L120 F3 R60 F3 R60 F3 R60 F3 L120 F3 R60 F3 L120\} \\
\underline{Then we did the above tests for the following reasos:}
\begin{enumerate}
\item Base case for basic correctness
\item Check for multiple tokens
\item Check for missing "-$\textgreater$" error
\item Check for swapping invalid tokens
\item check for "Missing command block" error when running
\item Check for swapping two tokens
\end{enumerate}

\noindent \textbf{Every team was able to pass every test.}

% TIMING and MEMORY
\subsection{Timing \& Memory:}
There was no way to accurately test for timing or memory usage.

\end{document}