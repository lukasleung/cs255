% java BurrowsWheeler - < inputs\dickens.txt | java MoveToFront - | java Huffman - > inputs\dickens_compressed.txt
% du -b dickens.txt | awk '{ print $1, "* 8" }' | bc
% Use this template to write your solutions to COS 423 problem sets

\documentclass[12pt]{article}
\usepackage[utf8]{inputenc}
\usepackage{amsmath, amsfonts, amsthm, amssymb, algorithm, graphicx, mathtools, xfrac}
\usepackage[noend]{algpseudocode}
\usepackage{fancyhdr, lastpage}
\usepackage{booktabs}
\usepackage{multirow}
\usepackage[vmargin=1.20in,hmargin=1.25in,centering,letterpaper]{geometry}
\setlength{\headsep}{.50in}
\setlength{\headheight}{15pt}

% Landau notation
\DeclareMathOperator{\BigOm}{\mathcal{O}}
\newcommand{\BigOh}[1]{\BigOm\left({#1}\right)}
\DeclareMathOperator{\BigTm}{\Theta}
\newcommand{\BigTheta}[1]{\BigTm\left({#1}\right)}
\DeclareMathOperator{\BigWm}{\Omega}
\newcommand{\BigOmega}[1]{\BigWm\left({#1}\right)}
\DeclareMathOperator{\LittleOm}{\mathrm{o}}
\newcommand{\LittleOh}[1]{\LittleOm\left({#1}\right)}
\DeclareMathOperator{\LittleWm}{\omega}
\newcommand{\LittleOmega}[1]{\LittleWm\left({#1}\right)}

% argmin and argmax
\newcommand{\argmin}{\operatornamewithlimits{argmin}}
\newcommand{\argmax}{\operatornamewithlimits{argmax}}

\newcommand{\calP}{\mathcal{P}}
\newcommand{\Z}{\mathbb{Z}}
\newcommand{\R}{\mathbb{R}}
\newcommand{\Exp}{\mathbb{E}}
\newcommand{\Q}{\mathbb{Q}}
\newcommand{\sign}{\mathrm{sign\ }}
\newcommand{\abs}{\mathrm{abs\ }}
\newcommand{\eps}{\varepsilon}
\newcommand{\zo}{\{0, 1\}}
\newcommand{\SAT}{\mathit{SAT}}
\renewcommand{\P}{\mathbf{P}}
\newcommand{\NP}{\mathbf{NP}}
\newcommand{\coNP}{\co{NP}}
\newcommand{\co}[1]{\mathbf{co#1}}
\renewcommand{\Pr}{\mathop{\mathrm{Pr}}}

% theorems, lemmas, invariants, etc.
\newtheorem{theorem}{Theorem}
\newtheorem{lemma}[theorem]{Lemma}
\newtheorem{invariant}[theorem]{Invariant}
\newtheorem{corollary}[theorem]{Corollary}
\newtheorem{definition}[theorem]{Definition}
\newtheorem{property}[theorem]{Property}
\newtheorem{proposition}[theorem]{Proposition}

% piecewise functions
\newenvironment{piecewise}{\left \{\begin{array}{l@{,\ }l}}
{\end{array}\right.}

% paired delimiters
\DeclarePairedDelimiter{\ceil}{\lceil}{\rceil}
\DeclarePairedDelimiter{\floor}{\lfloor}{\rfloor}
\DeclarePairedDelimiter{\len}{|}{|}
\DeclarePairedDelimiter{\set}{\{}{\}}

\makeatletter
\@addtoreset{equation}{section}
\makeatother
\renewcommand{\theequation}{\arabic{section}.\arabic{equation}}

% algorithms
\algnewcommand\algorithmicinput{\textbf{INPUT:}}
\algnewcommand\INPUT{\item[\algorithmicinput]}
\algnewcommand\algorithmicoutput{\textbf{OUTPUT:}}
\algnewcommand\OUTPUT{\item[\algorithmicoutput]}


% Formating Macros

\pagestyle{fancy}
\lhead{\sc \hmwkClass\ $\; \;\cdot \; \;$ \hmwkSemester\ $\; \;\cdot \; \;$
Problem \hmwkAssignmentNum.\hmwkProblemNum}
%\chead{\sc Problem \hmwkAssignmentNum.\hmwkProblemNum}
%\chead{}
\rhead{\em \hmwkAuthorName\ $($\hmwkAuthorID$)$\/}
\cfoot{}
\lfoot{}
\rfoot{\sc Page\ \thepage\ of\ \protect\pageref{LastPage}}
\renewcommand\headrulewidth{0.4pt}
\renewcommand\footrulewidth{0.4pt}

\fancypagestyle{fancycollab}
{
    \lfoot{\textit{Collaborators: \hmwkCollaborators}}
}

\fancypagestyle{problemstatement}
{
    \rhead{}
    \lfoot{}
}

%%%%%% Begin document with header and title %%%%%%%%%%%%%%%%%%%%%%%%%

\begin{document}

%%%%%% Header Information %%%%%%%%%%%%%%%%%%%%%%%%%%%%%%%%%%%%%%%%%%%

%%% Shouldn't need to change these
\newcommand{\hmwkClass}{COS 255}
\newcommand{\hmwkSemester}{Spring 2016}

%%% Your name, in standard First Last format
\newcommand{\hmwkAuthorName}{Annamalis \& Lukas}
%%% Your NetID
\newcommand{\hmwkAuthorID}{asharp\&lleung}

%%% The problem set number (just the number)
\newcommand{\hmwkAssignmentNum}{1}

%%% The problem number (just the number)
\newcommand{\hmwkProblemNum}{0}

%%% A list of your collaborators' NetIDs, separated by ", ".
%%% You can use a new line ("\\") in the middle to prevent a long
%%% list from overflowing.
\newcommand{\hmwkCollaborators}{}
%%% Sets the collaborator list to appear on the first page
\thispagestyle{fancycollab}

%%%%%%% begin Solution %%%%%%%%%%%%%%%%%%%%%%%%%%%%%%%%%%%%%%%%%%%%

\noindent
Was only able to finish correctness for the Move-To-Front. Updated in this, however the correctness
proofs for Circular Suffix Array and Burrows Wheeler are not complete. Can complete at a further date.
{\em This document contains an explaination of algorithmic design and solutions to ... problems.
 First will be the statement of the question followed by the mathematical formulation, the
 Brute-Force solution, our solution, proof of correctness for our solution and any slight
 improvements we could do.}

%%%%%%% begin Move-To-Front Problem %%%%%%%%%%%%%%%%%%%%%%%%%%%%%%%

\section{Move-To-Front}
This algorithm has two parts: encoding and decoding. \\
\indent The main idea of move-to-front encoding is to maintain an ordered sequence of  the characters
in the alphabet, and repeatedly read in a character from the input message, print out the
position in which that character appears, and move that character to the front of the sequence.
The task is to maintain an ordered sequence of the 256 extended ASCII characters. Initialize
the sequence by making the ith character in the sequence equal to the ith extended ASCII
character. Now, read in each 8-bit character c from standard input one at a time, output the
8-bit index in the sequence where c appears, and move c to the front. \\
\indent The main idea of move-to-front decoding is to initialize an ordered sequence of 256 characters,
where extended ASCII character i appears ith in the sequence. Now, read in each 8-bit character
i (but treat it as an integer between 0 and 255) from standard input one at a time, write the
ith character in the sequence, and move that character to the front. Check that the decoder
recovers any encoded message.

%%%%%%% end Problem %%%%%%%%%%%%%%%%%%%%%%%%%%%%%%%%%%%%%%%%%%%%%%%


%%%%%%% Mathematical Formulation %%%%%%%%%%%%%%%%%%%%%%%%%%%%%%%%%%
\subsection{Mathematical Formulation}
In the Move-To-Front algorithm we will be taking an input of size $N$ characters and either
encoding or decoding them using an alphabet (standard ASCI) of size $R = 256$.

%%%%%%% Algorithm %%%%%%%%%%%%%%%%%%%%%%%%%%%%%%%%%%%%%%%%%%%%%%%%%

\subsection{Solution}
Move-To-Front $encode$( ) initializes a Linked List of all the asci-characters.
Then each character is read in, call it c, from standard input one at a time.
The character is then outputed as its correspoding 8-bit index as it appears in
the sequence. Once this is done the character is removed from it's current spot
and is moved to the front of the Linked List.

\begin{algorithm}[H]
\caption{Move-To-Front Encode}
\begin{algorithmic}
    \Procedure{encode}{ }
        \State $alphabet \gets$ Character LinkedList storing all 256 asci-characters
        \While{there is still input}
            \State $c \gets$ the next character
            \State $index \gets$ the current index of c in the alphabet
            \State alphabet.\Call{remove}{index}
            \State output the 8-bit index in sequence where c appears
            \State alphabet.\Call{addFirst}{c}
        \EndWhile
    \EndProcedure
\end{algorithmic}
\end{algorithm}

\noindent Move-To-Front $decode$( ) initializes a Linked List of all the asci-characters.
Then each character is read in, call it index, from standard input one at a time. Then
each one is written out as the ith character in the sequence, and move that character
to the front.

\begin{algorithm}[H]
\caption{Move-To-Front Decode}
\begin{algorithmic}
    \Procedure{decode}{ }
        \State $alphabet \gets$ Character LinkedList storing all 256 asci-characters
        \While{there is still input}
            \State $index \gets$ the index corresponding to stored char
            \State $c \gets$ alphabet.\Call{remove}{index} // the char from that index
            \State output c
            \State alphabet.\Call{addFirst}{c}
        \EndWhile
    \EndProcedure
\end{algorithmic}
\end{algorithm}


%%%%%%% Correctness %%%%%%%%%%%%%%%%%%%%%%%%%%%%%%%%%%%%%%%%%%%%%%%

\subsection{Correctness}

%%%%%%% PROPOSITION 1 %%%%%%%%%%%%%%%
\begin{proposition}
Our move-to-front $encode()$ and $decode()$ methods will produce the correct outputs required.
\end{proposition}

\begin{proof}
~ \\ \indent Without loss of generality (W.L.O.G.) let us say our alphabet (of length m) is $A_0$ for the initial state
and the string we read in can be represented as $s$ with length $n$. We will denote the given
index where $s_i$ in $A_i$ as $A_{s_i}$ and the resulting alphabet is $A_{i+1}$ where $A_i$ =
$A_{i+1}$ $\iff$ $s_i$ = $s_{i-1}$. \\ \indent We begin by finding the first character $s_0$ within
$A_0$ which gives us the number $A_{s_0}$ that will be the first to be outputted. We then note
that $s_0$ will be moved to the front, i.e. $A_{s_0} = 0$, giving us the new alphabet $A_1$.
$A_1$ will be identical to $A_0$ except all $A_{s_k}$ s.t. $k \textless s_0$ will be located at
$A_{s_k+1}$ and $A_{s_0} = 0$. We can generalize this with changing 0 to $i$ such that for every
input $s_i$, we change from $A_{i}$ to $A_{i+1}$ s.t. $A_{s_k}$ = $A_{s_k+1}$ $\forall$ $k
\textless s_i$ and $A_{s_i} = 0 \implies$ producing $A_{i+1} = \{A_{s_i}, A_{s_0}, A_{s_1},...
, A_{s_i-1}, A_{s_i+1}, A_{s_i+2},..., A_{s_{m-1}}\}$. Note here that $s_m$ where $m \neq i$ does not
correlate to the string, but the index of a symbol in $A_i$. \\
\indent This is the exact process that is asked to execute in the problem description and is executed
in both $encode()$ and $decode()$. The only difference between the two is: in $encode()$ we are taking a
sequence of characters, $s = s_0 s_1 ... s_{n-1}$ and outputting the corresponding $A_{s_i}$ for
each $s_i \in s$; in $decode()$ we are taking a sequence of character codes, $c = c_0 c_1 ... c_{n-1}$ and
outputting the character corresponding to the code. i.e. $c_0 = A_{s_0}, c_1 = A_{s_1}, ...c_{n-1} =  A_{s_{n-1}}$
and output $= s_0 s_1 ... s_{n-1} = s$
\end{proof}

%%%%%%% PROPOSITION 2 %%%%%%%%%%%%%%%
\begin{proposition}
The move-to-front $encode$ and $decode$ methods will accurately be able to encode
and decode the \textbf{same} message.
\end{proposition}

\begin{proof}
~ \\ \indent At both the encoding and decoding stages, we start out with the same unaltered linked
list of ASCI characters (represented as ints). As explained above, both will go through the exact
same process, however reversing their inputs and outputs. Therefore putting in the input
\end{proof}

%%%%%%% Analysis %%%%%%%%%%%%%%%%%%%%%%%%%%%%%%%%%%%%%%%%%%%%%%%%%%
\subsection{Analysis}
For the following analysis, we will say that $N$ is the size of the given input.
We also will use $R = 256$ to denote our alphabet size i.e. ASCI characters.

%%%%%%% PROPOSITION 1 %%%%%%%%%%%%%%%
\begin{proposition}
\label{numq}
The $encode()$ and $decode$ algorithms will both have time and space complexity in
proportion to \textbf{O(R + N)}

\end{proposition}
\begin{proof}
This is because for both, we begin by constructing our alphabet of all 256 Asci Characters \textbf{(R)}.
Then for each character input (\textbf{N} of them):
\begin{itemize}
    \item \textbf{O(1)} ~ get the current index of the character, initially this will be worst
                        case R, however after the first instance is found, the same letter
                        should take a next to constant operation to find.
    \item \textbf{O(1)} ~ write out the number (if encoding) or character (if decoding)
    \item \textbf{O(1)} ~ remove and move character to the front of the alphabet
\end{itemize}
Now as for the Memory usage, all we use is the character array of the input (\textbf{N}),
and the linked list of our alphabet (of size \textbf{R})
\begin{center}
    Giving us a complexity of \textbf{O(R + N)} for both time and memory.
\end{center}
\end{proof}

%%%%%%% Example %%%%%%%%%%%%%%%%%%%%%%%%%%%%%%%%%%%%%%%%%%%%%%%%%%%

\subsection{An Example}
Going through the example of ABRACADABRA! which would be inputted as: \\
ARD!RCAAAABB after going through the Circular Suffix Array. We note that there
are only 6 characters: \{!, A, B, C, D, R\}. I will go through how their ASCI
Characters change in the table below as well as their final output.

\begin{table}[H]
	\centering
	\resizebox{\textwidth}{!}{
	\begin{tabular}{c|c|cccccccccccc}
		\toprule
		\multicolumn{14}{c}{\textbf{Encoding}} \\ \hline
		                & Original    & \multicolumn{12}{c}{Index of Character After Seeing} \\
		Character       & Value       & A           & R           & D           & !           & R           & C           & A           & A           & A           & A           & B           & B \\
		\midrule
		!               & 21          & 22          & 23          & \textbf{24} & 00          & 01          & 02          & 03          & --          & --          & --          & 04          & 04 \\
        A               & \textbf{41} & 00          & 01          & 02          & 03          & 03          & \textbf{04} & \textbf{00} & \textbf{00} & \textbf{00} & 00          & 01          & 01 \\
        B               & 42          & --          & 43          & 44          & --          & --          & 45          & --          & --          & --          & \textbf{45} & \textbf{00} & 00 \\
        C               & 43          & --          & 44          & 45          & --          & \textbf{45} & 00          & 01          & --          & --          & --          & 02          & 02 \\
        D               & 44          & --          & \textbf{45} & 00          & 01          & 02          & 03          & 04          & --          & --          & --          & 05          & 05 \\
        R               & 52          & \textbf{52} & 00          & 01          & \textbf{02} & 00          & 01          & 02          & --          & --          & --          & 03          & 03 \\ \hline
        \textbf{Output} & \textbf{41} & \textbf{52} & \textbf{45} & \textbf{24} & \textbf{02} & \textbf{45} & \textbf{04} & \textbf{00} & \textbf{00} & \textbf{00} & \textbf{45} & \textbf{00} & -- \\
		\bottomrule
	\end{tabular}}
\end{table}

$\therefore$ Our encoded message is: \textbf{41 52 45 24 02 45 04 00 00 00 45 00} \\
\noindent To decode this, we will go through the same process except for instead of writting down a number
we will be writting down the correspoding character. Therefore we will have the following:

\begin{table}[H]
	\centering
	\resizebox{\textwidth}{!}{
	\begin{tabular}{c|cccccccccccc}
		\toprule
		\multicolumn{13}{c}{\textbf{Decoding}} \\ \hline
		Encoded String: & \textbf{41} & \textbf{52} & \textbf{45} & \textbf{24} & \textbf{02} & \textbf{45} & \textbf{04} & \textbf{00} & \textbf{00} & \textbf{00} & \textbf{45} & \textbf{00} \\
		Characters      & \multicolumn{12}{c}{Index of Characters} \\
		\midrule
		!               & 21          & 22          & 23          & \textbf{24} & 00          & 01          & 02          & 03          & --          & --          & --          & 04          \\
        A               & \textbf{41} & 00          & 01          & 02          & 03          & 03          & \textbf{04} & \textbf{00} & \textbf{00} & \textbf{00} & --          & 01          \\
        B               & 42          & --          & 43          & 44          & --          & --          & 45          & --          & --          & --          & \textbf{45} & \textbf{00} \\
        C               & 43          & --          & 44          & 45          & --          & \textbf{45} & 00          & 01          & --          & --          & --          & 02          \\
        D               & 44          & --          & \textbf{45} & 00          & 01          & 02          & 03          & 04          & --          & --          & --          & 05          \\
        R               & 52          & \textbf{52} & 00          & 01          & \textbf{02} & 00          & 01          & 02          & --          & --          & --          & 03          \\ \hline
        \textbf{Output} & \textbf{A}  & \textbf{R}  & \textbf{D}  & \textbf{!}  & \textbf{R}  & \textbf{C}  & \textbf{A}  & \textbf{A}  & \textbf{A}  & \textbf{A}  & \textbf{B}  & \textbf{B}  \\
		\bottomrule
	\end{tabular}}
\end{table}
$\therefore$ Our decoded message is: \textbf{ARD!RCAAAABB} so we have returned to our original string!

%%%%%%% end Move-To-Front Problem %%%%%%%%%%%%%%%%%%%%%%%%%%%%%%%%%


\newpage%%%%%%%%%%%%%%%%%%%%%%%%%%%%%%%%%%%%%%%%%%%%%%%%%%%%%%%%%%%%%%%%%%%%%%%%%%%%%%%%%%%%%%%%%%%%%%%%%%%%%%%%%%%%%%%%


%%%%%%% begin Circular Suffix Problem %%%%%%%%%%%%%%%%%%%%%%%%%%%%%

\section{Circular Suffix}
Given a string, this algorithm constructs a circular suffix array who's i$^{th}$ entry
is the index of the original suffix that appears i$^{th}$ in the sorted array.
Holistically, we will enumerate every suffix of the word in a circular fashion and sort
them, storing the index of where they had originally been. (see example)


%%%%%%% Mathematical Formulation %%%%%%%%%%%%%%%%%%%%%%%%%%%%%%%%%%
\subsection{Mathematical Formulation}
In the Circular Suffix algorithm we will be taking an input of size $N$ characters
and creat an array of size $N$ which will contain the original indicies ($charAt(i)$)
of the suffixes in sorted order. For reasons which will become appearent in the analysis
section, we will also note here that for the average word we will do $C^{\ast}$ compares
to determine the relationship between circular suffixes.

%%%%%%% Algorithm %%%%%%%%%%%%%%%%%%%%%%%%%%%%%%%%%%%%%%%%%%%%%%%%%

\subsection{Solution}
When we construct the Circular Suffix Array, we will first check to see that the
input is valid.  If it is then we will continue and make an array of indicies
s.t. index[i] = i.  From here we will sort these indicies based off of our own
comparator. This comparator will use the character array aspect of our input to
execute comparisons. Given 2 starting indexes, the algorithm will check to see
if the two characters are the same, if they are not then it will return the
appropriate negative or positive value.  If they are the same, then the two
pointers will advance (wrapping around to the front if the pointer should ever
leave the string) and repeat.  If at the end of all $N$ compares the strings are
identical, then the comparator will return 0 $\implies$ equal. At the end of this
the object will store the generated sorted circular suffix array so therefore
calls on length() and index() will be constant.

\begin{algorithm}[H]
\caption{Circular Suffix Array Construction}
\begin{algorithmic}
    \Procedure{init}{String $s$}
        \If{$s$ is null}
            \State throw NullPointerException()
        \EndIf
        \State $indicies \gets$ Filled out array [0..(N-1)]
        \State Arrays.\Call{sort}{indicies,comparator()} // see next method
    \EndProcedure
    \Procedure{compare}{Integer i1, Integer i2}
        \For{$i \in$ \{0..(N-1)\}}
            \State (char) $c1 \gets$ s[(i1 + i) \% s.length]
            \State (char) $c2 \gets$ s[(i2 + i) \% s.length]
            \If{$c1 \neq c2$}
                \State \Call{return}{c1-c2}
            \EndIf
            \State \Call{return}{0}
        \EndFor
    \EndProcedure
\end{algorithmic}
\end{algorithm}


%%%%%%% Correctness %%%%%%%%%%%%%%%%%%%%%%%%%%%%%%%%%%%%%%%%%%%%%%%

\subsection{Correctness}

%%%%%%% PROPOSITION 1 %%%%%%%%%%%%%%%
\begin{proposition}
The Circular Suffix Array will, given an input of size $N$, sort the sircular suffixes
of the input and return an array of their sorted indexes.

\end{proposition}

\begin{proof}
To come soon...
\end{proof}

%%%%%%% Analysis %%%%%%%%%%%%%%%%%%%%%%%%%%%%%%%%%%%%%%%%%%%%%%%%%%
\subsection{Analysis}
For the following analysis, we will say that $N$ is the size of the given input which
implies that we will 'generate' $N$ circular suffixes. We will also note here that for
the average word we will do $C^{\ast}$ compares. (i.e. there will not be a significant
time taken to make several comparisons and will act as the MSD sorting does.)

%%%%%%% PROPOSITION 1 %%%%%%%%%%%%%%%
\begin{proposition}
\label{numq}
The $construction$ algorithm will have both time complexity in
proportion to \textbf{O(N$\cdot$log(N)$\cdot$C$^{\ast}$)} and space
in \textbf{O(N)}.

\end{proposition}
\begin{proof}
We describe each part of the analysis as follows: it will take \textbf{O(N$\cdot$log(N))}
comparisons to sort the $N$ indicies using $Arrays.sort()$. This will cost \textbf{O(C$^{\ast}$)}
per comparison. However we can count this as nearly negligable in the typical case (being
words in the english dictionary). However the worst case would be a string that is uniform
in character i.e. 'aaaaaa...a' would produce a cost of $N$ every comparison. Though there
no such words in the english language. \\
\indent Now as for the Memory usage, we are only using the indicies array of size $N$,
the original string of size $N$ and two integer pointers in our $compare()$ method. This would
give us a grand total of $N + N + 2$ = \textbf{O(N)}
\begin{center}
    $\therefore$ Giving us a complexity of \textbf{O(N$\cdot$log(N)$\cdot$C$^{\ast}$)} for time \\
    and \textbf{O(N)} for memory.
\end{center}
\end{proof}

%%%%%%% Example %%%%%%%%%%%%%%%%%%%%%%%%%%%%%%%%%%%%%%%%%%%%%%%%%%%
\newpage
\subsection{An Example}
As an example, consider the string $''ABRACADABRA!''$ of length 12. The table below shows its
12 circular suffixes and the result of sorting them.

\begin{table}[H]
	\centering
	\resizebox{\textwidth}{!}{
	\begin{tabular}{c|cc|c}
		\toprule
		i  & Original Suffixes       & Sorted Suffixes         & index[i] \\ \hline
		\midrule
        0  & A B R A C A D A B R A ! & ! A B R A C A D A B R A & 11  \\
        1  & B R A C A D A B R A ! A & A ! A B R A C A D A B R & 10  \\
        2  & R A C A D A B R A ! A B & A B R A ! A B R A C A D & 7   \\
        3  & A C A D A B R A ! A B R & A B R A C A D A B R A ! & 0   \\
        4  & C A D A B R A ! A B R A & A C A D A B R A ! A B R & 3   \\
        5  & A D A B R A ! A B R A C & A D A B R A ! A B R A C & 5   \\
        6  & D A B R A ! A B R A C A & B R A ! A B R A C A D A & 8   \\
        7  & A B R A ! A B R A C A D & B R A C A D A B R A ! A & 1   \\
        8  & B R A ! A B R A C A D A & C A D A B R A ! A B R A & 4   \\
        9  & R A ! A B R A C A D A B & D A B R A ! A B R A C A & 6   \\
        10 & A ! A B R A C A D A B R & R A ! A B R A C A D A B & 9   \\
        11 & ! A B R A C A D A B R A & R A C A D A B R A ! A B & 2   \\
		\bottomrule
	\end{tabular}}
\end{table}


%%%%%%% end Circular Suffix Problem %%%%%%%%%%%%%%%%%%%%%%%%%%%%%%%
\newpage
%%%%%%% begin Burrows Wheeler Transform  %%%%%%%%%%%%%%%%%%%%%%%%%%%%%%%

\section{Burrows Wheeler Transform}
The goal of the Burrows-Wheeler transform is not to compress a message, but
rather to transform it into a form that is more amenable to compression. The
transform rearranges the characters in the input so that there are lots of
clusters with repeated characters, but in such a way that it is still possible
to recover the original input. It relies on the following intuition: if you
see the letters hen in English text, then most of the time the letter preceding
it is t or w. If you could somehow group all such preceding letters together
(mostly t's and some w's), then you would have an easy opportunity for data
compression. Our algorithm for Burrows Wheeler performs two actions, transforming
a given string to be encoded and inverse transforming a given string to turn
it back into its original form.


%%%%%%% Mathematical Formulation %%%%%%%%%%%%%%%%%%%%%%%%%%%%%%%%%%

\subsection{Mathematical Formulation}
We take in an input of size $N$ and use an alphabet of size $R$ to either
transform or reverse transform it.

%%%%%%% Algorithm %%%%%%%%%%%%%%%%%%%%%%%%%%%%%%%%%%%%%%%%%%%%%%%%%

\subsection{Solution}
The transform function of the Burrows Wheeler Transform program first creates a
Circular Suffix Array of the given string. Then we record the index of the first
original suffix (the unshifted original string). We then iterate through the CSA
to record the index of the last character in each sorted suffix. These indices are
then used to print out the character at each index of the original string.

\begin{algorithm} [H]
\caption{Burrows Wheeler Transform}
\begin{algorithmic}
\Procedure {transform}{ }
	\State $originalWord \gets$ new StringBuilder
	\While {BinaryStdIn is not empty}
		\State $originalWord \gets$ BinaryStdIn.\Call {readChar} { }
	\EndWhile
	\State $csa \gets$ new CircularSuffixArray built from originalWord
	\For {element $\in$ csa}
		\If {csa.\Call {indexAt}{index} == 0}
			\State first = index
			\State BinaryStdOut .\Call{write}{first}
			\State break
		\EndIf
	\EndFor

	\For {element $\in$ csa}
		\If {csa.index(i) == 0}
			\State $index \gets$ csa length-1; // index of last character
		\Else
			\State $index \gets$ csa index at i – 1
		\EndIf
		\State Write originalWord character at index
	\EndFor
\EndProcedure
\end{algorithmic}
\end{algorithm}

To perform the inverse transformation, we read from Binary standard input the
encoded string and initialize an array,  $t$, to the characters corresponding
to the last character in each sorted suffix that will be used to reconstruct
the string. The $t$ array is then copied and the copy is sorted using bucket
sort. We now have the first characters of the sorted suffix array along with
the last characters.

\begin{algorithm} [H]
\caption {Burrows Wheeler Inverse Transform}
\begin{algorithmic}
\Procedure {inverse}{ }
	\State $first \gets$ BinaryStdIn.\Call{readInt}{ }
	\State $Char[]t \gets$ coded message from BinaryStdIn
	\State $length \gets$ t.\Call {length}{ }
	\State $Char[] sorted \gets$ bucket sort version of t
	\State $next \gets$ getNext{t, length} //see below
	\For {element $\in$ sorted}
		\State BinaryStdOut.\Call {write}{sorted[index]}
		\State $index \gets$ next[index]
	\EndFor
\EndProcedure
\end{algorithmic}
\end{algorithm}

From here we reconstruct the original string by using a variation of bucket
sorting the characters in $t$. This gives us the $next$ array in which will
we define next[i] to be the row in the sorted order where the (j + 1)st original
suffix appears. By moving through $next$ and taking the next[the previous next]
we are able to reconstruct the original string.

\begin{algorithm} [H]
\caption {getNext}
\begin{algorithmic}
\Procedure {getNext}{t, length}
	\State $next \gets$ int[]
	\State $buckets \gets$ LinkedList \textless Integer \textgreater [number of ACSI chars]
	\State $buckets \gets$ chars in t
	\For {c \textless number of ACSI chars and i \textless length}
		\While {i \textless length \&  buckets is not empty}
			\State next[i] = buckets[c].\Call {remove}{first}
		\EndWhile
	\EndFor
\EndProcedure
\end{algorithmic}
\end{algorithm}


%%%%%%% Correctness %%%%%%%%%%%%%%%%%%%%%%%%%%%%%%%%%%%%%%%%%%%%%%%

\subsection{Correctness}

%%%%%%% PROPOSITION 1 %%%%%%%%%%%%%%%
\begin{proposition}
	The...will...
\end{proposition}

\begin{proof}
	Since...
\end{proof}

%%%%%%% Analysis %%%%%%%%%%%%%%%%%%%%%%%%%%%%%%%%%%%%%%%%%%%%%%%%%%
\subsection{Analysis}

We take in an input of size $N$ and use an alphabet of size $R$ to either
transform or reverse transform it.

%%%%%%% PROPOSITION 1 %%%%%%%%%%%%%%%
\begin{proposition}
	\label{numq}
	The $transform()$ algorithm has Time complexity of O(N$\cdot$log(N)$\cdot$C$^{\ast}$) and Space of O($N$)
	
	
\end{proposition}

\begin{proof}
	This is the case since in $transform()$ we read in the string ($N$), build a CircularSuffixArray (N$\cdot$log(N)$\cdot$C$^{\ast}$, from above) and output the index of every character in the array ($N$).
	The space will be N because all we use is an array to store the string. 
	
	
	\begin{center}
		Giving us an overarching \textbf{O(N$\cdot$log(N)$\cdot$C$^{\ast}$)} complexity and space of \textbf{O(N)}.
	\end{center}
	
\end{proof}

%%%%%%% PROPOSITION 2 %%%%%%%%%%%%%%%
\begin{proposition}
	\label{numq}
	
	The $inverse()$ transform algorithm has a time complexity of \\ $O(5N + 2R)$
	
	
\end{proposition}

\begin{proof}
	This is the case since in $inverse()$ transform, we create an array of size N three times (3N) and make a call to getNext and sortArray. sortArray uses
	bucket sort to sort the chars and thusly has a time complexity of N + R,
	where R is the size of the alphabet. getNext uses a variation on bucket
	sort and has a time complexity of N + R. This gives inverse transform
	a time complexity of (5N + 2R)
	
	
	\begin{center}
		Giving us an overarching \textbf{O(N + R)} complexity.
		
		
		
	\end{center}
	\em{Note: the space complexity will be the same.}	
\end{proof}


%\newpage
%%%%%%% Example %%%%%%%%%%%%%%%%%%%%%%%%%%%%%%%%%%%%%%%%%%%%%%%%%%%

\subsection{An Example}
As our example, we will begin by encoding the string $ABRACADABRA!$. This is done
using the circular suffix array, explained on a previous page, and will produce the
index[ ] which signifies the original index of the now sorted suffix. As you
can see below, the last letter in each sorted suffix has been \textbf{bolded}; these are the
letters that we will be outputting. These will be preceeded with the site of the
original string, i.e. row $i=3$
\begin{table}[H]
	\centering
	\resizebox{\textwidth}{!}{
	\begin{tabular}{c|cc|c}
		\toprule
		i  & Original Suffixes       & Sorted Suffixes         & index[i] \\ \hline
		\midrule
        0  & A B R A C A D A B R A ! & ! A B R A C A D A B R \textbf{A} & 11  \\
        1  & B R A C A D A B R A ! A & A ! A B R A C A D A B \textbf{R} & 10  \\
        2  & R A C A D A B R A ! A B & A B R A ! A B R A C A \textbf{D} & 7   \\ \hline
        3  & A C A D A B R A ! A B R & A B R A C A D A B R A \textbf{!} & \textbf{0}   \\ \hline
        4  & C A D A B R A ! A B R A & A C A D A B R A ! A B \textbf{R} & 3   \\
        5  & A D A B R A ! A B R A C & A D A B R A ! A B R A \textbf{C} & 5   \\
        6  & D A B R A ! A B R A C A & B R A ! A B R A C A D \textbf{A} & 8   \\
        7  & A B R A ! A B R A C A D & B R A C A D A B R A ! \textbf{A} & 1   \\
        8  & B R A ! A B R A C A D A & C A D A B R A ! A B R \textbf{A} & 4   \\
        9  & R A ! A B R A C A D A B & D A B R A ! A B R A C \textbf{A} & 6   \\
        10 & A ! A B R A C A D A B R & R A ! A B R A C A D A \textbf{B} & 9   \\
        11 & ! A B R A C A D A B R A & R A C A D A B R A ! A \textbf{B} & 2   \\
		\bottomrule
	\end{tabular}}
\end{table}

\noindent The output from this then will the sequence: \\ \textbf{00 00 00 03 A R D ! R C A A A A B B} \\

The next stage then is the inverse transform portion where we will be given this same sequence
and parse it so that way we note that first = 3 and say an array, t, is
[41 52 44 21 52 43 41 41 41 41 42 42] (this corresponds to the unicode value of each letter).
We build a next[ ] which will represent which letter comes next by recalling three main things:
\begin{enumerate}
    \item We have all of the letters so can get the corresponding first letters to each last letter (t[i])
    \item These are circular suffixes
    \item If sorted row i and j both start with the same character then \\ i \textless\ j $\implies$ next[i] \textless\ next[j]
\end{enumerate}
By using these three simple rules, we can deduce that the first time that you see the '!' in t, its
correspoding $next$ value must be the suffix for which you find the first '!' in the sorted list. Now using
this simple notion combined with \textbf{3}, we see that first instance of A in t will be the next for the
first occurance of A in the sorted and the second for the second and so on.  \\
Our next[ ] will be: \textbf{[3 0 6 7 8 9 10 11 5 2 1 4]} \\
\indent Since we now have a fully filled out next array, all we must do is call next[first] to get the
second letter and next[next[first]] for the third and so on. Calling write() on each step will then give us the
output: \textbf{ABRACADABRA!}

\newpage
%%%%%%% begin Princeton Readme %%%%%%%%%%%%%%%%%%%%%%%%%%%%%%%%%%%%
\section{Princeton Readme}
%%%%%%% Question 1 %%%%%%%%%%%%%%%%%%%%%%%%%%%%%%%%%%%%
\subsection{Question 1:}
\underline{\textbf{Question:}} List in table format which input files you used to test your program.
Fill in columns for how long your program takes to compress and
decompress these instances (by applying BurrowsWheeler, MoveToFront,
and Huffman in succession). Also, fill in the third column for
the compression ratio (number of bytes in compressed message
divided by the number of bytes in the message).  \\

\noindent \underline{\textbf{Answer:}} Timing is done in seconds.
\begin{table}[H]
	\centering
	\resizebox{\textwidth}{!}{
	\begin{tabular}{|c|c|c|c|}
		\toprule
		File                    & Encoding Time & Decoding Time  &  Compression ratio \\ \hline
		\midrule
        bible.txt               & 15.570        & 5.218          &  26.05\% \\
        muchado.txt             & 1.218         & 0.519          &  35.37\% \\
        chromosome11-human.txt  & 33.146        & 8.979          &  27.88\% \\
        world192.txt            & 9.512         & 5.024          &  24.44\% \\
        mobydick.txt            & 3.928         & 2.799          &  34.74\% \\
        moby1.txt               & 0.359         & 0.317          &  45.81\% \\
        sedgewick-algc.txt      & 0.658         & 0.427          &  25.15\% \\
        pi-1million.txt         & 2.917         & 1.049          &  43.73\% \\
        pi-10million.txt        & 42.474        & 16.506         &  43.74\% \\
		\bottomrule
	\end{tabular}}
\end{table}

%%%%%%% Question 2 %%%%%%%%%%%%%%%%%%%%%%%%%%%%%%%%%%%%
\subsection{Question 2:}
\underline{\textbf{Question:}} Compare the results of your program (compression ratio and running
time) on mobydick.txt to that of the most popular Windows compression program (pkzip)
or the most popular Unix/Mac one (gzip). If you don't have pkzip, use 7zip and
compress using zip format.  \\

\noindent \underline{\textbf{Answer:}}
\begin{itemize}
    \item \underline{original:} 9,531,704 bits
    \item \underline{gzip:} 3,884,488 bits $\implies$ 40.75\% compression rate
    \item \underline{ours:} 3,311,696 bits $\implies$ 34.74\% compression rate
\end{itemize}
\begin{center}
    $\therefore$ Ours compresses 6.01\% more than theirs.
\end{center}

\newpage
%%%%%%% Question 3 %%%%%%%%%%%%%%%%%%%%%%%%%%%%%%%%%%%%
\subsection{Question 3:}
\underline{\textbf{Question:}}  Give the order of growth of the running time of each of the 6
methods as a function of the input size N and the alphabet size R both in practice
(on typical English text inputs) and in theory (in the worst case), e.g., N, N + R,
N log(N), N$^2$, or R N. Include the time for sorting circular suffixes in the
Burrows-Wheeler encoder.  \\

\noindent \underline{\textbf{Answer:}}
\begin{table}[H]
	\centering
	\resizebox{\textwidth}{!}{
	\begin{tabular}{|c|c|c|c|}
		\toprule
		 Class & Method                                & Typical          & Worst \\ \hline
		\midrule
        \textbf{BurrowsWheeler} & $transform()$        &    N$\cdot$log(N) &N$^2\cdot$log(N)           \\
        \textbf{BurrowsWheeler} & $inverseTransform()$ &    N + R         & N$\cdot$R       \\ \hline
        \textbf{MoveToFront} & $encode()$              &    N + R         & N$\cdot$R      \\
        \textbf{MoveToFront} & $decode()$              &    N + R         & N$\cdot$R      \\ \hline
        \textbf{Huffman} & $compress()$                &    N + R$\cdot$log(R)   & N + R$\cdot$log(R) \\
        \textbf{Huffman} & $expand()$                  &    N             & N                  \\
		\bottomrule
	\end{tabular}}
\end{table}

%%%%%%% Question 4 %%%%%%%%%%%%%%%%%%%%%%%%%%%%%%%%%%%%
\subsection{Question 4:}
\underline{\textbf{Known bugs / limitations}}
\begin{itemize}
    \item We are too awesome, truely limits us.
    \item Our Circular Suffix Array is not very efficient with corner cases where the
    input has long sections of repetition i.e. ''aaa...a''
    \item Cannot compress dickens.txt (will run out of memory) \\ \em{Note: this may be due to cpu, not algorithm.}
\end{itemize}

%%%%%%% Question 5 %%%%%%%%%%%%%%%%%%%%%%%%%%%%%%%%%%%%
\subsection{Question 5:}
\underline{\textbf{Help Received}}  \\
\textbf{Professor Han} helped with asking questions on how to design the BurrowsWheeler
$inverseTransform()$ method as well as answering clarifying questions on how to run some
testing scripts and formulate proofs for correctness. TA \textbf{Sam Kolvaka} gave us
help for using System.err.println(); for testing purposes so as not to mess up the
BinaryStdIn.getchar() function.

%%%%%%% Question 6 %%%%%%%%%%%%%%%%%%%%%%%%%%%%%%%%%%%%
\subsection{Question 6:}
\underline{\textbf{Serious Problems Encountered}}  \\
Our main issues came with learning how to build the coding environment and learning how
to use the packages. The next hurdle was how to comprehend all of the abstract concepts
that are associated with this project, namely the $inverseTransform()$ portion of the
BurrowsWheeler class.

%%%%%%% Question 7 %%%%%%%%%%%%%%%%%%%%%%%%%%%%%%%%%%%%
\subsection{Question 7:}
\underline{\textbf{Question:}}
If you worked with a partner, assert below that you followed the protocol as described on
the assignment page. Give one sentence explaining what each of you contributed. \\
\noindent \underline{\textbf{Answer:}}  \\
We worked on this project together and split up the work when it came time to writing up
this readme.



%%%%%%% end Princeton Readme %%%%%%%%%%%%%%%%%%%%%%%%%%%%%%%%%%%%%%

%%%%%%% end Solution %%%%%%%%%%%%%%%%%%%%%%%%%%%%%%%%%%%%%%%%%%%%%%

\end{document}