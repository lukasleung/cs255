% java BurrowsWheeler - < inputs\dickens.txt | java MoveToFront - | java Huffman - > inputs\dickens_compressed.txt
% du -b dickens.txt | awk '{ print $1, "* 8" }' | bc
% Use this template to write your solutions to COS 423 problem sets

\documentclass[12pt]{article}
\usepackage[utf8]{inputenc}
\usepackage{amsmath, amsfonts, amsthm, amssymb, algorithm, graphicx, mathtools, xfrac}
\usepackage[noend]{algpseudocode}
\usepackage{fancyhdr, lastpage}
\usepackage{booktabs}
\usepackage{multirow}
\usepackage{graphicx}
\usepackage{pgfplots}
\usepackage[vmargin=1.20in,hmargin=1.25in,centering,letterpaper]{geometry}
\setlength{\headsep}{.50in}
\setlength{\headheight}{15pt}

% Landau notation
\DeclareMathOperator{\BigOm}{\mathcal{O}}
\newcommand{\BigOh}[1]{\BigOm\left({#1}\right)}
\DeclareMathOperator{\BigTm}{\Theta}
\newcommand{\BigTheta}[1]{\BigTm\left({#1}\right)}
\DeclareMathOperator{\BigWm}{\Omega}
\newcommand{\BigOmega}[1]{\BigWm\left({#1}\right)}
\DeclareMathOperator{\LittleOm}{\mathrm{o}}
\newcommand{\LittleOh}[1]{\LittleOm\left({#1}\right)}
\DeclareMathOperator{\LittleWm}{\omega}
\newcommand{\LittleOmega}[1]{\LittleWm\left({#1}\right)}

% argmin and argmax
\newcommand{\argmin}{\operatornamewithlimits{argmin}}
\newcommand{\argmax}{\operatornamewithlimits{argmax}}

\newcommand{\calP}{\mathcal{P}}
\newcommand{\Z}{\mathbb{Z}}
\newcommand{\R}{\mathbb{R}}
\newcommand{\Exp}{\mathbb{E}}
\newcommand{\Q}{\mathbb{Q}}
\newcommand{\sign}{\mathrm{sign\ }}
\newcommand{\abs}{\mathrm{abs\ }}
\newcommand{\eps}{\varepsilon}
\newcommand{\zo}{\{0, 1\}}
\newcommand{\SAT}{\mathit{SAT}}
\renewcommand{\P}{\mathbf{P}}
\newcommand{\NP}{\mathbf{NP}}
\newcommand{\coNP}{\co{NP}}
\newcommand{\co}[1]{\mathbf{co#1}}
\renewcommand{\Pr}{\mathop{\mathrm{Pr}}}

% theorems, lemmas, invariants, etc.
\newtheorem{theorem}{Theorem}
\newtheorem{lemma}[theorem]{Lemma}
\newtheorem{invariant}[theorem]{Invariant}
\newtheorem{corollary}[theorem]{Corollary}
\newtheorem{definition}[theorem]{Definition}
\newtheorem{property}[theorem]{Property}
\newtheorem{proposition}[theorem]{Proposition}

% piecewise functions
\newenvironment{piecewise}{\left \{\begin{array}{l@{,\ }l}}
{\end{array}\right.}

% paired delimiters
\DeclarePairedDelimiter{\ceil}{\lceil}{\rceil}
\DeclarePairedDelimiter{\floor}{\lfloor}{\rfloor}
\DeclarePairedDelimiter{\len}{|}{|}
\DeclarePairedDelimiter{\set}{\{}{\}}

\makeatletter
\@addtoreset{equation}{section}
\makeatother
\renewcommand{\theequation}{\arabic{section}.\arabic{equation}}

% algorithms
\algnewcommand\algorithmicinput{\textbf{INPUT:}}
\algnewcommand\INPUT{\item[\algorithmicinput]}
\algnewcommand\algorithmicoutput{\textbf{OUTPUT:}}
\algnewcommand\OUTPUT{\item[\algorithmicoutput]}


% Formating Macros

\pagestyle{fancy}
\lhead{\sc \hmwkClass\ $\; \;\cdot \; \;$ \hmwkSemester\ $\; \;\cdot \; \;$
Problem \hmwkAssignmentNum.\hmwkProblemNum}
%\chead{\sc Problem \hmwkAssignmentNum.\hmwkProblemNum}
%\chead{}
\rhead{\em \hmwkAuthorName\ $($\hmwkAuthorID$)$\/}
\cfoot{}
\lfoot{}
\rfoot{\sc Page\ \thepage\ of\ \protect\pageref{LastPage}}
\renewcommand\headrulewidth{0.4pt}
\renewcommand\footrulewidth{0.4pt}

\fancypagestyle{fancycollab}
{
    \lfoot{\textit{Collaborators: \hmwkCollaborators}}
}

\fancypagestyle{problemstatement}
{
    \rhead{}
    \lfoot{}
}

%%%%%% Begin document with header and title %%%%%%%%%%%%%%%%%%%%%%%%%

\begin{document}

%%%%%% Header Information %%%%%%%%%%%%%%%%%%%%%%%%%%%%%%%%%%%%%%%%%%%

%%% Shouldn't need to change these
\newcommand{\hmwkClass}{COS 255}
\newcommand{\hmwkSemester}{Spring 2016}

%%% Your name, in standard First Last format
\newcommand{\hmwkAuthorName}{Lukas Leung}
%%% Your NetID
\newcommand{\hmwkAuthorID}{lleung}

%%% The problem set number (just the number)
\newcommand{\hmwkAssignmentNum}{3}

%%% The problem number (just the number)
\newcommand{\hmwkProblemNum}{1}

%%% A list of your collaborators' NetIDs, separated by ", ".
%%% You can use a new line ("\\") in the middle to prevent a long
%%% list from overflowing.
\newcommand{\hmwkCollaborators}{}
%%% Sets the collaborator list to appear on the first page
\thispagestyle{fancycollab}

%%%%%%% begin Solution %%%%%%%%%%%%%%%%%%%%%%%%%%%%%%%%%%%%%%%%%%%%


%%%%%%% begin GrapeVine Problem %%%%%%%%%%%%%%%%%%%%%%%%%%%%%%%%%%%

\section{UVA Problem 124: Following Orders}
\textbf{Background} \\
~\indent Order is an important concept in mathematics and in computer science. For example, Zorn's Lemma
states: "a partially ordered set in which every chain has an upper bound contains a maximal element."
Order is also important in reasoning about the x-point semantics of programs.
This problem involves neither Zorn's Lemma nor x-point semantics, but does involve order. \\
\textbf{The Problem} \\
~\indent Given a list of variable constraints of the form x \textless y , you are to write a program that prints all orderings
of the variables that are consistent with the constraints.
For example, given the constraints x \textless y and x \textless z there are two orderings of the variables x, y,
and z that are consistent with these constraints: x y z and x z y . \\
\textbf{The Input} \\
~\indent The input consists of a sequence of constraint specifications. A specification consists of two lines: a list
of variables on one line followed by a list of contraints on the next line. A constraint is given by a pair
of variables, where x y indicates that x \textless\ y.
All variables are single character, lower-case letters. There will be at least two variables, and no
more than 20 variables in a speci cation. There will be at least one constraint, and no more than 50
constraints in a speci cation. There will be at least one, and no more than 300 orderings consistent with
the contraints in a speci cation.
Input is terminated by end-of- le. \\
\textbf{The Output} \\
~\indent  For each constraint speci cation, all orderings consistent with the constraints should be printed. Order-
ings are printed in lexicographical (alphabetical) order, one per line. Characters on a line are separated
by whitespace.  \\
Output for different constraint specifications is separated by a blank line
%%%%%%% end Problem %%%%%%%%%%%%%%%%%%%%%%%%%%%%%%%%%%%%%%%%%%%%%%%

\newpage

%%%%%%% Mathematical Formulation %%%%%%%%%%%%%%%%%%%%%%%%%%%%%%%%%%
\subsection{Mathematical Formulation}
Given an input of $N$ distinct letters (2 $\leq$ N $\leq$ 26) and a set of $R$ rules (1 $\leq$ R $\leq$ 50). The algorithm will
enumerate all $M$ anagrams of the letters adhearing to the rules. $Note:\ worst\ case\ M \simeq N!$

%%%%%%% Algorithm %%%%%%%%%%%%%%%%%%%%%%%%%%%%%%%%%%%%%%%%%%%%%%%%%

\subsection{Solution}
Important Confusing Data Structures:
\begin{itemize}
    \item LinkedList\textless Integer\textgreater [n] \textbf{G} : ~ Each array slot corresponds to the symbol that appears in
        that index in the sorted version of the input. Each slot holds a linked list of the symbols that must appear
        after the given symbol according to the user inputted rules.
    \item int[n] \textbf{inDegree} : ~ This is an array that stores (at each recursive step) the number of symbols that must come
        before it. If the index is -1 $\implies$ that this was seen already in the recursion and has already been recorded
        within this branch of recursion.
    \item char[n] \textbf{result} : ~ This is an array storing the symbols being used so far in the recursive step. It is being
        used throughout all of the different branches and never copied.
\end{itemize}
In the first algorithm, the program initializes all of the data structures that will be used, filling in the pertainant
information. Once the structures have been properly formed, the program will then call it's recursive method which will
enumerate the valid strings in alphabetic order. If there is another case, it will reinitialize everything appropriately
and repeat, otherwise it will terminate.

\begin{algorithm}[H]
\caption{FollowingOrders main}
\begin{algorithmic}
    \Procedure{main}{ }
        \State $map \gets$ new int[26] // to store the index of each letter
        \While{inFile.\Call{hasNext}{ }}    // each case
            \State $symbols \gets$ sorted char[n] of the symbols
            \State $comparisons \gets$ char[2*r] of all the rules, each pair is a rule
            \State $inverseMap \gets$ char[n]
            \For{($i \in symbols.length$}
                \State Fill in map and inverseMap using symbols
            \EndFor
            \State $G \gets$ LinkedList\textless Integer\textgreater [n] all of the rules for each letter
            \State $inDegree \gets$ int[n] // stores indegree for each symbol
            \State // Fill in G and inDegree
            \For{$i \in comparisons.length$}
                \State $less \gets$ map[comparisons[i]]
                \State $greater \gets$ map[comparisons[i+1]]
                \State G[less].\Call{add}{greater}
                \State inDegree[greater]++
            \EndFor
            \State $results \gets$ char[n] // to store the results thusfar in recursion
            \State \Call{dfsHash}{n, inverseMap, inDegree, G, result)} // See Below
        \EndWhile
    \EndProcedure
\end{algorithmic}
\end{algorithm}

The main functionality of dfsHash is to do a depth first search recursively on the valid strings in alphabetic order.
It accomplishes this through using the combination of the data structures explained above, inDegree and G. The way
that they work is that, the algorithm will look through inDegree for the first occurance of a 0 $\implies$ the first
letter that can go first (unimpeeded by rules) and record it in the first index of result. Then the algorithm will
look at all of the letters that must come after it (found in G). For each of these letters, their inDegree will be
decrimented by 1 soas to suggest that one of the letters coming before it has been written down. Then a recursive
call is made, attempting to find the second letter in the same manner...etc. Once the first word has been found (all
given symbols have been used up) the algorithm will back track and (from the second to last letter) look for the next
occuring 0 in the array. This will be the second word. If there is none, it will kill this branch and return back to the
previous step. $Note:\ The\ recursive\ call\ will\ re-increment\ all\ decrimented\ indicies\ within\ inDegree$

\begin{algorithm}[H]
\caption{FollowingOrders Recursive Method}
\begin{algorithmic}
    \Procedure{dfsHash}{left,inverseMap[],inDegree[],G,result[]}
        \State // base case, this means we have all n letters in result
        \If{left == 0}
            \State print(result)
        \EndIf
        \For{$i \in results.length$}
            \If{inDegree[i] == 0}
                \State $inDegree[i] \gets$ -1
                \State $results[results.length - left] \gets$ inverseMap[i]
                \For{letter $l \in G[i]$}
                    \State inDegree[letter]-
                \EndFor
                dfsHash(left-1, im, inDegree, G, result)
                \For{letter $l \in G[i]$}
                    \State inDegree[letter]++
                \EndFor
                \State $inDegree[i] \gets$ -1
            \EndIf
        \EndFor
    \EndProcedure
\end{algorithmic}
\end{algorithm}


%%%%%%% Correctness %%%%%%%%%%%%%%%%%%%%%%%%%%%%%%%%%%%%%%%%%%%%%%%

\subsection{Correctness}
%%%%%%% PROPOSITION 1 %%%%%%%%%%%%%%%
\begin{proposition}
~ \\ \indent One call of the $dfsHash$ method will be able to explore all possible anagrams from this position in
alphabetical order while adhearing to the rules.
\end{proposition}

\begin{proof}
~ \\ \indent Using the fact that the inDegree array corresponds to how many non-used
letters are valid at this stage; we will exhibit this by saying that an entry can be denoted as $d_0,d_1,...d_k$ for $(k+1)$
distinct letters and $d_i \textless 0 \implies already\ used,\ d_i = 0 \implies valid\ letter\ and\ d_i = j \textgreater
0 \implies j\ non-used\ letters\ must\ come\ before\ it$. Since inDegree is the representation of all of the letters
in alphabetic order $\implies$ the first 0, is the first valid letter, the next 0 is the next valid letter and so on and
so forth. $\therefore$ we see that we will do all of the first possible ones first and then (starting from the last letter)
we will work our way back up continuing our depth first search.  This will do an in-order traversal of all the valid anagrams
that can be produced.
\end{proof}


%%%%%%% Analysis %%%%%%%%%%%%%%%%%%%%%%%%%%%%%%%%%%%%%%%%%%%%%%%%%%
\subsection{Analysis}
For the following analysis, we will say that Given an input of $N$ distinct letters (2 $\leq$ N $\leq$ 26) and a set of
$R$ rules (1 $\leq$ R $\leq$ 50). The algorithm will enumerate all $M$ anagrams of the letters adhearing to the rules.
$Note:\ worst\ case\ M \simeq N!$

%%%%%%% PROPOSITION 1 %%%%%%%%%%%%%%%
\begin{proposition}
\label{numq}
The \underline{space complexity} of this algorithm is \textbf{O(N + R)}
\end{proposition}

\begin{proof}
~ \\ \indent This is due to the fact that all of our data is stored in 7 data structures:
\begin{itemize}
    \item \underline{map}: int array of size 26 (worst case for $N$)
    \item \underline{inverseMap}: char array of size $N$
    \item \underline{symbols}: char array of size $N$
    \item \underline{comparisons}: char array of size $2\cdot R$
    \item \underline{G}: Integer Linked List Array which has size $N$ and $R$ Nodes
    \item \underline{inDegree}: int array of size $N$
    \item \underline{results}: char array of size $N$
\end{itemize}
The only points of contension here have to do with our recursive calls on inverseMap, G, inDegree, and results. If a
new copy were made each call, we would have at most $N$ of each of these, however since we are passing the object
as a parameter, we are only using the same pointer, pointing to a single object for each so the same one is being
accessed and modified each time. This allows for the algorithm to use only one object for each. Therefore, summing
all of these up, we have a space complexity of S($N + N + N + 2\cdot R + N + R + N + N$) = S($6\cdot N + 3\cdot R$)
\begin{center}
    Giving us a space complexity of \textbf{O(N + R)}
\end{center}
\end{proof}

%%%%%%% PROPOSITION 2 %%%%%%%%%%%%%%%
\begin{proposition}
\label{numq}
The \underline{time complexity} of this algorithm is \textbf{O(M)}
\end{proposition}

\begin{proof}
This is the case because our algorithm reads in all of the input in linear time O(N + R) which are both trivially small.
Then, for every recursive call, we search through the list of symbols (N) once. So we look at the case now that we are
enumerating a word which is $c_0 c_1 c_2 ... c_n$ (W.L.O.G.)lets assume the only rule is that $c_0 \textless c_k$ which
would be the largest amount of anagrams possible. We can say that anagrams starting with $c_0$ number roughly $(k-1)!$
and for each letter, we will only do $k^2$ k's. $\implies$ going through all of the letters we would only go through $k$
$k^3$ times $\implies$ $k^4$, replacing $k$ with $N$ we have $N^4$ yeilding a complexity of $M + N^4 + N + R$
\begin{center}
    Giving us a time complexity of \textbf{O(M)}
\end{center}
\end{proof}

%%%%%%% Example %%%%%%%%%%%%%%%%%%%%%%%%%%%%%%%%%%%%%%%%%%%%%%%%%%%

\subsection{An Example}
I did not have enough time to finish this section. However I have the correct output for the given example by the UVA site.


%%%%%%% end Grape Vine %%%%%%%%%%%%%%%%%%%%%%%%%%%%%%%%%%%%%%%%%%%%

%%%%%%% end Solution %%%%%%%%%%%%%%%%%%%%%%%%%%%%%%%%%%%%%%%%%%%%%%

\end{document}