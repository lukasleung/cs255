% Use this template to write your solutions to COS 423 problem sets

\documentclass[12pt]{article}
\usepackage[utf8]{inputenc}
\usepackage{amsmath, amsfonts, amsthm, amssymb, algorithm, graphicx, mathtools, xfrac}
\usepackage[noend]{algpseudocode}
\usepackage{fancyhdr, lastpage}
\usepackage{booktabs}
\usepackage{multirow}
\usepackage{graphicx}
\usepackage{pgfplots}
\usepackage[shortlabels]{enumitem}
\usepackage[vmargin=1.20in,hmargin=1.25in,centering,letterpaper]{geometry}
\setlength{\headsep}{.50in}
\setlength{\headheight}{15pt}

% Landau notation
\DeclareMathOperator{\BigOm}{\mathcal{O}}
\newcommand{\BigOh}[1]{\BigOm\left({#1}\right)}
\DeclareMathOperator{\BigTm}{\Theta}
\newcommand{\BigTheta}[1]{\BigTm\left({#1}\right)}
\DeclareMathOperator{\BigWm}{\Omega}
\newcommand{\BigOmega}[1]{\BigWm\left({#1}\right)}
\DeclareMathOperator{\LittleOm}{\mathrm{o}}
\newcommand{\LittleOh}[1]{\LittleOm\left({#1}\right)}
\DeclareMathOperator{\LittleWm}{\omega}
\newcommand{\LittleOmega}[1]{\LittleWm\left({#1}\right)}

% argmin and argmax
\newcommand{\argmin}{\operatornamewithlimits{argmin}}
\newcommand{\argmax}{\operatornamewithlimits{argmax}}

\newcommand{\calP}{\mathcal{P}}
\newcommand{\Z}{\mathbb{Z}}
\newcommand{\R}{\mathbb{R}}
\newcommand{\Exp}{\mathbb{E}}
\newcommand{\Q}{\mathbb{Q}}
\newcommand{\sign}{\mathrm{sign\ }}
\newcommand{\abs}{\mathrm{abs\ }}
\newcommand{\eps}{\varepsilon}
\newcommand{\zo}{\{0, 1\}}
\newcommand{\SAT}{\mathit{SAT}}
\renewcommand{\P}{\mathbf{P}}
\newcommand{\NP}{\mathbf{NP}}
\newcommand{\coNP}{\co{NP}}
\newcommand{\co}[1]{\mathbf{co#1}}
\renewcommand{\Pr}{\mathop{\mathrm{Pr}}}

% theorems, lemmas, invariants, etc.
\newtheorem{theorem}{Theorem}
\newtheorem{lemma}[theorem]{Lemma}
\newtheorem{invariant}[theorem]{Invariant}
\newtheorem{corollary}[theorem]{Corollary}
\newtheorem{definition}[theorem]{Definition}
\newtheorem{property}[theorem]{Property}
\newtheorem{proposition}[theorem]{Proposition}

% piecewise functions
\newenvironment{piecewise}{\left \{\begin{array}{l@{,\ }l}}
{\end{array}\right.}

% paired delimiters
\DeclarePairedDelimiter{\ceil}{\lceil}{\rceil}
\DeclarePairedDelimiter{\floor}{\lfloor}{\rfloor}
\DeclarePairedDelimiter{\len}{|}{|}
\DeclarePairedDelimiter{\set}{\{}{\}}

\makeatletter
\@addtoreset{equation}{section}
\makeatother
\renewcommand{\theequation}{\arabic{section}.\arabic{equation}}

% algorithms
\algnewcommand\algorithmicinput{\textbf{INPUT:}}
\algnewcommand\INPUT{\item[\algorithmicinput]}
\algnewcommand\algorithmicoutput{\textbf{OUTPUT:}}
\algnewcommand\OUTPUT{\item[\algorithmicoutput]}


% Formating Macros

\pagestyle{fancy}
\lhead{\sc \hmwkClass\ $\; \;\cdot \; \;$ \hmwkSemester\ $\; \;\cdot \; \;$
Problem \hmwkAssignmentNum.\hmwkProblemNum}
%\chead{\sc Problem \hmwkAssignmentNum.\hmwkProblemNum}
%\chead{}
\rhead{\em \hmwkAuthorName\ $($\hmwkAuthorID$)$\/}
\cfoot{}
\lfoot{}
\rfoot{\sc Page\ \thepage\ of\ \protect\pageref{LastPage}}
\renewcommand\headrulewidth{0.4pt}
\renewcommand\footrulewidth{0.4pt}

\fancypagestyle{fancycollab}
{
    \lfoot{\textit{Collaborators: \hmwkCollaborators}}
}

\fancypagestyle{problemstatement}
{
    \rhead{}
    \lfoot{}
}

%%%%%% Begin document with header and title %%%%%%%%%%%%%%%%%%%%%%%%%

\begin{document}

%%%%%% Header Information %%%%%%%%%%%%%%%%%%%%%%%%%%%%%%%%%%%%%%%%%%%

%%% Shouldn't need to change these
\newcommand{\hmwkClass}{COS 255}
\newcommand{\hmwkSemester}{Spring 2016}

%%% Your name, in standard First Last format
\newcommand{\hmwkAuthorName}{Lukas Leung}
%%% Your NetID
\newcommand{\hmwkAuthorID}{lleung}

%%% The problem set number (just the number)
\newcommand{\hmwkAssignmentNum}{4}

%%% The problem number (just the number)
\newcommand{\hmwkProblemNum}{0}

%%% A list of your collaborators' NetIDs, separated by ", ".
%%% You can use a new line ("\\") in the middle to prevent a long
%%% list from overflowing.
\newcommand{\hmwkCollaborators}{}
%%% Sets the collaborator list to appear on the first page
\thispagestyle{fancycollab}

%%%%%%% begin Solution %%%%%%%%%%%%%%%%%%%%%%%%%%%%%%%%%%%%%%%%%%%%
\section*{Results from UVA}
\includegraphics[width=\textwidth]{results}
\newpage

%%%%%%% start BlackBeard %%%%%%%%%%%%%%%%%%%%%%%%%%%%%%%%%%%%%%%%%%

\section{UVA Problem 10937: Blackbeard}
\textbf{Background} \\
~\indent Blackbeard the Pirate has stashed up to 10 treasures on a tropical
island, and now he wishes to retrieve them.  He is being chased
by several authorities however, and so would like to retrieve his
treasure as quickly as possible. Blackbeard is no fool; when he hid
the treasures, he carefully drew a map of the island which contains
the position of each treasure and positions of all obstacles and hostile
natives that are present on the island. \\
\indent Given a map of an island and the point where he comes ashore,
help Blackbeard determine the least amount of time necessary for
him to collect his treasure. \\ \\
\textbf{Input} \\
~ \indent Input consists of a number of test cases. The  rst line of each test
case contains two integers $h$ and $w$ giving the height and width of
the map, respectively, in miles. For simplicity, each map is divided
into grid points that are a mile square. The next $h$ lines contain $w$
characters, each describing one square on the map. Each point on
the map is one of the following:
\begin{itemize}
    \item @ The landing point where Blackbeard comes ashore.
    \item $\sim$ Water. Blackbeard cannot travel over water while on the island.
    \item \# A large group of palm trees; these are too dense for Blackbeard to travel through
    \item $\cdot$ Sand, which he can easily travel over.
    \item $\ast$ A camp of angry natives.  Blackbeard must stay at least one square away or risk
    being captured by them which will terminate his quest. Note, this is one square in any of
    eight directions, including diagonals.
    \item ! A treasure. Blackbeard is a stubborn pirate and will not leave unless he
    collects all of them.
\end{itemize}
\indent Blackbeard can only travel in the four cardinal directions; that is, he cannot travel
diagonally. Blackbeard travels at a nice slow pace of one mile (or square) per hour, but he sure
can dig fast, because digging up a treasure incurs no time penalty whatsoever. \\
\indent The maximum dimension of the map is 50 by 50. The input ends with a case where both
$h$ and $w$ are 0. This case should not be processed. \\ \\
\textbf{Output} \\
~ \indent For each test case, simply print the least number of hours Blackbeard needs to
collect all his treasure and return to the landing point. If it is impossible to reach all
the treasures, print out `'-1'

%%%%%%% end Problem %%%%%%%%%%%%%%%%%%%%%%%%%%%%%%%%%%%%%%%%%%%%%%%

\newpage

%%%%%%% Mathematical Formulation %%%%%%%%%%%%%%%%%%%%%%%%%%%%%%%%%%
\subsection{Mathematical Formulation}
Given an input of the above described map with $n$ treasures and a starting position we will
determine whether or not all treasures are reachable, if so the distances between all of them
and the length of the shortest path to start at the starting point, visit every treasure, and
return to the start point.

%%%%%%% Algorithm %%%%%%%%%%%%%%%%%%%%%%%%%%%%%%%%%%%%%%%%%%%%%%%%%

\subsection{Solution}
Important Confusing Data Structures:
\begin{itemize}
    \item int \textbf{R, C, numTreasures} : ~ Stores (respectively), the max number rows and columns
     in the given map and the total number of treasures.
    \item char[R][C] \textbf{map} : ~ Stores the map provided from the input, then altered so that any
        un-crossable peice of land is represented as the symbol '\#' in $replaceCamps()$
    \item int[numTreasures+1][numTreasures+1] \textbf{dist} : ~ Stores the distnaces between two treasures
        (or the source). Built through calls to $bfs()$.
    \item int[numTreasures+1][2] \textbf{tresLoc} : ~ Stores the row and column coordinates of each
        treasure, $t_1\rightarrow(x_1,y_1), t_2\rightarrow(x_2,y_2), ..., t_n\rightarrow(x_n,y_n)$ at
        indexes $1..n$. Index 0 corresponds with the coordinates of the source.
    \item HashMap$\textless$Integer$\textgreater\textless$Integer$\textgreater$ \textbf{index} : ~
        Keeps track of the index of each treasure based off their [row][col] coordinates. The keys
        were (row$\cdot$\textbf{R}+col) and the value as the treasures index in the \textbf{dist} array.
        This is initialized and built in $locateTreasure()$ and used in $bfs()$.
    \item int[] \textbf{dr, dc} : ~ These arrays are \textbf{dr} $\gets$ \{-1,  0, 0, 1, 1, -1, -1,  1\}
        and \textbf{dr} $\gets$ \{ 0, -1, 1, 0, 1, -1,  1, -1\} which when put together will give you
        left, down, up, right, up/right, down/left, up/left, down/right. The whole arrays are used in the
        $replaceCamps()$ whereas, only the first 4 entries are used for the $bfs()$ since Blackbeard can
        only move in the 4 cardinal directions (not diagonally).
    \item HashMap$\textless$String$\textgreater\textless$Integer$\textgreater$ \textbf{mtsp} : ~ This is
        the hashmap which stores the intermediate values for the tsp $\therefore$ only seen in $tsp()$.
        Given $n$ total treasures each key will be some string i.e. n = 6, string = 10001003. We can break
        this up such that the first n+1 characters are either 0 or 1, (1 is the presence of that indexed
        treasure and 0 is the absence) and the first character (string.charAt(0)) is always 1 signifying
        we always have the source. Then the final character is a number $s,\ 1\leq s\leq n$, symbolizing
        that this number is the index of the treasure which we want to be last. Therefore putting all of
        them together gives us a subset where the source is always present followed by either 1's or 0's
        (if any intermeditate treasures) and finally the index of the last treasure in the subset.
\end{itemize}

The main functionality of this algorithm is to process the given map and put more constraints on it, then
construct a distance adjacency list using a breadth first search, essentially building a fully connected
graph, then finally running the Traveling Salesman Problem solution on this.

\begin{algorithm}[H]
\caption{Main}
\begin{algorithmic}
    \Procedure{main}{ }
        \While{true}
            \State $R, C \gets$ number of Rows and Columns respectively
            \If{R == C == 0}
                break;
            \EndIf
            \State Fill $map$ from input and count number of Treasures
            \State $numTreasures \gets$ number of Treasures on map
            \If{numTreasures == 0}
                \State \Call{Print}{-1}; continue;
            \EndIf
            \State \Call{replaceCamps}{ }  // see Algorithm 2
            \If{!locateTreasures()} // see Algorithm 3
                \State \Call{Print}{-1}; continue;
            \Else
                \State $boolean reachable \gets$ true
                \State $dist[][] \gets$ init
                \For{each treasure, t}
                    \If{!bfs($t_x$,$t_y$)}  // see Algorithm 4
                        \State $reachable \gets$ false; break;
                    \EndIf
                \EndFor
                \If{!reachable}
                    \State \Call{Print}{-1}; continue;
                \EndIf
            \EndIf
        \EndWhile
    \EndProcedure
\end{algorithmic}
\end{algorithm}

When we are altering the map[ ][ ], what we want to do is to replace all instances of water and
angry native camps with the same symbol for trees. In addition, we are also ensuring that we
change the surrounding area of the camps as described in the problem description.

\begin{algorithm}[H]
\caption{Alter the Map}
\begin{algorithmic}
    \Procedure{replaceCamps}{ }
        \For{row $\in$ 0..R, col $\in$ 0..C}
            \State $char c \gets$ map[row][col]
            \If{c == '$\sim$'}
                $map[row][col] \gets$ '\#';
            \ElsIf{c == '$\ast$'}
                \State $map[row][col] \gets$ '\#';
                \For{i $\in$ 0..dc.length}
                    \State $int\ rPrime, cPrime \gets$ row + dr[i], col + dc[i]
                    \If{(rPrime and cPrime in bounds) and (not a camp)}
                        \State $map[rPrime][cPrime] \gets$ '\#'
                    \EndIf
                \EndFor
            \EndIf
        \EndFor
    \EndProcedure
\end{algorithmic}
\end{algorithm}

Once we have Altered this we need to initialize index and tresLoc and ensure that we still
have all of our treasures and start point (these could have been erased in the $replaceCamps()$
method). We will return false if either of these have been altered.

\begin{algorithm}[H]
\caption{Build Support for dist array and ensure treasures preserved}
\begin{algorithmic}
    \Procedure{locateTreasures}{ }
        \State $index, tresLoc \gets$ init, boolean start = false, count = 0
        \For{row $\in$ 0..R, col $\in$ 0..C}
            \If{map[row][col] == '@'}
                \State $start \gets$ true
                \State $tresLoc[0][0,1] \gets$ row, col
            \ElsIf{map[row][col] == '!'}
                \State index.\Call{put}{row*R+col, ++count}
                \State $tresLoc[count][0,1] \gets$ row, col
            \EndIf
        \EndFor
        \State return start \&\& (count==numTreasures)
    \EndProcedure
\end{algorithmic}
\end{algorithm}

Once we have built our adjacency distance array, we can perform the Travelling Salesman Problem
solution. We do this using our HashMap mtsp  (see explanation of data structure above for details).

\begin{algorithm}[H]
\caption{Traveling Salesman Problem, determine cost of minimum path}
\begin{algorithmic}
    \Procedure{tsp}{ }
        \State $mtsp \gets$ init; int length = (2 \textless\textless (dist.length-2));
        \State // build up the dp hashmap
        \For{$i \in [length,(length \textless\textless 1)]$}
            \State $char[ ]\ bitRep \gets$ Binary representation of i
            \For{$c \in [1,BitRep.length]$}
                \If{bitRep[c] == '1'}
                    \State $bestVal \gets$ MAX\_VALUE;
                    \For{$k \in [1,bitRep.length]$}
                        \If{bitRep[k] == '1'}
                            \State $val \gets$ mtsp.get(cToString(bitRep,k)) + dist[k][c]
                            \If{val \textless bestVal}
                                bestVal = val
                            \EndIf
                        \EndIf
                    \EndFor
                    \State mtsp.put(cToString(bitRep, c), bestVal);
                \EndIf
            \EndFor
        \EndFor
        \State // determine the best option on how to return to the source
        \State $lastLevel \gets$ full bit representation (all 1's); $best \gets$ MAX\_VALUE
        \For{$i \in [1,lastLevel.length]$}
            \State $val \gets$ mtsp.get(cToString(lastLevel, i)) + dist[i][0];
            \If{val \textless best}
                $best \gets$ val
            \EndIf
        \EndFor
        \State Print best;
    \EndProcedure
\end{algorithmic}
\end{algorithm}

%%%%%%% Correctness %%%%%%%%%%%%%%%%%%%%%%%%%%%%%%%%%%%%%%%%%%%%%%%

\subsection{Correctness}
%%%%%%% PROPOSITION 1 %%%%%%%%%%%%%%%
\begin{proposition}
~ \\ \indent This is clearly a TSP problem for which we need to construct a fully connected graph G
to run it on. Therefore this algorithm will construct such a graph (if one exists) that the TSP
solution can be run on.
\end{proposition}

\begin{proof}
~ \\ \indent We build this graph G such that each node is either a treasure or the source. We determine
if this is possible by first recording how many treasures there are. Then we clean up the map such
that the native camps and all of their surrounding aread are marked as unreachable. Then we ensure
that all of the treasures are still there; if they are not, we know that we cannot create G. \\
\indent Next we begin to create the connections between treasures by running a bfs with the source
and each treasure as the starting point and only marking tiles where Blackbeard can travel. i.e. we
check to see if all treasures/source are reachable from eachother and how far each one is from one
another. These distances are recorded as the the edge weights between each point. Therefore, after
fully running each n+1 bfs's we have created a fully connected graph G.
\end{proof}


%%%%%%% Analysis %%%%%%%%%%%%%%%%%%%%%%%%%%%%%%%%%%%%%%%%%%%%%%%%%%
\subsection{Analysis}
For the following analysis, we will say that the map that we are given is of dimensions
$RxC$ which contains N treasures and 1 source. It should be noted that N is (in the typical case)
much smaller than $R\cdot C$.


%%%%%%% PROPOSITION 1 %%%%%%%%%%%%%%%
\begin{proposition}
\label{numq}
The \underline{space complexity} of this algorithm is \textbf{O(R$\cdot$C+N$\cdot$2$^N$)}
\end{proposition}

\begin{proof}
~ \\ \indent This is due to the fact that all of our data is stored in data structures:
\begin{itemize}
    \item \underline{map}: Stores the map $\implies R\cdot C$
    \item \underline{tresLoc}: Stores the location of each treasure on the map $\implies 2\cdot (N+1)$
    \item \underline{dist}: The adjacency matrix for all treasures and the source $\implies (N+1)^2$
    \item \underline{index}: Indicies of the treasures in the distance array based off of
            their coordinates in the map $\implies N$
    \item \underline{mtsp}: Stores all of the sub-problem solutions to tsp $\implies N\cdot 2^N$
\end{itemize}
Summing these all together we get $(R\cdot C) + (2\cdot(N+1)^3) + (N) + (N\cdot2^N)$
\begin{center}
    Giving us a space complexity of \textbf{O(R$\cdot$C+N$\cdot$2$^N$)}
\end{center}
\end{proof}

%%%%%%% PROPOSITION 2 %%%%%%%%%%%%%%%
\begin{proposition}
\label{numq}
The \underline{time complexity} of this algorithm is \textbf{O(N$\cdot$R$\cdot$C+N$\cdot$2$^N$)}
\end{proposition}

\begin{proof}
This is the case because our algorithm performs the following actions: build the map ($R\cdot C$);
alter the map ($R\cdot C$); locate the treasures ($R\cdot C$); perform a bfs witch each treasure as
the source ($N\cdot R\cdot C$); perform traveling salesman problem using all of the treasures as
nodes ($N\cdot 2^N$).  When we sum this together we get $2\cdot(R\cdot C) + (N\cdot R\cdot C) +
(N\cdot 2^N)$
\begin{center}
    Giving us a time complexity of \textbf{O(N$\cdot$R$\cdot$C+N$\cdot$2$^N$)}
\end{center}
\end{proof}

%%%%%%% Example %%%%%%%%%%%%%%%%%%%%%%%%%%%%%%%%%%%%%%%%%%%%%%%%%%%

\subsection{An Example}
We read in the input in our map[5][5] and then convert it to our uniform form:
\[
map =
\begin{bmatrix}
 .& !& .& \#& \sim         \\
 \sim& .& .& .& \sim       \\
 \ast& .& \#& .& @         \\
 \sim& \#& \#& .& \sim     \\
 \sim& \sim& \sim& !& \sim \\
\end{bmatrix}
\rightarrow
\begin{bmatrix}
.&   !&  .& \#& \# \\
\#& \#&  .&  .& \# \\
\#& \#& \#&  .&  @ \\
\#& \#& \#&  .& \# \\
\#& \#& \#&  !& \# \\
\end{bmatrix}
\]
We see that both treasures and the source are still there so we proceed with
computing a bfs with each source and two treasures as the source producing:
\[
source =
\begin{bmatrix}
6&   \textbf{5}&  4& \#& \# \\
\#& \#&  3&  2& \# \\
\#& \#& \#&  1&  \textbf{0} \\
\#& \#& \#&  2& \# \\
\#& \#& \#&  \textbf{3}& \# \\
\end{bmatrix}
,\ t_1 =
\begin{bmatrix}
1&   \textbf{0}&  1& \#& \# \\
\#& \#&  2&  3& \# \\
\#& \#& \#&  4&  \textbf{5} \\
\#& \#& \#&  5& \# \\
\#& \#& \#&  \textbf{6}& \# \\
\end{bmatrix}
,\ t_2 =
\begin{bmatrix}
7&   \textbf{6}&  5& \#& \# \\
\#& \#&  4&  3& \# \\
\#& \#& \#&  2&  \textbf{3} \\
\#& \#& \#&  1& \# \\
\#& \#& \#&  \textbf{0}& \# \\
\end{bmatrix}
\]
Then from these we get the distance array:
\[
dist =
\begin{bmatrix}
0 & 5 & 3 \\
5 & 0 & 6 \\
3 & 6 & 0 \\
\end{bmatrix}
\]
Therefore we can now begin our Traveling salesman problem, we start with 101 and can therefore
start building up our DP solution.
\[
101 \implies \{s,t_2\} =
\begin{cases}
1002
\end{cases}
= dist[s][t_2] = dist[0][2] = 3
\]
\[
110 \implies \{s,t_1\} =
\begin{cases}
1001
\end{cases}
= dist[s][t_1] = dist[0][1] = 5
\]
\[
111 \implies \{s,t_2,t_1\} =
\begin{cases}
1011 \\
1102
\end{cases}
=
\begin{cases}
101 + dist[t_2][t_1] = 1002 + dist[1][2] = 3 + 6 = 9 \\
110 + dist[t_1][t_2] = 1001 + dist[2][1] = 5 + 6 = 11
\end{cases}
\]
\[
best \implies \{s,t_2,t_1,s\} =
\begin{cases}
1011 + d[t_1][0] = 9  + 5 = 14\\
1102 + d[t_2][0] = 11 + 3 = 14
\end{cases}
\]
Therefore we have the least cost as 14.

%%%%%%% end BlackBeard %%%%%%%%%%%%%%%%%%%%%%%%%%%%%%%%%%%%%%%%%%%
\newpage
%%%%%%% start TA_Scheduling %%%%%%%%%%%%%%%%%%%%%%%%%%%%%%%%%%%%%%%

\section{Book 7.28: TA Scheduling}

%%%%%%% Mathematical Formulation %%%%%%%%%%%%%%%%%%%%%%%%%%%%%%%%%%
\subsection{Mathematical Formulation}
\textbf{(a)} Given an input of $T$ TA's, $S$ sessions, and parameters a, b, and c where the number of sessions each TA
can hold a week is $t_i$, $a \leq t_i \leq b, \forall i \in 1..T$ and the total number of seesions that can
be held a week must be $\leq c$. Determine which TAs cover which sessions if all requirements are met. \\

\noindent \textbf{(b)} Given the above information, add in a density $d_i$ representing the minimum number of sessions that must be
held that day of the week. Again determine which TAs cover which sessions if all requirements are met.

%%%%%%% Algorithm %%%%%%%%%%%%%%%%%%%%%%%%%%%%%%%%%%%%%%%%%%%%%%%%%

\subsection{Solution}
\indent In both cases we will begin by determining if the number of TA's $T$ can acceed the minimum capacity
$c$ by checking $c \geq T\cdot a$. We then construct a circulation digraph max flow
representation with $2 + T + S + 2$ nodes: 1 source, 1 virtual source, $T$ TA, $S$ session, 1 virtual
sink, and 1 sink node. To ensure that the minimum capacity for each TA is met, we connect the source to each
TA node with capacity $a$, then, to account for this we will connect the source to the virtual source
with with a capacity $c-T\cdot a$ and the virtual source will connect to each of the TA nodes with a
capacity of $b-a$ so that the total sum of in-capacities for each TA node is $a + b - a = b$. Therefore
our total out-degree from the source node will be $T\cdot a + (c - T\cdot a) = c$. Now, according to
availablility of the TA's, we connect them to the corresponding session nodes with a capacity of 1. This
is the same build for both parts (a) and (b). \\

\indent For part \textbf{(a)}, we will assign edges from each of the sesssion nodes to
the virtual sink with capacity 1 to represent that each session can only have 1 TA. Then to ensure that
the cycle is complete and that the maximum number of sessions is $\leq c$, we connect the virtual sink
with a capacity of $c$ to the sink. \\

\indent For part \textbf{(b)}, we will add in 7 extra nodes, each representing a day with a correspoding
value of $d_i, i \in 1..7$. We continue our connections by connecting each seesion, to their correspoding
day with a capacity of 1 to represent that each session can only have 1 TA. The for each day, $i$, we connect
it directly to the sink with a capacity of $d_i$ to represent the minimum density that must be met. Next, we
connect it to the virtual sink with a capacity of $\infty$ since they have met the minimum requitrement of $d_i$.
Now we just connect the virtual sink to the sink with a capacity of $c - \sum_{i = 1}^7 d_i$ such that the
total in-capacity of the sink is  $\sum_{i = 1}^7 d_i + c - \sum_{i = 1}^{7} d_i = c$. \\

\indent Now that we are fully connected we run Ford Fulkerson and if the max flow is $\textless T\cdot a$ then we have
not fufilled the required minimum flow. Otherwise we go through each TA node and check its edges, if the
edge has a flow value, then print the TA and the Session information.  \\

\noindent For Purposes of simplicity we will show the algorithm for \textbf{(b)} as it is part (a) with an additional constaint.
\begin{algorithm}[H]
\caption{Build Network (part b)}
\begin{algorithmic}
    \Procedure{connect}{out, in, capacity}
    \EndProcedure
    \Procedure{Main}{T, S, a, b, c}
        \If{$c \geq T\cdot a$}
            Not Possible To Compute
        \EndIf
        \State $G \gets$ initialize nodes
        \For{each TA node $t \in 1..T$}
            \State G.\Call{connect}{source, t, a}
            \State G.\Call{connect}{cSource, t, (b-a)}
            \For{each session, $s\in 1..S$, that $t$ can lead}
                \State G.\Call{connect}{t, s, 1}
            \EndFor
        \EndFor
        \State \Call{connect}{source, vSource, $c-T\cdot a$}
        \For{each session, $s \in 1..S$}
            \State $d \gets$ day that $s$ occurs on
            \State G.\Call{connect}{s, d, 1}
        \EndFor
        \For{each day, $i$}
            \State $d_i \gets$ minimum sessions day $i$ must have
            \State G.\Call{connect}{i, sink, $d_i$}
            \State G.\Call{connect}{i, vSink, $\infty$}
        \EndFor
        \State G.\Call{connect}{vSink, sink, $c - \sum_{i = 1}^{7} d_i$}
        \State G.\Call{MaxFlow}{ }
        \For{edge, $e$, from source to not vSource}
            \If{e.flowValue() != a}
                Not Possible To Compute
            \EndIf
        \EndFor
        \For{each day, $i$}
            \State $d_i \gets$ minimum sessions day $i$ must have
            \State edge, $e$, from $i$ to sink
            \If{e.flowValue() != $d_i$}
                Not Possible To Compute
            \EndIf
        \EndFor
        \For{each TA node $t \in 1..T$}
            \For{edge, $e$, from t}
                \If{e.flowValue() == 1}
                    \State session $s \gets$ e.to()
                    \State Print $t$ and $s$
                \EndIf
            \EndFor
        \EndFor
    \EndProcedure
\end{algorithmic}
\end{algorithm}


%%%%%%% Correctness %%%%%%%%%%%%%%%%%%%%%%%%%%%%%%%%%%%%%%%%%%%%%%%
\newpage
\subsection{Correctness}
%%%%%%% PROPOSITION 1 %%%%%%%%%%%%%%%
\begin{proposition}
~ \\ \indent This algorithm will determine (if one exists) a valid schedule for office hours,
specifying which TA will cover which time slots. If one does not exist, then it will report so.
\end{proposition}

\begin{proof}
~ \\ \indent Above we have described how to construct the flow network in detail. As so we
show now that this supports the constraints set forth by the question. The first constraint
is that each TA must have a minimum of $a$ office hours and a maximum of $b$ each week. It is
simple to upper bound each TA by $b$ by ensuring that the sum of the maximum capacity of all
edges coming into each TA is equal to $b$. Therefore the only thing we need to constrain is
that we have at least $a$ for each TA. We acheive this by making an edge of maximum capacity
$a$ between the source and each TA, which will be filled out first. Similarly we will have a
"whatever we have left" node which we call the virtual source. The connections from the
virtual source, we connect to each TA node with edges with max capacity of $b-a$ to ensure
that each node has and input of $b$. \\
\indent Now to account for the second constraint that at most $c$ office hours may occur each
week. We do this by making the maximum capacity of all edges leaving the source and entering
the sink to be $c$. Therefore (since $T\cdot a$ has already left the source going to each TA),
we connect the source to the virtual source with a value of $c - T\cdot a$. \\
\indent The last constraint of the problem is that each day $i$ have a specific density $d_i$
that acts as a minimum value. To account for this, we have each Session connect to its
respective day with a max capacity of 1 so that only one TA can be present at each session.
Then from each day, we will connect directly to the sink with maximum capacity $d_i$ therefore
we need to adjust our edge capacity from the virtual sink to the sink to be $c - \sum d_i$ and
connect each day to the virtual sink with max capacity $\infty$. \\
\indent Now we have shown that each constraint has been accounted for, but to ensure that each
one has held up, we simply will check the flow values of each of the edges from the source to
the TA nodes are full and those from each day to the sink are full.
\end{proof}


%%%%%%% Analysis %%%%%%%%%%%%%%%%%%%%%%%%%%%%%%%%%%%%%%%%%%%%%%%%%%
\subsection{Analysis}
For this section we will use the following variables: $T$ for number of TAs, S for number of
sessions, C for number of connections between TAs and Sessions, $V = T + S + 7 + 4$ for total
number of nodes and $E = 2\cdot T + C + 2\cdot S$ for total number of edges. We say then that
we can make our calculations based off V + E.

%%%%%%% PROPOSITION 1 %%%%%%%%%%%%%%%
\begin{proposition}
\label{numq}
The \underline{space complexity} of this algorithm is \textbf{O(V+E)}
\end{proposition}

\begin{proof}
~ \\ \indent This is due to the fact that we only store the nodes and the connections between
them in a datastructure and the rest is stored in constant variables.
\end{proof}

%%%%%%% PROPOSITION 2 %%%%%%%%%%%%%%%
\begin{proposition}
\label{numq}
The \underline{time complexity} of this algorithm is \textbf{O(T$\cdot$S + MaxFlowValue$\cdot$E + C)}
\end{proposition}

\begin{proof}
~ \\ \indent We can break this down into the loops which we have described in our algorithm:
\begin{enumerate}
    \item TA Node Connection: we go through each TA node (T) and connect it to the source and
        virtual source (2) and then go through every session and see if it is a valid time
        for each TA to work (S) $\implies T\cdot (2+S) \implies O(T\cdot S)$
    \item Session Node: we have already connected each session to its corresponding TAs, now
        we just connect each one (S) to the specific day that it belongs to (1) $\implies O(S)$
    \item Days: when we have to connect each day to the sink and virtual sink $(7\cdot 2)$ which
        is considered arbitrary in the large cases $\implies O(1)$
    \item MAX FLOW: Known to be $O(MaxFlowValue\cdot E)$
    \item Check TAs and days: we check that each TA and day have met the minimum requirements
        $\implies O(T)$
    \item Print out: Go through every edge going from TA to Session nodes $\implies O(C)$
\end{enumerate}
Thus summing to be $T\cdot S + S + 1 + MaxFlowValue\cdot E + T + C$ which in Big-O becomes
\textbf{O(T$\cdot$S + MaxFlowValue$\cdot$E + C)}
\end{proof}


%%%%%%% end TA_Scheduling %%%%%%%%%%%%%%%%%%%%%%%%%%%%%%%%%%%%%%%%%
\newpage
%%%%%%% start Resource Allocation %%%%%%%%%%%%%%%%%%%%%%%%%%%%%%%%%

\section{Book 8.4: Resource Allocation}

%%%%%%% Mathematical Formulation %%%%%%%%%%%%%%%%%%%%%%%%%%%%%%%%%%
\subsection*{Mathematical Formulation}
Given an input of $n$ processes, $P = \{p_1, p_2, ..., p_n\}$,  and a set of $m$
resources, $R = \{r_1, r_2, ..., r_m\}$, each process requires a set of resources
$R^{\ast}, R^{\ast} \subseteq R$. Each process in required active iff every resource
$r \in R^{\ast}$ is allocated to it, however each resource can only be used once.
Given a number $k \textgreater 0$, determine if resources $\in R$ can be allocated
so that at least $k$ processes $\in P$ will be active. For the following cases,
give a polynomial algorithm or proove it is NP-Complete.
\begin{enumerate}[(a)]
    \item General Case i.e. $k \textgreater 0$
    \item $k = 2$
    \item Have 2 types of resources, if either one is fufilled for a process, then
        the process is considered active
    \item Each resource can be allocated a maximum of 2 times
\end{enumerate}

\subsection{Part (a)}
\begin{proposition}
This is an NP-Complete problem.
\end{proposition}

\begin{proof}
~ \\ \indent We show first that this problem is NP. Given a set of k processes, check
to see if there are resources shared between them all which take $k\cdot m^{\ast} = totalResources$
time where $m^{\ast}$ is the average number of resources all of the processes contain.
More specifically we can create a HashSet which we will add each resource as we see them
in each set. Therefore we will check to see if the resource is already present in the
set, if it is then there is a repetition; if we can get to the end, then there are no more. \\
\indent Now we say that this is an NP-Hard problem by doing a reduction from the Independent
Set problem i.e. $IndependentSet \leq_P Problem$. Assume that we have an independent set of
graph $G$ and the number $k$. We take the verticies, $V \in G$ to represent each process and
all edges $E \in G$ to be resources. Therefore verticies are only considered to be adjacent if
they have a common resource. So if \underline{there exists an independent set}, $G^{\ast}\in G$
s.t. $|G^{\ast}| \geq k$, we say that this implies that verticies within $G^{\ast}$ have no
common edges, i.e. there are no shared resources $\implies$ solved the problem. Now if
\underline{there exist k processes} which resources from a disjoint set, then the corresponding
graph would have no edges in common for these processes $\implies$ independent set. \\
\indent Since Independent Set is NP-hard $\implies$ this problem is at least NP-Hard
\begin{center}$\therefore$ this process is both NP and NP-Hard $\implies$ NP-Complete.\end{center}
\end{proof}

\subsection{Part (b)}
This can be solved in polynomial time by brute force since there only exist $n^2$ process
combinations. Therefore doing the above mentioned algorithm for checking if sets are indeed
valid, we can go through all pairs; if at any pair they have a disjoint set of resources, then
we accept, if we go through every possible combination and never get into an accepting state,
we will say it is impossible.

\subsection{Part (c) and Part (d)}
These are seen as more specialized versions of problem (a), therefore we can can generalize
and say that these will also be NP-Complete.
%%%%%%% end Resource Allocation %%%%%%%%%%%%%%%%%%%%%%%%%%%%%%%%%%%

%%%%%%% end Solution %%%%%%%%%%%%%%%%%%%%%%%%%%%%%%%%%%%%%%%%%%%%%%

\end{document}