% Use this template to write your solutions to COS 423 problem sets

\documentclass[12pt]{article}
\usepackage[utf8]{inputenc}
\usepackage{amsmath, amsfonts, amsthm, amssymb, algorithm, graphicx, mathtools, xfrac}
\usepackage[noend]{algpseudocode}
\usepackage{fancyhdr, lastpage}
\usepackage{booktabs}
\usepackage{multirow}
\usepackage{graphicx}
\usepackage{pgfplots}
\usepackage[shortlabels]{enumitem}
\usepackage[vmargin=1.20in,hmargin=1.25in,centering,letterpaper]{geometry}
\setlength{\headsep}{.50in}
\setlength{\headheight}{15pt}

% Landau notation
\DeclareMathOperator{\BigOm}{\mathcal{O}}
\newcommand{\BigOh}[1]{\BigOm\left({#1}\right)}
\DeclareMathOperator{\BigTm}{\Theta}
\newcommand{\BigTheta}[1]{\BigTm\left({#1}\right)}
\DeclareMathOperator{\BigWm}{\Omega}
\newcommand{\BigOmega}[1]{\BigWm\left({#1}\right)}
\DeclareMathOperator{\LittleOm}{\mathrm{o}}
\newcommand{\LittleOh}[1]{\LittleOm\left({#1}\right)}
\DeclareMathOperator{\LittleWm}{\omega}
\newcommand{\LittleOmega}[1]{\LittleWm\left({#1}\right)}

% argmin and argmax
\newcommand{\argmin}{\operatornamewithlimits{argmin}}
\newcommand{\argmax}{\operatornamewithlimits{argmax}}

\newcommand{\calP}{\mathcal{P}}
\newcommand{\Z}{\mathbb{Z}}
\newcommand{\R}{\mathbb{R}}
\newcommand{\Exp}{\mathbb{E}}
\newcommand{\Q}{\mathbb{Q}}
\newcommand{\sign}{\mathrm{sign\ }}
\newcommand{\abs}{\mathrm{abs\ }}
\newcommand{\eps}{\varepsilon}
\newcommand{\zo}{\{0, 1\}}
\newcommand{\SAT}{\mathit{SAT}}
\renewcommand{\P}{\mathbf{P}}
\newcommand{\NP}{\mathbf{NP}}
\newcommand{\coNP}{\co{NP}}
\newcommand{\co}[1]{\mathbf{co#1}}
\renewcommand{\Pr}{\mathop{\mathrm{Pr}}}

% theorems, lemmas, invariants, etc.
\newtheorem{theorem}{Theorem}
\newtheorem{lemma}[theorem]{Lemma}
\newtheorem{invariant}[theorem]{Invariant}
\newtheorem{corollary}[theorem]{Corollary}
\newtheorem{definition}[theorem]{Definition}
\newtheorem{property}[theorem]{Property}
\newtheorem{proposition}[theorem]{Proposition}

% piecewise functions
\newenvironment{piecewise}{\left \{\begin{array}{l@{,\ }l}}
{\end{array}\right.}

% paired delimiters
\DeclarePairedDelimiter{\ceil}{\lceil}{\rceil}
\DeclarePairedDelimiter{\floor}{\lfloor}{\rfloor}
\DeclarePairedDelimiter{\len}{|}{|}
\DeclarePairedDelimiter{\set}{\{}{\}}

\makeatletter
\@addtoreset{equation}{section}
\makeatother
\renewcommand{\theequation}{\arabic{section}.\arabic{equation}}

% algorithms
\algnewcommand\algorithmicinput{\textbf{INPUT:}}
\algnewcommand\INPUT{\item[\algorithmicinput]}
\algnewcommand\algorithmicoutput{\textbf{OUTPUT:}}
\algnewcommand\OUTPUT{\item[\algorithmicoutput]}


% Formating Macros

\pagestyle{fancy}
\lhead{\sc \hmwkClass\ $\; \;\cdot \; \;$ \hmwkSemester\ $\; \;\cdot \; \;$
Problem \hmwkAssignmentNum.\hmwkProblemNum}
%\chead{\sc Problem \hmwkAssignmentNum.\hmwkProblemNum}
%\chead{}
\rhead{\em \hmwkAuthorName\ $($\hmwkAuthorID$)$\/}
\cfoot{}
\lfoot{}
\rfoot{\sc Page\ \thepage\ of\ \protect\pageref{LastPage}}
\renewcommand\headrulewidth{0.4pt}
\renewcommand\footrulewidth{0.4pt}

\fancypagestyle{fancycollab}
{
    \lfoot{\textit{Collaborators: \hmwkCollaborators}}
}

\fancypagestyle{problemstatement}
{
    \rhead{}
    \lfoot{}
}

%%%%%% Begin document with header and title %%%%%%%%%%%%%%%%%%%%%%%%%

\begin{document}

%%%%%% Header Information %%%%%%%%%%%%%%%%%%%%%%%%%%%%%%%%%%%%%%%%%%%

%%% Shouldn't need to change these
\newcommand{\hmwkClass}{COS 255}
\newcommand{\hmwkSemester}{Spring 2016}

%%% Your name, in standard First Last format
\newcommand{\hmwkAuthorName}{Lukas Leung}
%%% Your NetID
\newcommand{\hmwkAuthorID}{lleung}

%%% The problem set number (just the number)
\newcommand{\hmwkAssignmentNum}{4}

%%% The problem number (just the number)
\newcommand{\hmwkProblemNum}{0}

%%% A list of your collaborators' NetIDs, separated by ", ".
%%% You can use a new line ("\\") in the middle to prevent a long
%%% list from overflowing.
\newcommand{\hmwkCollaborators}{}
%%% Sets the collaborator list to appear on the first page
\thispagestyle{fancycollab}

%%%%%%% begin Solution %%%%%%%%%%%%%%%%%%%%%%%%%%%%%%%%%%%%%%%%%%%%

%%%%%%% start Resource Allocation %%%%%%%%%%%%%%%%%%%%%%%%%%%%%%%%%

\section{Book 8.4: Resource Allocation}

%%%%%%% Mathematical Formulation %%%%%%%%%%%%%%%%%%%%%%%%%%%%%%%%%%
\subsection*{Mathematical Formulation}
Given an input of $n$ processes, $P = \{p_1, p_2, ..., p_n\}$,  and a set of $m$
resources, $R = \{r_1, r_2, ..., r_m\}$, each process requires a set of resources
$R^{\ast}, R^{\ast} \subseteq R$. Each process in required active iff every resource
$r \in R^{\ast}$ is allocated to it, however each resource can only be used once.
Given a number $k \textgreater 0$, determine if resources $\in R$ can be allocated
so that at least $k$ processes $\in P$ will be active. For the following cases,
give a polynomial algorithm or proove it is NP-Complete.
\begin{enumerate}[(a)]
    \item General Case i.e. $k \textgreater 0$
    \item $k = 2$
    \item Have 2 types of resources, if either one is fufilled for a process, then
        the process is considered active
    \item Each resource can be allocated a maximum of 2 times
\end{enumerate}

\subsection{Part (a)}
\begin{proposition}
This is an NP-Complete problem.
\end{proposition}

\begin{proof}
~ \\ \indent We show first that this problem is NP. Given a set of k processes, check
to see if there are resources shared between them all which take $k\cdot m^{\ast} = totalResources$
time where $m^{\ast}$ is the average number of resources all of the processes contain.
More specifically we can create a HashSet which we will add each resource as we see them
in each set. Therefore we will check to see if the resource is already present in the
set, if it is then there is a repetition; if we can get to the end, then there are no more. \\
\indent Now we say that this is an NP-Hard problem by doing a reduction from the Independent
Set problem i.e. $IndependentSet \leq_P Problem$. Assume that we have an independent set of
graph $G$ and the number $k$. We take the verticies, $V \in G$ to represent each process and
all edges $E \in G$ to be resources. Therefore verticies are only considered to be adjacent if
they have a common resource. So if \underline{there exists an independent set}, $G^{\ast}\in G$
s.t. $|G^{\ast}| \geq k$, we say that this implies that verticies within $G^{\ast}$ have no
common edges, i.e. there are no shared resources $\implies$ solved the problem. Now if
\underline{there exist k processes} which resources from a disjoint set, then the corresponding
graph would have no edges in common for these processes $\implies$ independent set. \\
\indent Since Independent Set is NP-hard $\implies$ this problem is at least NP-Hard
\begin{center}$\therefore$ this process is both NP and NP-Hard $\implies$ NP-Complete.\end{center}
\end{proof}

\subsection{Part (b)}
This can be solved in polynomial time by brute force since there only exist $n^2$ process
combinations. Therefore doing the above mentioned algorithm for checking if sets are indeed
valid, we can go through all pairs; if at any pair they have a disjoint set of resources, then
we accept, if we go through every possible combination and never get into an accepting state,
we will say it is impossible.

\subsection{Part (c) and Part (d)}
These are seen as more specialized versions of problem (a), therefore we can can generalize
and say that these will also be NP-Complete.
%%%%%%% end Resource Allocation %%%%%%%%%%%%%%%%%%%%%%%%%%%%%%%%%%%


%%%%%%% end Solution %%%%%%%%%%%%%%%%%%%%%%%%%%%%%%%%%%%%%%%%%%%%%%

\end{document}