% Use this template to write your solutions to COS 423 problem sets

\documentclass[12pt]{article}
\usepackage[utf8]{inputenc}
\usepackage{amsmath, amsfonts, amsthm, amssymb, algorithm, graphicx, mathtools, xfrac}
\usepackage[noend]{algpseudocode}
\usepackage{fancyhdr, lastpage}
\usepackage{booktabs}
\usepackage{multirow}
\usepackage{graphicx}
\usepackage{pgfplots}
\usepackage[vmargin=1.20in,hmargin=1.25in,centering,letterpaper]{geometry}
\setlength{\headsep}{.50in}
\setlength{\headheight}{15pt}

% Landau notation
\DeclareMathOperator{\BigOm}{\mathcal{O}}
\newcommand{\BigOh}[1]{\BigOm\left({#1}\right)}
\DeclareMathOperator{\BigTm}{\Theta}
\newcommand{\BigTheta}[1]{\BigTm\left({#1}\right)}
\DeclareMathOperator{\BigWm}{\Omega}
\newcommand{\BigOmega}[1]{\BigWm\left({#1}\right)}
\DeclareMathOperator{\LittleOm}{\mathrm{o}}
\newcommand{\LittleOh}[1]{\LittleOm\left({#1}\right)}
\DeclareMathOperator{\LittleWm}{\omega}
\newcommand{\LittleOmega}[1]{\LittleWm\left({#1}\right)}

% argmin and argmax
\newcommand{\argmin}{\operatornamewithlimits{argmin}}
\newcommand{\argmax}{\operatornamewithlimits{argmax}}

\newcommand{\calP}{\mathcal{P}}
\newcommand{\Z}{\mathbb{Z}}
\newcommand{\R}{\mathbb{R}}
\newcommand{\Exp}{\mathbb{E}}
\newcommand{\Q}{\mathbb{Q}}
\newcommand{\sign}{\mathrm{sign\ }}
\newcommand{\abs}{\mathrm{abs\ }}
\newcommand{\eps}{\varepsilon}
\newcommand{\zo}{\{0, 1\}}
\newcommand{\SAT}{\mathit{SAT}}
\renewcommand{\P}{\mathbf{P}}
\newcommand{\NP}{\mathbf{NP}}
\newcommand{\coNP}{\co{NP}}
\newcommand{\co}[1]{\mathbf{co#1}}
\renewcommand{\Pr}{\mathop{\mathrm{Pr}}}

% theorems, lemmas, invariants, etc.
\newtheorem{theorem}{Theorem}
\newtheorem{lemma}[theorem]{Lemma}
\newtheorem{invariant}[theorem]{Invariant}
\newtheorem{corollary}[theorem]{Corollary}
\newtheorem{definition}[theorem]{Definition}
\newtheorem{property}[theorem]{Property}
\newtheorem{proposition}[theorem]{Proposition}

% piecewise functions
\newenvironment{piecewise}{\left \{\begin{array}{l@{,\ }l}}
{\end{array}\right.}

% paired delimiters
\DeclarePairedDelimiter{\ceil}{\lceil}{\rceil}
\DeclarePairedDelimiter{\floor}{\lfloor}{\rfloor}
\DeclarePairedDelimiter{\len}{|}{|}
\DeclarePairedDelimiter{\set}{\{}{\}}

\makeatletter
\@addtoreset{equation}{section}
\makeatother
\renewcommand{\theequation}{\arabic{section}.\arabic{equation}}

% algorithms
\algnewcommand\algorithmicinput{\textbf{INPUT:}}
\algnewcommand\INPUT{\item[\algorithmicinput]}
\algnewcommand\algorithmicoutput{\textbf{OUTPUT:}}
\algnewcommand\OUTPUT{\item[\algorithmicoutput]}


% Formating Macros

\pagestyle{fancy}
\lhead{\sc \hmwkClass\ $\; \;\cdot \; \;$ \hmwkSemester\ $\; \;\cdot \; \;$
Problem \hmwkAssignmentNum.\hmwkProblemNum}
%\chead{\sc Problem \hmwkAssignmentNum.\hmwkProblemNum}
%\chead{}
\rhead{\em \hmwkAuthorName\ $($\hmwkAuthorID$)$\/}
\cfoot{}
\lfoot{}
\rfoot{\sc Page\ \thepage\ of\ \protect\pageref{LastPage}}
\renewcommand\headrulewidth{0.4pt}
\renewcommand\footrulewidth{0.4pt}

\fancypagestyle{fancycollab}
{
    \lfoot{\textit{Collaborators: \hmwkCollaborators}}
}

\fancypagestyle{problemstatement}
{
    \rhead{}
    \lfoot{}
}

%%%%%% Begin document with header and title %%%%%%%%%%%%%%%%%%%%%%%%%

\begin{document}

%%%%%% Header Information %%%%%%%%%%%%%%%%%%%%%%%%%%%%%%%%%%%%%%%%%%%

%%% Shouldn't need to change these
\newcommand{\hmwkClass}{COS 255}
\newcommand{\hmwkSemester}{Spring 2016}

%%% Your name, in standard First Last format
\newcommand{\hmwkAuthorName}{Lukas Leung}
%%% Your NetID
\newcommand{\hmwkAuthorID}{lleung}

%%% The problem set number (just the number)
\newcommand{\hmwkAssignmentNum}{7}

%%% The problem number (just the number)
\newcommand{\hmwkProblemNum}{2}

%%% A list of your collaborators' NetIDs, separated by ", ".
%%% You can use a new line ("\\") in the middle to prevent a long
%%% list from overflowing.
\newcommand{\hmwkCollaborators}{}
%%% Sets the collaborator list to appear on the first page
\thispagestyle{fancycollab}

%%%%%%% begin Solution %%%%%%%%%%%%%%%%%%%%%%%%%%%%%%%%%%%%%%%%%%%%
%%%%%%% start TA_Scheduling %%%%%%%%%%%%%%%%%%%%%%%%%%%%%%%%%%%%%%%

\section{Book 7.28: TA Scheduling}

%%%%%%% Mathematical Formulation %%%%%%%%%%%%%%%%%%%%%%%%%%%%%%%%%%
\subsection{Mathematical Formulation}
\textbf{(a)} Given an input of $T$ TA's, $S$ sessions, and parameters a, b, and c where the number of sessions each TA
can hold a week is $t_i$, $a \leq t_i \leq b, \forall i \in 1..T$ and the total number of seesions that can
be held a week must be $\leq c$. Determine which TAs cover which sessions if all requirements are met. \\

\noindent \textbf{(b)} Given the above information, add in a density $d_i$ representing the minimum number of sessions that must be
held that day of the week. Again determine which TAs cover which sessions if all requirements are met.

%%%%%%% Algorithm %%%%%%%%%%%%%%%%%%%%%%%%%%%%%%%%%%%%%%%%%%%%%%%%%

\subsection{Solution}
\indent In both cases we will begin by determining if the number of TA's $T$ can acceed the minimum capacity
$c$ by checking $c \geq T\cdot a$. We then construct a circulation digraph max flow 
representation with $2 + T + S + 2$ nodes: 1 source, 1 virtual source, $T$ TA, $S$ session, 1 virtual 
sink, and 1 sink node. To ensure that the minimum capacity for each TA is met, we connect the source to each
TA node with capacity $a$, then, to account for this we will connect the source to the virtual source
with with a capacity $c-T\cdot a$ and the virtual source will connect to each of the TA nodes with a
capacity of $b-a$ so that the total sum of in-capacities for each TA node is $a + b - a = b$. Therefore
our total out-degree from the source node will be $T\cdot a + (c - T\cdot a) = c$. Now, according to 
availablility of the TA's, we connect them to the corresponding session nodes with a capacity of 1. This 
is the same build for both parts (a) and (b). \\  

\indent For part \textbf{(a)}, we will assign edges from each of the sesssion nodes to
the virtual sink with capacity 1 to represent that each session can only have 1 TA. Then to ensure that
the cycle is complete and that the maximum number of sessions is $\leq c$, we connect the virtual sink
with a capacity of $c$ to the sink. \\

\indent For part \textbf{(b)}, we will add in 7 extra nodes, each representing a day with a correspoding
value of $d_i, i \in 1..7$. We continue our connections by connecting each seesion, to their correspoding
day with a capacity of 1 to represent that each session can only have 1 TA. The for each day, $i$, we connect
it directly to the sink with a capacity of $d_i$ to represent the minimum density that must be met. Next, we 
connect it to the virtual sink with a capacity of $\infty$ since they have met the minimum requitrement of $d_i$.
Now we just connect the virtual sink to the sink with a capacity of $c - \sum_{i = 1}^7 d_i$ such that the
total in-capacity of the sink is  $\sum_{i = 1}^7 d_i + c - \sum_{i = 1}^{7} d_i = c$. \\

\indent Now that we are fully connected we run Ford Fulkerson and if the max flow is $\textless T\cdot a$ then we have
not fufilled the required minimum flow. Otherwise we go through each TA node and check its edges, if the 
edge has a flow value, then print the TA and the Session information.  \\

\noindent For Purposes of simplicity we will show the algorithm for \textbf{(b)} as it is part (a) with an additional constaint.
\begin{algorithm}[H]
\caption{Build Network (part b)}
\begin{algorithmic}
    \Procedure{connect}{out, in, capacity}
    \EndProcedure
    \Procedure{Main}{T, S, a, b, c}
        \If{$c \geq T\cdot a$}
            Not Possible To Compute
        \EndIf
        \State $G \gets$ initialize nodes
        \For{each TA node $t \in 1..T$}
            \State G.\Call{connect}{source, t, a}
            \State G.\Call{connect}{cSource, t, (b-a)}
            \For{each session, $s\in 1..S$, that $t$ can lead}
                \State G.\Call{connect}{t, s, 1}
            \EndFor
        \EndFor
        \State \Call{connect}{source, vSource, $c-T\cdot a$}
        \For{each session, $s \in 1..S$}
            \State $d \gets$ day that $s$ occurs on
            \State G.\Call{connect}{s, d, 1}
        \EndFor
        \For{each day, $i$}
            \State $d_i \gets$ minimum sessions day $i$ must have
            \State G.\Call{connect}{i, sink, $d_i$}
            \State G.\Call{connect}{i, vSink, $\infty$}
        \EndFor
        \State G.\Call{connect}{vSink, sink, $c - \sum_{i = 1}^{7} d_i$}
        \State G.\Call{MaxFlow}{ }
        \For{edge, $e$, from source to not vSource}
            \If{e.flowValue() != a}
                Not Possible To Compute
            \EndIf
        \EndFor
        \For{each day, $i$}
            \State $d_i \gets$ minimum sessions day $i$ must have
            \State edge, $e$, from $i$ to sink
            \If{e.flowValue() != $d_i$}
                Not Possible To Compute
            \EndIf
        \EndFor
        \For{each TA node $t \in 1..T$}
            \For{edge, $e$, from t}
                \If{e.flowValue() == 1}
                    \State session $s \gets$ e.to()
                    \State Print $t$ and $s$
                \EndIf
            \EndFor
        \EndFor
    \EndProcedure
\end{algorithmic}
\end{algorithm}


%%%%%%% Correctness %%%%%%%%%%%%%%%%%%%%%%%%%%%%%%%%%%%%%%%%%%%%%%%
\newpage
\subsection{Correctness}
%%%%%%% PROPOSITION 1 %%%%%%%%%%%%%%%
\begin{proposition}
~ \\ \indent This algorithm will determine (if one exists) a valid schedule for office hours,
specifying which TA will cover which time slots. If one does not exist, then it will report so.
\end{proposition}

\begin{proof}
~ \\ \indent Above we have described how to construct the flow network in detail. As so we
show now that this supports the constraints set forth by the question. The first constraint
is that each TA must have a minimum of $a$ office hours and a maximum of $b$ each week. It is
simple to upper bound each TA by $b$ by ensuring that the sum of the maximum capacity of all
edges coming into each TA is equal to $b$. Therefore the only thing we need to constrain is
that we have at least $a$ for each TA. We acheive this by making an edge of maximum capacity
$a$ between the source and each TA, which will be filled out first. Similarly we will have a
"whatever we have left" node which we call the virtual source. The connections from the
virtual source, we connect to each TA node with edges with max capacity of $b-a$ to ensure
that each node has and input of $b$. \\
\indent Now to account for the second constraint that at most $c$ office hours may occur each
week. We do this by making the maximum capacity of all edges leaving the source and entering
the sink to be $c$. Therefore (since $T\cdot a$ has already left the source going to each TA),
we connect the source to the virtual source with a value of $c - T\cdot a$. \\
\indent The last constraint of the problem is that each day $i$ have a specific density $d_i$
that acts as a minimum value. To account for this, we have each Session connect to its
respective day with a max capacity of 1 so that only one TA can be present at each session.
Then from each day, we will connect directly to the sink with maximum capacity $d_i$ therefore
we need to adjust our edge capacity from the virtual sink to the sink to be $c - \sum d_i$ and
connect each day to the virtual sink with max capacity $\infty$. \\
\indent Now we have shown that each constraint has been accounted for, but to ensure that each
one has held up, we simply will check the flow values of each of the edges from the source to
the TA nodes are full and those from each day to the sink are full.
\end{proof}


%%%%%%% Analysis %%%%%%%%%%%%%%%%%%%%%%%%%%%%%%%%%%%%%%%%%%%%%%%%%%
\subsection{Analysis}
For this section we will use the following variables: $T$ for number of TAs, S for number of
sessions, C for number of connections between TAs and Sessions, $V = T + S + 7 + 4$ for total
number of nodes and $E = 2\cdot T + C + 2\cdot S$ for total number of edges. We say then that
we can make our calculations based off V + E.

%%%%%%% PROPOSITION 1 %%%%%%%%%%%%%%%
\begin{proposition}
\label{numq}
The \underline{space complexity} of this algorithm is \textbf{O(V+E)}
\end{proposition}

\begin{proof}
~ \\ \indent This is due to the fact that we only store the nodes and the connections between
them in a datastructure and the rest is stored in constant variables.
\end{proof}

%%%%%%% PROPOSITION 2 %%%%%%%%%%%%%%%
\begin{proposition}
\label{numq}
The \underline{time complexity} of this algorithm is \textbf{O(T$\cdot$S + MaxFlowValue$\cdot$E + C)}
\end{proposition}

\begin{proof}
~ \\ \indent We can break this down into the loops which we have described in our algorithm:
\begin{enumerate}
    \item TA Node Connection: we go through each TA node (T) and connect it to the source and
        virtual source (2) and then go through every session and see if it is a valid time
        for each TA to work (S) $\implies T\cdot (2+S) \implies O(T\cdot S)$
    \item Session Node: we have already connected each session to its corresponding TAs, now
        we just connect each one (S) to the specific day that it belongs to (1) $\implies O(S)$
    \item Days: when we have to connect each day to the sink and virtual sink $(7\cdot 2)$ which
        is considered arbitrary in the large cases $\implies O(1)$
    \item MAX FLOW: Known to be $O(MaxFlowValue\cdot E)$
    \item Check TAs and days: we check that each TA and day have met the minimum requirements
        $\implies O(T)$
    \item Print out: Go through every edge going from TA to Session nodes $\implies O(C)$
\end{enumerate}
Thus summing to be $T\cdot S + S + 1 + MaxFlowValue\cdot E + T + C$ which in Big-O becomes
\textbf{O(T$\cdot$S + MaxFlowValue$\cdot$E + C)}
\end{proof}


%%%%%%% end TA_Scheduling %%%%%%%%%%%%%%%%%%%%%%%%%%%%%%%%%%%%%%%%%


%%%%%%% end Solution %%%%%%%%%%%%%%%%%%%%%%%%%%%%%%%%%%%%%%%%%%%%%%

\end{document}
