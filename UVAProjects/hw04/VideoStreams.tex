% Use this template to write your solutions to COS 423 problem sets

\documentclass[12pt]{article}
\usepackage[utf8]{inputenc}
\usepackage{amsmath, amsfonts, amsthm, amssymb, algorithm, graphicx, mathtools, xfrac}
\usepackage[noend]{algpseudocode}
\usepackage{fancyhdr, lastpage}
\usepackage{booktabs}
\usepackage{multirow}
\usepackage{graphicx}
\usepackage{pgfplots}
\usepackage[vmargin=1.20in,hmargin=1.25in,centering,letterpaper]{geometry}
\setlength{\headsep}{.50in}
\setlength{\headheight}{15pt}

% Landau notation
\DeclareMathOperator{\BigOm}{\mathcal{O}}
\newcommand{\BigOh}[1]{\BigOm\left({#1}\right)}
\DeclareMathOperator{\BigTm}{\Theta}
\newcommand{\BigTheta}[1]{\BigTm\left({#1}\right)}
\DeclareMathOperator{\BigWm}{\Omega}
\newcommand{\BigOmega}[1]{\BigWm\left({#1}\right)}
\DeclareMathOperator{\LittleOm}{\mathrm{o}}
\newcommand{\LittleOh}[1]{\LittleOm\left({#1}\right)}
\DeclareMathOperator{\LittleWm}{\omega}
\newcommand{\LittleOmega}[1]{\LittleWm\left({#1}\right)}

% argmin and argmax
\newcommand{\argmin}{\operatornamewithlimits{argmin}}
\newcommand{\argmax}{\operatornamewithlimits{argmax}}

\newcommand{\calP}{\mathcal{P}}
\newcommand{\Z}{\mathbb{Z}}
\newcommand{\R}{\mathbb{R}}
\newcommand{\Exp}{\mathbb{E}}
\newcommand{\Q}{\mathbb{Q}}
\newcommand{\sign}{\mathrm{sign\ }}
\newcommand{\abs}{\mathrm{abs\ }}
\newcommand{\eps}{\varepsilon}
\newcommand{\zo}{\{0, 1\}}
\newcommand{\SAT}{\mathit{SAT}}
\renewcommand{\P}{\mathbf{P}}
\newcommand{\NP}{\mathbf{NP}}
\newcommand{\coNP}{\co{NP}}
\newcommand{\co}[1]{\mathbf{co#1}}
\renewcommand{\Pr}{\mathop{\mathrm{Pr}}}

% theorems, lemmas, invariants, etc.
\newtheorem{theorem}{Theorem}
\newtheorem{lemma}[theorem]{Lemma}
\newtheorem{invariant}[theorem]{Invariant}
\newtheorem{corollary}[theorem]{Corollary}
\newtheorem{definition}[theorem]{Definition}
\newtheorem{property}[theorem]{Property}
\newtheorem{proposition}[theorem]{Proposition}

% piecewise functions
\newenvironment{piecewise}{\left \{\begin{array}{l@{,\ }l}}
{\end{array}\right.}

% paired delimiters
\DeclarePairedDelimiter{\ceil}{\lceil}{\rceil}
\DeclarePairedDelimiter{\floor}{\lfloor}{\rfloor}
\DeclarePairedDelimiter{\len}{|}{|}
\DeclarePairedDelimiter{\set}{\{}{\}}

\makeatletter
\@addtoreset{equation}{section}
\makeatother
\renewcommand{\theequation}{\arabic{section}.\arabic{equation}}

% algorithms
\algnewcommand\algorithmicinput{\textbf{INPUT:}}
\algnewcommand\INPUT{\item[\algorithmicinput]}
\algnewcommand\algorithmicoutput{\textbf{OUTPUT:}}
\algnewcommand\OUTPUT{\item[\algorithmicoutput]}


% Formating Macros

\pagestyle{fancy}
\lhead{\sc \hmwkClass\ $\; \;\cdot \; \;$ \hmwkSemester\ $\; \;\cdot \; \;$
Problem \hmwkAssignmentNum.\hmwkProblemNum}
%\chead{\sc Problem \hmwkAssignmentNum.\hmwkProblemNum}
%\chead{}
\rhead{\em \hmwkAuthorName\ $($\hmwkAuthorID$)$\/}
\cfoot{}
\lfoot{}
\rfoot{\sc Page\ \thepage\ of\ \protect\pageref{LastPage}}
\renewcommand\headrulewidth{0.4pt}
\renewcommand\footrulewidth{0.4pt}

\fancypagestyle{fancycollab}
{
    \lfoot{\textit{Collaborators: \hmwkCollaborators}}
}

\fancypagestyle{problemstatement}
{
    \rhead{}
    \lfoot{}
}

%%%%%% Begin document with header and title %%%%%%%%%%%%%%%%%%%%%%%%%

\begin{document}

%%%%%% Header Information %%%%%%%%%%%%%%%%%%%%%%%%%%%%%%%%%%%%%%%%%%%

%%% Shouldn't need to change these
\newcommand{\hmwkClass}{COS 255}
\newcommand{\hmwkSemester}{Spring 2016}

%%% Your name, in standard First Last format
\newcommand{\hmwkAuthorName}{Lukas Leung}
%%% Your NetID
\newcommand{\hmwkAuthorID}{lleung}

%%% The problem set number (just the number)
\newcommand{\hmwkAssignmentNum}{4}

%%% The problem number (just the number)
\newcommand{\hmwkProblemNum}{12}

%%% A list of your collaborators' NetIDs, separated by ", ".
%%% You can use a new line ("\\") in the middle to prevent a long
%%% list from overflowing.
\newcommand{\hmwkCollaborators}{}
%%% Sets the collaborator list to appear on the first page
\thispagestyle{fancycollab}

%%%%%%% begin Solution %%%%%%%%%%%%%%%%%%%%%%%%%%%%%%%%%%%%%%%%%%%%
%%%%%%% start Video Streams %%%%%%%%%%%%%%%%%%%%%%%%%%%%%%%%%%%%%%%

\section{Book 4.12: Video Streams}
\textbf{Background} \\
~\indent Suppose you have $n$ video streams that need to be sent,
one after another, over a communication link. Stream $i$ consists
of a total of $b_i$ bits that need to be sent, at a constant rate
over a period of $t_i$ seconds. You cannot send two streams at the
same time, so you need to determine a $schedule$ for the streams:
an order in which to send them. Whichever order you choose, there
cannot be any delays between the end of one stream and the start
of the next. Suppose your schedule starts at time 0 (and therefore
ends at time $\sum_{i=1}^{n} t_i$, whichever order you choose). We
assume that all the values $b_i$ and $t_i$ are positive integers. \\
\indent Now, because you're just one user, the link does not want
you taking up too much bandwidth, so it imposes the following
constraint, using a fixed parameter $r$.

~ \\ \indent \textit{($\ast$) For each natural number t
$\textgreater$ 0, the total number of bits you send over the \\
~ \indent time interval from 0 to t cannot exceed r$\cdot$t.} \\

Note that this constraint is only imposed for time intervals that
start at 0, $not$ for time intervals that start at any other value.
We say that a schedule is $valid$ if it satisfies the constraint
($\ast$) imposed by the link. \\

\textbf{Questions:} \\
~\indent Given a set of $n$ streams, each specified by its number
of bits $b_i$ and its time duration $t_i$, as well as the link
parameter r, determine whether $\exists$ a valid schedule.

\subsubsection{Question 1:}
\begin{center}\textit{Claim: $\exists$ a valid schedule $\iff$
each stream i satisfies $b_i\leq r\cdot t_i$.}\end{center}
Decide whether you think the claim is true or false, and give a
proof of either the claim or its negation.

\subsubsection{Question 2:}
Give an algorithm that takes a set of $n$ streams, each specified
by its number of bits $b_i$ and its time duration $t_i$, as well as
the link parameter $r$, and determines whether there exists a valid
schedule. The running time of your algorithm should be polynomial
in $n$.

%%%%%%% end Problem %%%%%%%%%%%%%%%%%%%%%%%%%%%%%%%%%%%%%%%%%%%%%%%

\newpage
%%%%%%% first Problem %%%%%%%%%%%%%%%%%%%%%%%%%%%%%%%%%%%%%%%%%%%%%
\subsection{Question 1:}
\textbf{False.} \\
Proof by example, giving the case:  $s_1$ = (1,1), $s_2$ = (2,1)
and $r$ = 1. We see here that this is clearly an acceptable order as
there is no space and for \underline{t=1} $\implies$ $b_1 \leq r\cdot t$
i.e. $1 \leq 1\cdot 1$ so $\ast$ holds. Similarly for \underline{t=2}
$\implies$ $b_2 \leq r\cdot t$ i.e. $2 \leq 1\cdot 2$ so $\ast$ holds.
However this will fail the case since for $s_2$, $b_2 \nleq r\cdot t_2$
i.e. $2 \nleq 1\cdot 1 \implies$ fails the case. $\square$

\subsection{Question 2:}
%%%%%%% Mathematical Formulation %%%%%%%%%%%%%%%%%%%%%%%%%%%%%%%%%%
\subsubsection{Mathematical Formulation}
Given an input of $n$ video streams, $s_1 = (b_1,t_1), s_2 = (b_2, t_1),
..., s_n = (b_n,t_n)$ where $b_i$ = the load of stream $i$ and $t_i$ =
the duration for how long it takes to transmit all data. We will say
that in the correct order, there cannot be any delays between the end
of one stream and the start of the next. i.e. time starts t=1 and goes
to $\sum_{i=1}^{n} t_i$. We will denote the current time as $T_i$
for when each step $i$ begins where $T_1 = 1$. Additionally there is a
rule that $b_i \leq r\cdot T_i$ for each stream $i$ where $r$ is a
fixed parameter.

%%%%%%% Algorithm %%%%%%%%%%%%%%%%%%%%%%%%%%%%%%%%%%%%%%%%%%%%%%%%%
\subsubsection{Solution}
Important Confusing Data Structures:
\begin{itemize}
    \item VideoStream[n] \textbf{streams} : ~ Holds the video streams
    storing them as objects able to retrieve their $(d_i, t_i)$
\end{itemize}

\noindent The main functionality of this is to sort the video streams
by $d_i/t_i$, and then check to see if the condition is met. If at
any point it is not, then there does not exist a valid arrangement.

\begin{algorithm}[H]
\caption{Method}
\begin{algorithmic}
    \Procedure{sortStreams}{streams, r}
        \State \Call{sort}{streams, Comparator()}
        \State $T \gets$ 1
        \For{$s_i \in streams$}
            \If{$b_i\ \textgreater\ r\cdot T$}
                \State return null
            \EndIf
            \State $T += t_i$
        \EndFor
        \State return streams
    \EndProcedure
    \Procedure{compare}{$d_a,t_a,d_b,t_b$}
        \State $pos_a \gets d_a/t_a$
        \State $pos_b \gets d_b/t_b$
        \State return $pos_a - pos_b$
    \EndProcedure
\end{algorithmic}
\end{algorithm}


%%%%%%% Correctness %%%%%%%%%%%%%%%%%%%%%%%%%%%%%%%%%%%%%%%%%%%%%%%

\subsubsection{Correctness}
%%%%%%% PROPOSITION 1 %%%%%%%%%%%%%%%
\begin{proposition}
~ \\ \indent Propose that by sorting via $d_i/t_i$ we gaurantee
$\exists$ a valid ordered input.
\end{proposition}

\begin{proof}
~ \\ \indent The motivation behind this is using the equality
$d_i \leq r\cdot T_i \implies d_i/T_i \leq r$ must also hold. Therefore,
we see that the $i^{th}$ stream is depenent on the ratio between the
load and the current time. $\therefore$ We develop our rational this
way by saying that if we order them then in the order of their
ratio, then those with lower $b_i$ compared to $t_i$ would appear
first, i.e. $d_1/t_1 \leq d_2/t_2 \leq ... \leq d_n/t_n$. If $\exists$
a valid output then this will give us a valid output. This is due
to the fact that if you were to rearrange any part, then the ratios
would be inverted and then streams would be worse off.
\end{proof}


%%%%%%% Analysis %%%%%%%%%%%%%%%%%%%%%%%%%%%%%%%%%%%%%%%%%%%%%%%%%%
\subsubsection{Analysis}
\begin{center}Let $N$ be the total number of streams being considered.\end{center}

%%%%%%% PROPOSITION 1 %%%%%%%%%%%%%%%
\begin{proposition}
\label{numq}
The \underline{space complexity} of this algorithm is \textbf{O(N)}
\end{proposition}

\begin{proof}
~ \\ \indent This is due to the fact that all of our data is stored
within the object array of size $N$ and two integers $T\ and\ r$.
\begin{center}
    Giving us a space complexity of \textbf{O(N)}
\end{center}
\end{proof}

%%%%%%% PROPOSITION 2 %%%%%%%%%%%%%%%
\begin{proposition}
\label{numq}
The \underline{time complexity} of this algorithm is \textbf{O(N$\cdot$log(N))}
\end{proposition}

\begin{proof}
This is the case because our algorithm takes ($N\cdot log(N)$) to sort
based off of this comparator and then linearly searches through, ($N$),
doing the single check statement (1)
\begin{center}
    Giving us a time complexity of \textbf{O(N$\cdot$log(N))}
\end{center}
\end{proof}

%%%%%%% end Video Streams %%%%%%%%%%%%%%%%%%%%%%%%%%%%%%%%%%%%%%%%%


%%%%%%% end Solution %%%%%%%%%%%%%%%%%%%%%%%%%%%%%%%%%%%%%%%%%%%%%%

\end{document}