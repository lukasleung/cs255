% java BurrowsWheeler - < inputs\dickens.txt | java MoveToFront - | java Huffman - > inputs\dickens_compressed.txt
% du -b dickens.txt | awk '{ print $1, "* 8" }' | bc
% Use this template to write your solutions to COS 423 problem sets

\documentclass[12pt]{article}
\usepackage[utf8]{inputenc}
\usepackage{amsmath, amsfonts, amsthm, amssymb, algorithm, graphicx, mathtools, xfrac}
\usepackage[noend]{algpseudocode}
\usepackage{fancyhdr, lastpage}
\usepackage{booktabs}
\usepackage{multirow}
\usepackage{graphicx}
\usepackage{pgfplots}
\usepackage[vmargin=1.20in,hmargin=1.25in,centering,letterpaper]{geometry}
\setlength{\headsep}{.50in}
\setlength{\headheight}{15pt}

% Landau notation
\DeclareMathOperator{\BigOm}{\mathcal{O}}
\newcommand{\BigOh}[1]{\BigOm\left({#1}\right)}
\DeclareMathOperator{\BigTm}{\Theta}
\newcommand{\BigTheta}[1]{\BigTm\left({#1}\right)}
\DeclareMathOperator{\BigWm}{\Omega}
\newcommand{\BigOmega}[1]{\BigWm\left({#1}\right)}
\DeclareMathOperator{\LittleOm}{\mathrm{o}}
\newcommand{\LittleOh}[1]{\LittleOm\left({#1}\right)}
\DeclareMathOperator{\LittleWm}{\omega}
\newcommand{\LittleOmega}[1]{\LittleWm\left({#1}\right)}

% argmin and argmax
\newcommand{\argmin}{\operatornamewithlimits{argmin}}
\newcommand{\argmax}{\operatornamewithlimits{argmax}}

\newcommand{\calP}{\mathcal{P}}
\newcommand{\Z}{\mathbb{Z}}
\newcommand{\R}{\mathbb{R}}
\newcommand{\Exp}{\mathbb{E}}
\newcommand{\Q}{\mathbb{Q}}
\newcommand{\sign}{\mathrm{sign\ }}
\newcommand{\abs}{\mathrm{abs\ }}
\newcommand{\eps}{\varepsilon}
\newcommand{\zo}{\{0, 1\}}
\newcommand{\SAT}{\mathit{SAT}}
\renewcommand{\P}{\mathbf{P}}
\newcommand{\NP}{\mathbf{NP}}
\newcommand{\coNP}{\co{NP}}
\newcommand{\co}[1]{\mathbf{co#1}}
\renewcommand{\Pr}{\mathop{\mathrm{Pr}}}

% theorems, lemmas, invariants, etc.
\newtheorem{theorem}{Theorem}
\newtheorem{lemma}[theorem]{Lemma}
\newtheorem{invariant}[theorem]{Invariant}
\newtheorem{corollary}[theorem]{Corollary}
\newtheorem{definition}[theorem]{Definition}
\newtheorem{property}[theorem]{Property}
\newtheorem{proposition}[theorem]{Proposition}

% piecewise functions
\newenvironment{piecewise}{\left \{\begin{array}{l@{,\ }l}}
{\end{array}\right.}

% paired delimiters
\DeclarePairedDelimiter{\ceil}{\lceil}{\rceil}
\DeclarePairedDelimiter{\floor}{\lfloor}{\rfloor}
\DeclarePairedDelimiter{\len}{|}{|}
\DeclarePairedDelimiter{\set}{\{}{\}}

\makeatletter
\@addtoreset{equation}{section}
\makeatother
\renewcommand{\theequation}{\arabic{section}.\arabic{equation}}

% algorithms
\algnewcommand\algorithmicinput{\textbf{INPUT:}}
\algnewcommand\INPUT{\item[\algorithmicinput]}
\algnewcommand\algorithmicoutput{\textbf{OUTPUT:}}
\algnewcommand\OUTPUT{\item[\algorithmicoutput]}


% Formating Macros

\pagestyle{fancy}
\lhead{\sc \hmwkClass\ $\; \;\cdot \; \;$ \hmwkSemester\ $\; \;\cdot \; \;$
Problem \hmwkAssignmentNum.\hmwkProblemNum}
%\chead{\sc Problem \hmwkAssignmentNum.\hmwkProblemNum}
%\chead{}
\rhead{\em \hmwkAuthorName\ $($\hmwkAuthorID$)$\/}
\cfoot{}
\lfoot{}
\rfoot{\sc Page\ \thepage\ of\ \protect\pageref{LastPage}}
\renewcommand\headrulewidth{0.4pt}
\renewcommand\footrulewidth{0.4pt}

\fancypagestyle{fancycollab}
{
    \lfoot{\textit{Collaborators: \hmwkCollaborators}}
}

\fancypagestyle{problemstatement}
{
    \rhead{}
    \lfoot{}
}

%%%%%% Begin document with header and title %%%%%%%%%%%%%%%%%%%%%%%%%

\begin{document}

%%%%%% Header Information %%%%%%%%%%%%%%%%%%%%%%%%%%%%%%%%%%%%%%%%%%%

%%% Shouldn't need to change these
\newcommand{\hmwkClass}{COS 255}
\newcommand{\hmwkSemester}{Spring 2016}

%%% Your name, in standard First Last format
\newcommand{\hmwkAuthorName}{Lukas Leung}
%%% Your NetID
\newcommand{\hmwkAuthorID}{lleung}

%%% The problem set number (just the number)
\newcommand{\hmwkAssignmentNum}{4}

%%% The problem number (just the number)
\newcommand{\hmwkProblemNum}{3}

%%% A list of your collaborators' NetIDs, separated by ", ".
%%% You can use a new line ("\\") in the middle to prevent a long
%%% list from overflowing.
\newcommand{\hmwkCollaborators}{}
%%% Sets the collaborator list to appear on the first page
\thispagestyle{fancycollab}

%%%%%%% begin Solution %%%%%%%%%%%%%%%%%%%%%%%%%%%%%%%%%%%%%%%%%%%%


%%%%%%% start Bank Cards %%%%%%%%%%%%%%%%%%%%%%%%%%%%%%%%%%%%%%%%%%

\section{Book 5.3: Bank Cards}
\textbf{Background:} \\
~\indent Suppose you're consulting for a bank that's concerned about
fraud detection, and they come to you with the following problem.
They have a collenction of $n$ bank cards that they've confiscated,
suspecting them of being used in fraud. Each bank card is a small
plastic object, containing a magnetic stripe with some encrypted
data, and it corresponds to a unique account in the bank. Each account
can have many bank cards corresponding to it, and we'll say that
two bank cards are $equivalent$ if they correspond to the same
account. It's very difficult to read the account number off a bank
card directly, but the bank has a high-tech "equivalence tester"
that takes two bank cards and, after performing some computations,
determines whether they are equivalent. \\
\textbf{Question:} \\
~\indent Among the collection of $n$ cards, is there a set of more
than $n/2$ of them that are all equivalent to one another? Assume
that the only feasible operations you can do with the cards are to
pick two of them and plug them in to the equivalence tester. Show
how to decide the answer to their question with only $(n\cdot log(n))$
invocations of the equivalence tester.
%%%%%%% end Problem %%%%%%%%%%%%%%%%%%%%%%%%%%%%%%%%%%%%%%%%%%%%%%%

\newpage

%%%%%%% Mathematical Formulation %%%%%%%%%%%%%%%%%%%%%%%%%%%%%%%%%%
\subsection{Mathematical Formulation}
Given an input of $n$ bank cards, determine if $\exists$ a subset $M$
where all cards $c_i \in M$ are identical and $n/2\ \textless\ |M|$.
s.t. $c_i = c_j$.

%%%%%%% Algorithm %%%%%%%%%%%%%%%%%%%%%%%%%%%%%%%%%%%%%%%%%%%%%%%%%
\subsection{Solution}
Important Confusing Data Structures:
\begin{itemize}
    \item BankCard[n] \textbf{bankCards} : ~ Store each Bank Card in
    each slot. We note that we are not able to sort this or do anything
    but access and swap positions i.e. can only access.
\end{itemize}

We will begin by making the observation that if $\exists$ a subset $M$
where all cards $c_i \in M$ are identical and $n/2\ \textless\ m\ =\ |M|$,
then in one of the halves $bankCards[0..(n/2)]$ or $bankCards[(n/2+1)..n]$
$\exists$ at least $m/2$ identical cards. This algorithm will implement
this recursively in order to determine whether or not the afformentioned
statement holds. $Note: for\ the\ sake\ of\ this\ algorithm\ I\ will\ use\ BankCard[\ ]\
to\ pass\ through$ $but\ the\ actual\ algorithm\ would\ use\ a\ global\
array\ and\ pass\ the\ corresponding\ indices$

\begin{algorithm}[H]
\caption{BankCard}
\begin{algorithmic}
    \Procedure{Equivalence}{BankCard[] bankCards}
        \State $n \gets$ \Call{sizeOf}{bankCards}
        \State $m \gets$ n/2
        \If{n == 1}
            \State return bankCard[0]
        \Else{\ n == 2}
            \If{bankCards[0].\Call{equals}{bankCards[1]}}
                \State return bankCard[0]
            \EndIf
        \EndIf
        \State $bankCards1, bankCards2 \gets$ bankCards[0..m], bankCards[(m+1)..n]
        \State $card1, card2 \gets$ \Call{Equivalence}{bankCards1}, \Call{Equivalence}{bankCards2}
        \If{card1 is a card}
            \State test card1 against all other cards in bankCards and numberEqual++ when equal
            \If{numberEqual == GOAL\_SIZE}
                \State return card1
            \EndIf
        \EndIf
        \If{card2 is a card}
            \State test card2 against all other cards in bankCards and numberEqual++ when equal
            \If{numberEqual == GOAL\_SIZE}
                \State return card2
            \EndIf
        \EndIf
        \State return bankCards
    \EndProcedure
\end{algorithmic}
\end{algorithm}

If at the end of this method no card has been returned $\implies$
there is no matching of size n/2



%%%%%%% Correctness %%%%%%%%%%%%%%%%%%%%%%%%%%%%%%%%%%%%%%%%%%%%%%%

\subsection{Correctness}
%%%%%%% PROPOSITION 1 %%%%%%%%%%%%%%%
\begin{proposition}
~ \\ \indent Propose that the algorithm will determine if $\exists$
a subset $M$ where all cards $c_i \in M$ are identical and
$n/2\ \textless\ m\ =\ |M|$.

\end{proposition}

\begin{proof}
~ \\ \indent This holds because if more than $n/2$ cards are equivalent,
one of the halves $bankCards[0..(n/2)]$ or $bankCards[(n/2+1)..n]$
$\exists$ more than $m/2$ identical cards. $\therefore$ one of the two
recursive calls must return a card equivalent to the whole set’s
majority equivalence $\therefore$ this algorithm compares all returned
cards to the whole set, so then the majority equivalence will be found.
\end{proof}


%%%%%%% Analysis %%%%%%%%%%%%%%%%%%%%%%%%%%%%%%%%%%%%%%%%%%%%%%%%%%
\subsection{Analysis}
For the following analysis, we will say that..

%%%%%%% PROPOSITION 1 %%%%%%%%%%%%%%%
\begin{proposition}
\label{numq}
The \underline{space complexity} of this algorithm is \textbf{O(N)}
\end{proposition}

\begin{proof}
~ \\ \indent This is due to the fact that all of our data is stored
in the array containing all the cards. Everything else is pointers
which are being initialized recursively so at most $3\cdot log(N)$
of them will be active at a time.
\begin{center}
    Giving us a space complexity of \textbf{O(N)}
\end{center}
\end{proof}

%%%%%%% PROPOSITION 2 %%%%%%%%%%%%%%%
\begin{proposition}
\label{numq}
The \underline{time complexity} of this algorithm is \textbf{O(N$\cdot$log(N))}
\end{proposition}

\begin{proof}
This is the case because our algorithm calls the $Equivalence$ method,
where N cards = T(N), as two recursive calls (each of size n/2) and
at most 2n tests for the two returned cards at each level. So T(n)
$\cong$ 2T(n/2) + 2n = O(n log n) from class.
\begin{center}
    Giving us a time complexity of \textbf{O(N$\cdot$log(N))}
\end{center}
\end{proof}

%%%%%%% end Bank Cards %%%%%%%%%%%%%%%%%%%%%%%%%%%%%%%%%%%%%%%%%%%


%%%%%%% end Solution %%%%%%%%%%%%%%%%%%%%%%%%%%%%%%%%%%%%%%%%%%%%%%

\end{document}