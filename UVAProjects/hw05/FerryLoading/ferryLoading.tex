% Use this template to write your solutions to COS 423 problem sets

\documentclass[12pt]{article}
\usepackage[utf8]{inputenc}
\usepackage{amsmath, amsfonts, amsthm, amssymb, algorithm, graphicx, mathtools, xfrac}
\usepackage[noend]{algpseudocode}
\usepackage{fancyhdr, lastpage}
\usepackage{booktabs}
\usepackage{multirow}
\usepackage{graphicx}
\usepackage{pgfplots}
\usepackage[vmargin=1.20in,hmargin=1.25in,centering,letterpaper]{geometry}
\setlength{\headsep}{.50in}
\setlength{\headheight}{15pt}

% Landau notation
\DeclareMathOperator{\BigOm}{\mathcal{O}}
\newcommand{\BigOh}[1]{\BigOm\left({#1}\right)}
\DeclareMathOperator{\BigTm}{\Theta}
\newcommand{\BigTheta}[1]{\BigTm\left({#1}\right)}
\DeclareMathOperator{\BigWm}{\Omega}
\newcommand{\BigOmega}[1]{\BigWm\left({#1}\right)}
\DeclareMathOperator{\LittleOm}{\mathrm{o}}
\newcommand{\LittleOh}[1]{\LittleOm\left({#1}\right)}
\DeclareMathOperator{\LittleWm}{\omega}
\newcommand{\LittleOmega}[1]{\LittleWm\left({#1}\right)}

% argmin and argmax
\newcommand{\argmin}{\operatornamewithlimits{argmin}}
\newcommand{\argmax}{\operatornamewithlimits{argmax}}

\newcommand{\calP}{\mathcal{P}}
\newcommand{\Z}{\mathbb{Z}}
\newcommand{\R}{\mathbb{R}}
\newcommand{\Exp}{\mathbb{E}}
\newcommand{\Q}{\mathbb{Q}}
\newcommand{\sign}{\mathrm{sign\ }}
\newcommand{\abs}{\mathrm{abs\ }}
\newcommand{\eps}{\varepsilon}
\newcommand{\zo}{\{0, 1\}}
\newcommand{\SAT}{\mathit{SAT}}
\renewcommand{\P}{\mathbf{P}}
\newcommand{\NP}{\mathbf{NP}}
\newcommand{\coNP}{\co{NP}}
\newcommand{\co}[1]{\mathbf{co#1}}
\renewcommand{\Pr}{\mathop{\mathrm{Pr}}}

% theorems, lemmas, invariants, etc.
\newtheorem{theorem}{Theorem}
\newtheorem{lemma}[theorem]{Lemma}
\newtheorem{invariant}[theorem]{Invariant}
\newtheorem{corollary}[theorem]{Corollary}
\newtheorem{definition}[theorem]{Definition}
\newtheorem{property}[theorem]{Property}
\newtheorem{proposition}[theorem]{Proposition}

% piecewise functions
\newenvironment{piecewise}{\left \{\begin{array}{l@{,\ }l}}
{\end{array}\right.}

% paired delimiters
\DeclarePairedDelimiter{\ceil}{\lceil}{\rceil}
\DeclarePairedDelimiter{\floor}{\lfloor}{\rfloor}
\DeclarePairedDelimiter{\len}{|}{|}
\DeclarePairedDelimiter{\set}{\{}{\}}

\makeatletter
\@addtoreset{equation}{section}
\makeatother
\renewcommand{\theequation}{\arabic{section}.\arabic{equation}}

% algorithms
\algnewcommand\algorithmicinput{\textbf{INPUT:}}
\algnewcommand\INPUT{\item[\algorithmicinput]}
\algnewcommand\algorithmicoutput{\textbf{OUTPUT:}}
\algnewcommand\OUTPUT{\item[\algorithmicoutput]}


% Formating Macros

\pagestyle{fancy}
\lhead{\sc \hmwkClass\ $\; \;\cdot \; \;$ \hmwkSemester\ $\; \;\cdot \; \;$
Problem \hmwkAssignmentNum.\hmwkProblemNum}
%\chead{\sc Problem \hmwkAssignmentNum.\hmwkProblemNum}
%\chead{}
\rhead{\em \hmwkAuthorName\ $($\hmwkAuthorID$)$\/}
\cfoot{}
\lfoot{}
\rfoot{\sc Page\ \thepage\ of\ \protect\pageref{LastPage}}
\renewcommand\headrulewidth{0.4pt}
\renewcommand\footrulewidth{0.4pt}

\fancypagestyle{fancycollab}
{
    \lfoot{\textit{Collaborators: \hmwkCollaborators}}
}

\fancypagestyle{problemstatement}
{
    \rhead{}
    \lfoot{}
}

%%%%%% Begin document with header and title %%%%%%%%%%%%%%%%%%%%%%%%%

\begin{document}

%%%%%% Header Information %%%%%%%%%%%%%%%%%%%%%%%%%%%%%%%%%%%%%%%%%%%

%%% Shouldn't need to change these
\newcommand{\hmwkClass}{COS 255}
\newcommand{\hmwkSemester}{Spring 2016}

%%% Your name, in standard First Last format
\newcommand{\hmwkAuthorName}{Lukas Leung}
%%% Your NetID
\newcommand{\hmwkAuthorID}{lleung}

%%% The problem set number (just the number)
\newcommand{\hmwkAssignmentNum}{4}

%%% The problem number (just the number)
\newcommand{\hmwkProblemNum}{0}

%%% A list of your collaborators' NetIDs, separated by ", ".
%%% You can use a new line ("\\") in the middle to prevent a long
%%% list from overflowing.
\newcommand{\hmwkCollaborators}{}
%%% Sets the collaborator list to appear on the first page
\thispagestyle{fancycollab}

%%%%%%% begin Solution %%%%%%%%%%%%%%%%%%%%%%%%%%%%%%%%%%%%%%%%%%%%

%%%%%%% start Bank Cards %%%%%%%%%%%%%%%%%%%%%%%%%%%%%%%%%%%%%%%%%%

\section{UVA Problem 10561: Ferry Loading}
\textbf{Background} \\
~\indent Before bridges were common, ferries were used to transport cars across rivers. River ferries, unlike their
larger cousins, run on a guide line and are powered by the river's current. Two lanes of cars drive onto
the ferry from one end, the ferry crosses the river, and the cars exit from the other end of the ferry. \\
\indent The cars waiting to board the ferry form a single queue, and the operator directs each car in turn
to drive onto the port (left) or starboard (right) lane of the ferry so as to balance the load. Each car in
the queue has a different length, which the operator estimates by inspecting the queue. Based on this
inspection, the operator decides which side of the ferry each car should board, and boards as many cars
as possible from the queue, subject to the length limit of the ferry. Your job is to write a program that
will tell the operator which car to load on which side so as to maximize the number of cars loaded. \\
\textbf{Input} \\
~\indent The input begins with a single positive integer on a line by itself indicating the number of the cases
following, each of them as described below. This line is followed by a blank line, and there is also a
blank line between two consecutive inputs. \\
\indent The first line of input contains a single integer between 1 and 100: the length of the ferry (in metres).
For each car in the queue there is an additional line of input specifying the length of the car (in cm, an
integer between 100 and 3000 inclusive). A  nal line of input contains the integer 0. The cars must be
loaded in order, subject to the constraint that the total length of cars on either side does not exceed
the length of the ferry. Subject to this constraint as many cars should be loaded as possible, starting
with the  rst car in the queue and loading cars in order until no more can be loaded. \\
\textbf{Output} \\
~\indent For each test case, the output must follow the description below. The outputs of two consecutive cases
will be separated by a blank line. \\
\indent The first line of outuput should give the number of cars that can be loaded onto the ferry. For each
car that can be loaded onto the ferry, in the order the cars appear in the input, output a line containing
"port" if the car is to be directed to the port side and "starboard" if the car is to be directed to the
starboard side. If several arrangements of the cars meet the criteria above, any one will do.



%%%%%%% end Problem %%%%%%%%%%%%%%%%%%%%%%%%%%%%%%%%%%%%%%%%%%%%%%%

\newpage

%%%%%%% Mathematical Formulation %%%%%%%%%%%%%%%%%%%%%%%%%%%%%%%%%%
\subsection{Mathematical Formulation}
We are given an input of $L$ for length of ferry, followed by an input of N cars in the
form $l_1, l_2, ... , l_N$ of which $n, 1 \leq n \leq N$ will fit onto the ferry if optimally placed.
The algorithm will decide what the number $n$ is as well as whether each of these $n$ cars
should go to port or starboard in the order seen.
%%%%%%% Algorithm %%%%%%%%%%%%%%%%%%%%%%%%%%%%%%%%%%%%%%%%%%%%%%%%%

\subsection{Solution}
Important Confusing Data Structures:
\begin{itemize}
    \item boolean[2][LENGT+1] \textbf{lastUsed} : ~ keeps track of the last seen possible car lengths
    \item boolean[202][LENGT+1] \textbf{solutions} : ~ yes
    \item int[202] \textbf{carLen} : ~ Keeps track of the car lengths that have been seen thus far. carLength[1] = $l_1$
\end{itemize}

We treat this problem as the classic napsack dynamic programming problem except,
instead of having one napsack we have two. Therefore the relationship we will
be doing is for each open state and given a new value $v_i$:
\[
state(i, v_1, v_2) =
    \begin{cases}
        state(i+1, v_1 + v_i, v_2),\ if\ v_1\ +\ v_i\ \leq\ L \\
        state(i+1, v_1, v_2 + v_i),\ if\ v_2\ +\ v_i\ \leq\ L
    \end{cases}
\]
If we use this exsact model we will have to use data structures for storage in the $3^{rd}$
degree, so as a trick to stay on a 2-Dimentional grid we use a trick. Since we know
that $|v_1| + |v_2| + |v_i| = \ell_i$ when item $v_i$ has been added where $\ell_i = \sum\limits_{i=1}^{k} l_i$ and $k \leq n$. This
fact is useful because then we just have to keep track of two things: $\ell_i$ and either
$|v_1|\ or\ |v_2|$. For this specific problem, $v_1 = port$ and $v_2 = starbord$ we can have the
following:
\[
state(i, v_2) =
    \begin{cases}
        state(i+1, v_2 + v_i),\ if\ |v_2|\ +\ |v_i|\ \leq\ L \implies starboard\\
        state(i+1, v_2),\ if\ \ell_i\ -\ |v_2|\ \leq\ L \implies port
    \end{cases}
\]
This way we only need to know $v_2$ length of starboard side, $v_i$ length of current car,
and $\ell$ length of total cars seen thusfar. This is only to build the table which we will
in turn backtrack to determine for each car wheather to go to starboard or port.

\begin{algorithm}[H]
\caption{Main and Build}
\begin{algorithmic}
    \Procedure{build}{}
        \For{numberOfCases}
            \State $LENGTH \gets$ Integer.parseInt(line) * 100
            \State $lastUsed, solutions, carLen \gets$ initialized (see above)
            \State $boolean done \gets$ true
            \State $curRow, invRow, curCar, n, sumLen, lastLen \gets$ initialized to 0
            \While{true}
                \State $currentLen \gets$ Integer.parseInt(in.nextLine)
                \If{currentLen == 0}
                    \State break
                \EndIf
                \If{done}
                    \State continue
                \EndIf
                \State $invRow \gets$ curRow
                \State $curRow \gets$ (curRow + 1) \% 2
                \State curCar++
                \State $carLen[curCar] \gets$ currentLen
                \State sumLen += currentLen
                \State Arrays.fill(lastUsed[curRow], false)
                \State $boolean\ canLoad \gets$ false;
                \For{$len \in 0..LENGTH$}
                    \If{!lastUsed[invRow][len]}
                        \State continue
                    \EndIf
                    \State $int pos \gets$ len + currentLen
                    \If{$pos \leq LENGTH$} // Starboard
                        \State lastUsed[curRow][pos] = true
                        \State lastLen = pos
                        \State canLoad = true
                    \EndIf
                    \If{$sumLen - len \leq LENGTH$} // Port
                        \State lastUsed[curRow][len] = true
                        \State solutions[curCar][len] = true
                        \State lastLen = len
                        \State canLoad = true
                    \EndIf
                \EndFor
                \If{!canLoad}
                    \State done = true
                \Else
                    \State n++
                \EndIf
            \EndWhile
            \State \Call{backTrack}{n, lastLen, carLen, solutions}
        \EndFor
    \EndProcedure
\end{algorithmic}
\end{algorithm}
\newpage
The next and final step then is to bakctrack through the built solutions 2-D array and
determine (from the last element) whether each car should go to starboard or port
side. Now from the build phase we note that we have only been marking down
true on the solutions array if the specified car can go to port. Another thing that we
keep in mind is that we break out of the search once we have found the max number
of cars we can place on the ferry, we will simply traceback through the algorithm
starting with our last seen option.
\begin{algorithm}[H]
\caption{Backtrack and Print}
\begin{algorithmic}
    \Procedure{backTrack}{int n, int lastLen, int[ ] carLen, boolean[ ][ ] solutions}
        \State $backtrack \gets$ new boolean[n+1]
        \For{i = n..1}
            \If{!solutions[i][lastLen]}
                \State lastLen -= carLen[i]
                \State backtrack[i] = false
            \Else
                \State backtrack[i] = true
            \EndIf
        \EndFor
        \State \Call{print}{n}
        \For{i = 1..n}
            \If{!backtrack[i]}
                \State \Call{print}{starboard}
            \Else
                \State \Call{print}{port}
            \EndIf
        \EndFor
    \EndProcedure
\end{algorithmic}
\end{algorithm}
%%%%%%% Correctness %%%%%%%%%%%%%%%%%%%%%%%%%%%%%%%%%%%%%%%%%%%%%%%

\subsection{Correctness}
%%%%%%% PROPOSITION 1 %%%%%%%%%%%%%%%
\begin{proposition}
~ \\ \indent propose
\end{proposition}

\begin{proof}
~ \\ \indent Using the fact that 
\end{proof}


%%%%%%% Analysis %%%%%%%%%%%%%%%%%%%%%%%%%%%%%%%%%%%%%%%%%%%%%%%%%%
\subsection{Analysis}
For the following analysis, we will say that..

%%%%%%% PROPOSITION 1 %%%%%%%%%%%%%%%
\begin{proposition}
\label{numq}
The \underline{space complexity} of this algorithm is \textbf{O(..)}
\end{proposition}

\begin{proof}
~ \\ \indent This is due to the fact that all of our data is stored in data structures:
\begin{itemize}
    \item \underline{cause}: reason $\implies complexity$
\end{itemize}
\begin{center}
    Giving us a space complexity of \textbf{O(..)}
\end{center}
\end{proof}

%%%%%%% PROPOSITION 2 %%%%%%%%%%%%%%%
\begin{proposition}
\label{numq}
The \underline{time complexity} of this algorithm is \textbf{O(..)}
\end{proposition}

\begin{proof}
This is the case because our algorithm...
\begin{center}
    Giving us a time complexity of \textbf{O(..)}
\end{center}
\end{proof}

%%%%%%% Example %%%%%%%%%%%%%%%%%%%%%%%%%%%%%%%%%%%%%%%%%%%%%%%%%%%

\subsection{An Example}

%%%%%%% end Bank Cards %%%%%%%%%%%%%%%%%%%%%%%%%%%%%%%%%%%%%%%%%%%


%%%%%%% end Solution %%%%%%%%%%%%%%%%%%%%%%%%%%%%%%%%%%%%%%%%%%%%%%

\end{document}