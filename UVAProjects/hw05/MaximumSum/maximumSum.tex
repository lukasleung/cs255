% Use this template to write your solutions to COS 423 problem sets

\documentclass[12pt]{article}
\usepackage[utf8]{inputenc}
\usepackage{amsmath, amsfonts, amsthm, amssymb, algorithm, graphicx, mathtools, xfrac}
\usepackage[noend]{algpseudocode}
\usepackage{fancyhdr, lastpage}
\usepackage{booktabs}
\usepackage{multirow}
\usepackage{graphicx}
\usepackage{pgfplots}
\usepackage[vmargin=1.20in,hmargin=1.25in,centering,letterpaper]{geometry}
\setlength{\headsep}{.50in}
\setlength{\headheight}{15pt}

% Landau notation
\DeclareMathOperator{\BigOm}{\mathcal{O}}
\newcommand{\BigOh}[1]{\BigOm\left({#1}\right)}
\DeclareMathOperator{\BigTm}{\Theta}
\newcommand{\BigTheta}[1]{\BigTm\left({#1}\right)}
\DeclareMathOperator{\BigWm}{\Omega}
\newcommand{\BigOmega}[1]{\BigWm\left({#1}\right)}
\DeclareMathOperator{\LittleOm}{\mathrm{o}}
\newcommand{\LittleOh}[1]{\LittleOm\left({#1}\right)}
\DeclareMathOperator{\LittleWm}{\omega}
\newcommand{\LittleOmega}[1]{\LittleWm\left({#1}\right)}

% argmin and argmax
\newcommand{\argmin}{\operatornamewithlimits{argmin}}
\newcommand{\argmax}{\operatornamewithlimits{argmax}}

\newcommand{\calP}{\mathcal{P}}
\newcommand{\Z}{\mathbb{Z}}
\newcommand{\R}{\mathbb{R}}
\newcommand{\Exp}{\mathbb{E}}
\newcommand{\Q}{\mathbb{Q}}
\newcommand{\sign}{\mathrm{sign\ }}
\newcommand{\abs}{\mathrm{abs\ }}
\newcommand{\eps}{\varepsilon}
\newcommand{\zo}{\{0, 1\}}
\newcommand{\SAT}{\mathit{SAT}}
\renewcommand{\P}{\mathbf{P}}
\newcommand{\NP}{\mathbf{NP}}
\newcommand{\coNP}{\co{NP}}
\newcommand{\co}[1]{\mathbf{co#1}}
\renewcommand{\Pr}{\mathop{\mathrm{Pr}}}

% theorems, lemmas, invariants, etc.
\newtheorem{theorem}{Theorem}
\newtheorem{lemma}[theorem]{Lemma}
\newtheorem{invariant}[theorem]{Invariant}
\newtheorem{corollary}[theorem]{Corollary}
\newtheorem{definition}[theorem]{Definition}
\newtheorem{property}[theorem]{Property}
\newtheorem{proposition}[theorem]{Proposition}

% piecewise functions
\newenvironment{piecewise}{\left \{\begin{array}{l@{,\ }l}}
{\end{array}\right.}

% paired delimiters
\DeclarePairedDelimiter{\ceil}{\lceil}{\rceil}
\DeclarePairedDelimiter{\floor}{\lfloor}{\rfloor}
\DeclarePairedDelimiter{\len}{|}{|}
\DeclarePairedDelimiter{\set}{\{}{\}}

\makeatletter
\@addtoreset{equation}{section}
\makeatother
\renewcommand{\theequation}{\arabic{section}.\arabic{equation}}

% algorithms
\algnewcommand\algorithmicinput{\textbf{INPUT:}}
\algnewcommand\INPUT{\item[\algorithmicinput]}
\algnewcommand\algorithmicoutput{\textbf{OUTPUT:}}
\algnewcommand\OUTPUT{\item[\algorithmicoutput]}


% Formating Macros

\pagestyle{fancy}
\lhead{\sc \hmwkClass\ $\; \;\cdot \; \;$ \hmwkSemester\ $\; \;\cdot \; \;$
Problem \hmwkAssignmentNum.\hmwkProblemNum}
%\chead{\sc Problem \hmwkAssignmentNum.\hmwkProblemNum}
%\chead{}
\rhead{\em \hmwkAuthorName\ $($\hmwkAuthorID$)$\/}
\cfoot{}
\lfoot{}
\rfoot{\sc Page\ \thepage\ of\ \protect\pageref{LastPage}}
\renewcommand\headrulewidth{0.4pt}
\renewcommand\footrulewidth{0.4pt}

\fancypagestyle{fancycollab}
{
    \lfoot{\textit{Collaborators: \hmwkCollaborators}}
}

\fancypagestyle{problemstatement}
{
    \rhead{}
    \lfoot{}
}

%%%%%% Begin document with header and title %%%%%%%%%%%%%%%%%%%%%%%%%

\begin{document}

%%%%%% Header Information %%%%%%%%%%%%%%%%%%%%%%%%%%%%%%%%%%%%%%%%%%%

%%% Shouldn't need to change these
\newcommand{\hmwkClass}{COS 255}
\newcommand{\hmwkSemester}{Spring 2016}

%%% Your name, in standard First Last format
\newcommand{\hmwkAuthorName}{Lukas Leung}
%%% Your NetID
\newcommand{\hmwkAuthorID}{lleung}

%%% The problem set number (just the number)
\newcommand{\hmwkAssignmentNum}{5}

%%% The problem number (just the number)
\newcommand{\hmwkProblemNum}{2}

%%% A list of your collaborators' NetIDs, separated by ", ".
%%% You can use a new line ("\\") in the middle to prevent a long
%%% list from overflowing.
\newcommand{\hmwkCollaborators}{}
%%% Sets the collaborator list to appear on the first page
\thispagestyle{fancycollab}

%%%%%%% begin Solution %%%%%%%%%%%%%%%%%%%%%%%%%%%%%%%%%%%%%%%%%%%%

%%%%%%% start Maximum Sum %%%%%%%%%%%%%%%%%%%%%%%%%%%%%%%%%%%%%%%%%

\section{UVA Problem 108: Maximum Sum}
\textbf{Background} \\
~\indent A problem that is simple to solve in one dimension is often much more difficult
to solve in more than one dimension. Consider satisfying a boolean expression in
conjunctive normal form in which each conjunct consists of exactly 3 disjuncts. This
problem (3-SAT) is NP-complete. The problem 2-SAT is solved quite efficiently. In
contrast, some problems belong to the same complexity class regardless of the
dimensionality of the problem. \\
\\
\textbf{The Problem} \\
\indent Given a 2-dimensional array of positive and negative integers, and the sub-rectangle
with the largest sum. The sum of a rectangle is the sum of all the elements in that
rectangle. In this problem the sub-rectangle with the largest sum is referred to as the
maximal sub-rectangle . A sub-rectangle is any contiguous sub-array of size 1x1 or
greater located within the whole array. \\
\\
\textbf{Input and Output} \\
\indent The \underline{input} consists of an $NxN$ array of integers. The input begins with a single
positive integer N on a line by itself indicating the size of the square two dimensional
array. This is followed by $N^2$ integers separated by white-space (newlines and spaces).
These $N^2$ integers make up the array in row-major order (i.e., all numbers on the
first row, left-to-right, then all numbers on the second row, left-to-right, etc.). N may
be as large as 100. The numbers in the array will be in the range [-127, 127]. The
\underline{output} is the sum of the maximal sub-rectangle.
%%%%%%% end Problem %%%%%%%%%%%%%%%%%%%%%%%%%%%%%%%%%%%%%%%%%%%%%%%

\newpage

%%%%%%% Mathematical Formulation %%%%%%%%%%%%%%%%%%%%%%%%%%%%%%%%%%
\subsection{Mathematical Formulation}
Given an input of $N^2$ numbers in the range of [-127, 127] in the form of an $N$x$N$
matrix, lets call this matrix E. Given E, we will deterermine the maximal sub-rectangle.
We define a sub-rectangle as being synonymous to a sub-matrix (E$^{\ast}$) of E. i.e
\[ E =
\begin{bmatrix}
    e_{1,1} & e_{1,2} & \dots  & e_{1,N} \\
    e_{2,1} & e_{2,2} & \dots  & e_{2,N} \\
    \vdots  & \vdots  & \ddots & \vdots  \\
    e_{N,1} & e_{N,2} & \dots  & e_{N,N}
\end{bmatrix}
, E^{\ast} =
\begin{bmatrix}
    e_{i,j}     & \dots  & e_{i,n}     \\
    \vdots      & \ddots & \vdots      \\
    e_{m,j}     & \dots  & e_{m,n}
\end{bmatrix}
\]
Where $1 \leq i \leq m \leq N$ and $1 \leq i \leq m \leq N$, where each one's value is defined as
$\sum\limits_{x=i}^{m} \sum\limits_{y=j}^{n} e_{x,y} = S$. We will determine the maximal of these S.

%%%%%%% Algorithm %%%%%%%%%%%%%%%%%%%%%%%%%%%%%%%%%%%%%%%%%%%%%%%%%

\subsection{Solution}
Important Confusing Data Structures:
\begin{itemize}
    \item int[N][N] \textbf{map} : ~ Used to quick compute sub-rectangles, see below for details
    \item int \textbf{colSum} : ~ Stores the value the elements being examined currently above
        and including the target element $e_{i,j}$
    \item int \textbf{sum} : ~ Intermediate value stores the sum of possible preceeding colomns
        on the same row which have potential to combine with the current row.
    \item int \textbf{max} : ~ Stores the value of the maximal sub-rectangle seen thusfar.
\end{itemize}

Given E as described above from input, we begin our algoithm by creating a quick
sum matrix (map) for the input. We do this by, for each element $e_{i,j}$ we sum it with
all of the elements directly above it in the same column. i.e. we let:
\[ map =
\begin{bmatrix}
    s_{1,1} & s_{1,2} & \dots  & s_{1,N} \\
    s_{2,1} & s_{2,2} & \dots  & s_{2,N} \\
    \vdots  & \vdots  & \ddots & \vdots  \\
    s_{N,1} & s_{N,2} & \dots  & s_{N,N}
\end{bmatrix}
\]
Where every element: $s_{i,j} = \sum\limits_{k=1}^{i} e_{k,j}$ for $1 \leq i,j \leq N$

\begin{algorithm}[H]
\caption{Build sum}
\begin{algorithmic}
    \Procedure{Main}{Scanner in}
        \State $N \gets$ in.nextInt()
        \State $map \gets$ initialize
        \For{$row \in 1..N$}
            \For{$col \in 1..N$}
                \State map[row][col] = in.nextInt() + map[row-1][col] // fill in map
            \EndFor
        \EndFor
        \State \Call{print}{maxSum(map, N)} // see below
    \EndProcedure
\end{algorithmic}
\end{algorithm}

We now will implement the main portion of our algorithm which will in turn
calculate the value associated with the maximal sub-rectangle. It does this by looking
at every potential height ($h$) of sub-rectangle, where $1 \leq h \leq N$. And considering
every valid bottom right element i.e.
\[ E^{\ast} =
    \begin{bmatrix}
        e_{i,j}     & \dots  & e_{i,n}     \\
        \vdots      & \ddots & \vdots      \\
        e_{m,j}     & \dots  & e_{m,n}
    \end{bmatrix}
    ,\ h = (m-i)+1\ and\ the\ bottom\ right\ corner\ is\ e_{m,n}
\]
We determine the width of the rectangle through dynamic programming. The rule
that we use is simple: if the width of the current sum of $E^{\ast}$ is greater than 0, we will
continue to extend our rectangle, otherwise this is the new left most column. At the
end of each decision we check to see if our current value associated with $E^{\ast}$ is greater
than our currently stored greatest value, if it is, we update and continue otherwise
we simply continue.

\begin{algorithm}[H]
\caption{Compute Maximal Sum}
\begin{algorithmic}
    \Procedure{maxSum}{int[][] map, int N}
        \State $max \gets$ -128
        \For{$h \in 1..N$}
            \State $sum \gets$ 0
            \For{$row \in h..N$}
                \For{$col \in 1..N$}
                    \State $colSum \gets$ map[row][col] - map[row-h][col]
                    \If{$sum \geq 0$}
                        \State sum += colSum
                    \Else
                        \State sum = colSum
                    \EndIf
                    \If{$sum\ \textgreater\ max$}
                        \State $max \gets$ sum
                    \EndIf
                \EndFor
            \EndFor
        \EndFor
        \State return max
    \EndProcedure
\end{algorithmic}
\end{algorithm}


%%%%%%% Correctness %%%%%%%%%%%%%%%%%%%%%%%%%%%%%%%%%%%%%%%%%%%%%%%

\subsection{Correctness}
%%%%%%% PROPOSITION 1 %%%%%%%%%%%%%%%
\begin{lemma}
~ \\ \indent Suppose we are given a sum matrix $map$ from the original matrix $E$ (both described
above), a height $h$, and any position $s_{i,j}$ on the map. Then we define this as:\\
\[
\sigma(i,j,h) = \sum\limits_{k=(i-h)+1}^{i} e_{k,j} \implies valueOf
\begin{bmatrix}
    e_{(i-h)+1,j}   \\
    e_{(i-h)+2,j}   \\
    \vdots          \\
    e_{i,j}
\end{bmatrix}
\]
\begin{center}(Note: the height of this is matrix is h)\end{center}
\end{lemma}

\begin{proposition}
~ \\ \indent Given the sum matrix $map$ of the original matrix $E$ and the $\sigma$ function from
Lemma 1 we can determine the maximal sub-rectangle of $E$ of arbitrary height.
\end{proposition}

\begin{proof}
~ \\ \indent To proove this we use induction by prooving that we can find the maximal sub-
rectangle of height $h, 1 \leq h \leq N$. We do this through a construction proof. In our
algorithm we have 3 variables to store pertinant information in: colSum, sum and
max. For our proof we will use the same three having them represent the same things
as described at the begining of the Solution section. We will use the following rules then
for updating sum and max:
\[  sum(i, j, h) = max
        \begin{cases}
            sum(i, j-1, h) + colSum(i, j, h) \\
            colSum(i, j, h)
        \end{cases}
    ,\ max(i, j) = max
        \begin{cases}
            max(i, j-1) \\
            sum(i, j, h)
        \end{cases}
\]
It can be noted here that i, j are the row, column coordinates and h is the height as
mentioned before. The value of sum and max however never get stored as we only
need to keep track of one instance at a time which is the reason for them able to be
represented with one variable each. Also sum(i, 1, h) = colSum(i, j, h) and max(i, 1)
= max(i-1, N). \\
For the first step of induction we will let \underline{h = 1}, we begin with $max = -128, sum =
0,\ and\ colSum(1,2,1) = \sigma(1,1,1) \implies$ sum(1,1,1) = max(1,1) = $\sigma(1,1,1)$ then we move
onto the next element and $colSum(1,2,1) = \sigma(1,2,1)$. Using the relationships:
\[  sum(1, 2, 1) = max
        \begin{cases}
            sum(1, 1, 1) + colSum(1, 2, 1) \\
            colSum(1, 2, 1)
        \end{cases}
    ,\ max(1, 2) = max
        \begin{cases}
            max(1, 1) \\
            sum(1, 2, 1)
        \end{cases}
\]
As we can see we would have updated our highest sum seen thusfar for this row and
the max seen thus-far. Therefore continuing on we will update these at each stage
giving us the proper sum seen thus far for the specified height in a specified row and
therefore it trivially follows we can keep track of the largest in total.
\end{proof}


%%%%%%% Analysis %%%%%%%%%%%%%%%%%%%%%%%%%%%%%%%%%%%%%%%%%%%%%%%%%%
\subsection{Analysis}
For the following analysis, we will say that..

%%%%%%% PROPOSITION 1 %%%%%%%%%%%%%%%
\begin{proposition}
\label{numq}
The \underline{space complexity} of this algorithm is \textbf{O(N$^2$)}
\end{proposition}

\begin{proof}
~ \\ \indent This is due to the fact that all of our data is stored in one NxN int array and 3
integer variables, which yeilds $N^2+3$
\begin{center}
    Giving us a space complexity of \textbf{O(N$^2$)}
\end{center}
\end{proof}

%%%%%%% PROPOSITION 2 %%%%%%%%%%%%%%%
\begin{proposition}
\label{numq}
The \underline{time complexity} of this algorithm is \textbf{O(N$^3$)}
\end{proposition}

\begin{proof}
This is the case because our algorithm will first fill out our map NxN array
($N^2$) and then runs through a triple for loop. This is, for each height $h \in 1..N$, we
run N,  (N - h) times. yeilding $[(N \cdot N) + ((N-1) \cdot N) + \dots + (1 \cdot N)] = N \cdot [\frac{N \cdot (N+1)}{2}]$

\begin{center}
    Simplifying to a time complexity of \textbf{O(N$^3$)}
\end{center}
\end{proof}

%%%%%%% Example %%%%%%%%%%%%%%%%%%%%%%%%%%%%%%%%%%%%%%%%%%%%%%%%%%%

\subsection{An Example}

Let the input be $N = 2$ and our given table be denoted as E:
\[ E =
\begin{bmatrix}
    1 & -2 \\
    3 & 3
\end{bmatrix}
\implies map =
\begin{bmatrix}
    1 & -2 \\
    4 & 1
\end{bmatrix}
\]
Now the algorithm will look at the segments with height 1 i.e.
\[
Step\ 1:
    \begin{bmatrix}
        \underline{\textbf{1}} & -2 \\
        4 & 1
    \end{bmatrix}
,\ Step\ 2:
    \begin{bmatrix}
        1 & \underline{\textbf{-2}} \\
        4 & 1
    \end{bmatrix}
,\ Step\ 3:
    \begin{bmatrix}
        1 & -2 \\
        \underline{\textbf{4}} & 1
    \end{bmatrix}
,\ Step\ 4:
    \begin{bmatrix}
        1 & -2 \\
        4 & \underline{\textbf{1}}
    \end{bmatrix}
\]
Next we look at the segments with height 2:
\[
Step\ 5:
    \begin{bmatrix}
        \textbf{1} & -2 \\
        \underline{\textbf{4}} & 1
    \end{bmatrix}
,\ Step\ 6:
    \begin{bmatrix}
        1 & \textbf{-2} \\
        4 & \underline{\textbf{1}}
    \end{bmatrix}
\]
Which we can see below how the values change; note when we go to a new row, we reset the value of sum to 0.
\begin{table}[H]
	\centering
	\resizebox{\textwidth}{!}{
	\begin{tabular}{c||c|c|c|c||c|c|}
		\toprule
		Variable & \multicolumn{4}{|c|}{Height = 1}                                                    & \multicolumn{2}{||c|}{Height = 2}     \\ \hline
		         & Step 1            & Step 2             & Step 3             & Step 4                & Step 5            & Step 6            \\
		\midrule
		colSum   & 1 $\implies$ 1    & -2 $\implies$ -2   & 4 - 1 $\implies$ 3 & 1 - (-2) $\implies$ 3 & 4 $\implies$ 4    & 1 $\implies$ 1    \\
        sum      & 0 $\rightarrow$ 1 & 1 $\rightarrow$ -1 & 0 $\rightarrow$ 3  & 3 $\rightarrow$ 6     & 0 $\rightarrow$ 4 & 4 $\rightarrow$ 5 \\
        max      & 1                 & 1                  & 3                  & 6                     & 6                 & 6                 \\
        \bottomrule
	\end{tabular}}
\end{table}
Therefore giving us a maximum size sub-rectangle of value = 6.

%%%%%%% end Bank Cards %%%%%%%%%%%%%%%%%%%%%%%%%%%%%%%%%%%%%%%%%%%


%%%%%%% end Solution %%%%%%%%%%%%%%%%%%%%%%%%%%%%%%%%%%%%%%%%%%%%%%

\end{document}