% Use this template to write your solutions to COS 423 problem sets

\documentclass[12pt]{article}
\usepackage[utf8]{inputenc}
\usepackage{amsmath, amsfonts, amsthm, amssymb, algorithm, graphicx, mathtools, xfrac}
\usepackage[noend]{algpseudocode}
\usepackage{fancyhdr, lastpage}
\usepackage{booktabs}
\usepackage{multirow}
\usepackage{graphicx}
\usepackage{pgfplots}
\usepackage[vmargin=1.20in,hmargin=1.25in,centering,letterpaper]{geometry}
\setlength{\headsep}{.50in}
\setlength{\headheight}{15pt}

% Landau notation
\DeclareMathOperator{\BigOm}{\mathcal{O}}
\newcommand{\BigOh}[1]{\BigOm\left({#1}\right)}
\DeclareMathOperator{\BigTm}{\Theta}
\newcommand{\BigTheta}[1]{\BigTm\left({#1}\right)}
\DeclareMathOperator{\BigWm}{\Omega}
\newcommand{\BigOmega}[1]{\BigWm\left({#1}\right)}
\DeclareMathOperator{\LittleOm}{\mathrm{o}}
\newcommand{\LittleOh}[1]{\LittleOm\left({#1}\right)}
\DeclareMathOperator{\LittleWm}{\omega}
\newcommand{\LittleOmega}[1]{\LittleWm\left({#1}\right)}

% argmin and argmax
\newcommand{\argmin}{\operatornamewithlimits{argmin}}
\newcommand{\argmax}{\operatornamewithlimits{argmax}}

\newcommand{\calP}{\mathcal{P}}
\newcommand{\Z}{\mathbb{Z}}
\newcommand{\R}{\mathbb{R}}
\newcommand{\Exp}{\mathbb{E}}
\newcommand{\Q}{\mathbb{Q}}
\newcommand{\sign}{\mathrm{sign\ }}
\newcommand{\abs}{\mathrm{abs\ }}
\newcommand{\eps}{\varepsilon}
\newcommand{\zo}{\{0, 1\}}
\newcommand{\SAT}{\mathit{SAT}}
\renewcommand{\P}{\mathbf{P}}
\newcommand{\NP}{\mathbf{NP}}
\newcommand{\coNP}{\co{NP}}
\newcommand{\co}[1]{\mathbf{co#1}}
\renewcommand{\Pr}{\mathop{\mathrm{Pr}}}

% theorems, lemmas, invariants, etc.
\newtheorem{theorem}{Theorem}
\newtheorem{lemma}[theorem]{Lemma}
\newtheorem{invariant}[theorem]{Invariant}
\newtheorem{corollary}[theorem]{Corollary}
\newtheorem{definition}[theorem]{Definition}
\newtheorem{property}[theorem]{Property}
\newtheorem{proposition}[theorem]{Proposition}

% piecewise functions
\newenvironment{piecewise}{\left \{\begin{array}{l@{,\ }l}}
{\end{array}\right.}

% paired delimiters
\DeclarePairedDelimiter{\ceil}{\lceil}{\rceil}
\DeclarePairedDelimiter{\floor}{\lfloor}{\rfloor}
\DeclarePairedDelimiter{\len}{|}{|}
\DeclarePairedDelimiter{\set}{\{}{\}}

\makeatletter
\@addtoreset{equation}{section}
\makeatother
\renewcommand{\theequation}{\arabic{section}.\arabic{equation}}

% algorithms
\algnewcommand\algorithmicinput{\textbf{INPUT:}}
\algnewcommand\INPUT{\item[\algorithmicinput]}
\algnewcommand\algorithmicoutput{\textbf{OUTPUT:}}
\algnewcommand\OUTPUT{\item[\algorithmicoutput]}


% Formating Macros

\pagestyle{fancy}
\lhead{\sc \hmwkClass\ $\; \;\cdot \; \;$ \hmwkSemester\ $\; \;\cdot \; \;$
Problem \hmwkAssignmentNum.\hmwkProblemNum}
%\chead{\sc Problem \hmwkAssignmentNum.\hmwkProblemNum}
%\chead{}
\rhead{\em \hmwkAuthorName\ $($\hmwkAuthorID$)$\/}
\cfoot{}
\lfoot{}
\rfoot{\sc Page\ \thepage\ of\ \protect\pageref{LastPage}}
\renewcommand\headrulewidth{0.4pt}
\renewcommand\footrulewidth{0.4pt}

\fancypagestyle{fancycollab}
{
    \lfoot{\textit{Collaborators: \hmwkCollaborators}}
}

\fancypagestyle{problemstatement}
{
    \rhead{}
    \lfoot{}
}

%%%%%% Begin document with header and title %%%%%%%%%%%%%%%%%%%%%%%%%

\begin{document}

%%%%%% Header Information %%%%%%%%%%%%%%%%%%%%%%%%%%%%%%%%%%%%%%%%%%%

%%% Shouldn't need to change these
\newcommand{\hmwkClass}{COS 255}
\newcommand{\hmwkSemester}{Spring 2016}

%%% Your name, in standard First Last format
\newcommand{\hmwkAuthorName}{Lukas Leung}
%%% Your NetID
\newcommand{\hmwkAuthorID}{lleung}

%%% The problem set number (just the number)
\newcommand{\hmwkAssignmentNum}{8}

%%% The problem number (just the number)
\newcommand{\hmwkProblemNum}{0}

%%% A list of your collaborators' NetIDs, separated by ", ".
%%% You can use a new line ("\\") in the middle to prevent a long
%%% list from overflowing.
\newcommand{\hmwkCollaborators}{lab08}
%%% Sets the collaborator list to appear on the first page
\thispagestyle{fancycollab}

%%%%%%% begin Solution %%%%%%%%%%%%%%%%%%%%%%%%%%%%%%%%%%%%%%%%%%%%
%%%%%%% start Bank Cards %%%%%%%%%%%%%%%%%%%%%%%%%%%%%%%%%%%%%%%%%%

\section{UVA Problem 10154: Weights and Measures}
\textbf{Background} \\
~\indent I know, up on top you are seeing great sights,
But down at the bottom, we, too, should have rights.
We turtles can't stand it. Our shells will all crack!
Besides, we need food. We are starving!" groaned Mack.
Mack, in an effort to avoid being cracked, has enlisted your advice as to the order in which turtles
should be dispatched to form Yertle's throne. Each of the  ve thousand, six hundred and seven turtles
ordered by Yertle has a different weight and strength. Your task is to build the largest stack of turtles
possible \\ \\
\textbf{Input} \\
~\indent Standard input consists of several lines, each containing a pair of integers separated by one or more
space characters, specifying the weight and strength of a turtle. The weight of the turtle is in grams.
The strength, also in grams, is the turtle's overall carrying capacity, including its own weight. That is,
a turtle weighing 300g with a strength of 1000g could carry 700g of turtles on its back. There are at
most 5,607 turtles. \\ \\
\textbf{Output} \\
~\indent Your output is a single integer indicating the maximum number of turtles that can be stacked without
exceeding the strength of any one.
%%%%%%% end Problem %%%%%%%%%%%%%%%%%%%%%%%%%%%%%%%%%%%%%%%%%%%%%%%

\newpage

%%%%%%% Mathematical Formulation %%%%%%%%%%%%%%%%%%%%%%%%%%%%%%%%%%
\subsection{Mathematical Formulation}
Given an input of $N$ turtles in the format $w_i, s_i$ for $1\leq i \leq N$ where
$w_i$ is the wieght and $s_i$ the strength of the $i^{th}$ trutle. Let then the
difference $d_i$ be $w_i - s_i$ for each turtle and W.L.O.G. say that the turtles
are sorted based off of their $d_i$ in increasing order. i.e. $d_1 \leq d_2 \leq ... \leq d_N$.
This algorithm will use the Look-Ahead greedy approach to determine the maximum number of turtles
that can be stacked on top of eachother.

%%%%%%% Algorithm %%%%%%%%%%%%%%%%%%%%%%%%%%%%%%%%%%%%%%%%%%%%%%%%%

\subsection{Solution}
Important Confusing Data Structures:
\begin{itemize}
    \item int[ ] \textbf{allWeights} : ~ All of the correspoding weights of the turtles
    \item int[ ] \textbf{allDifferences} : ~ All of the corresponding differences of the turtles
    \item int \textbf{space} : ~ The least amount of difference currently available
\end{itemize}

The main functionality of this algorithm will be to

\begin{algorithm}[H]
\caption{ Method}
\begin{algorithmic}
    \Procedure{main}{}
        \State $allWeights \gets$ from input
        \State $allDifferences \gets$ from input
        \State $int space \gets$ allDifferences[0]
        \State $int c \gets$ 1
        \For{$i \in 1,N$}
            \State $int w \gets$ allWeights[i] // current turtle weight
            \If{$w \leq space$}
                \If{$(space - w) \leq allDifferences[i]$}
                    \State space -= w
                \Else
                    \State $space \gets$ allDifferences[i]
                \EndIf
            \Else
                \State continue;
            \EndIf
        \EndFor
        \State print(c);
    \EndProcedure
\end{algorithmic}
\end{algorithm}


%%%%%%% Correctness %%%%%%%%%%%%%%%%%%%%%%%%%%%%%%%%%%%%%%%%%%%%%%%

\subsection{Correctness}
%%%%%%% PROPOSITION 1 %%%%%%%%%%%%%%%
\begin{proposition}
~ \\ \indent I know this is incorrect. Figured it out after writing everything
out.
\end{proposition}


\begin{proof}
\underline{Counter Example:} $t_1$ = (100, 300), $t_2$ = (250,350) $\implies$
$d_1$ = 200 and $d_2$ = 100. Going through this algorithm we will get this order
as the sorted order and then see that we cannot put $t_2$ on top of $t_1$ therefore
we have only 1, however we can do $t_2$ on top of $t_1$. There-in-lies the
contradiction.
\end{proof}


%%%%%%% Analysis %%%%%%%%%%%%%%%%%%%%%%%%%%%%%%%%%%%%%%%%%%%%%%%%%%
\subsection{Analysis}
%%%%%%% PROPOSITION 1 %%%%%%%%%%%%%%%
\begin{proposition}
\label{numq}
The \underline{space complexity} of this algorithm is \textbf{O(N)}
\end{proposition}

\begin{proof}
~ \\ \indent This is due to the fact that all of our data is stored in data structures:
\begin{itemize}
    \item int[ ] \textbf{allWeights} : ~ All of the correspoding weights of the turtles \textbf{O(N)}
    \item int[ ] \textbf{allDifferences} : ~ All of the corresponding differences of the turtles \textbf{O(N)}
    \item int \textbf{space} : ~ The least amount of difference currently available \textbf{O(1)}
\end{itemize}
\begin{center}
    Giving us a space complexity of \textbf{O(N)}
\end{center}
\end{proof}

%%%%%%% PROPOSITION 2 %%%%%%%%%%%%%%%
\begin{proposition}
\label{numq}
The \underline{time complexity} of this algorithm is \textbf{O(N$\cdot$log(N))}
\end{proposition}

\begin{proof}
This is the case because our algorithm will sort in $N\cdot log(N)$ time and then
it will do one linear search (N) to determine the number of valid.
\begin{center}
    Giving us a time complexity of \textbf{O(N$\cdot$log(N))}
\end{center}
\end{proof}

%%%%%%% end Weights & Measures %%%%%%%%%%%%%%%%%%%%%%%%%%%%%%%%%%%%%%

\newpage

\section{Book Problem 6.2, hackers jobs}
\subsection{Part A}
Counter Example: let (10, 1, 10, 10) $\in$ L and (5,50,100,1) $\in$ H
$\therefore$ we would pick 0 + 50 + 10 + 10 = 70 as our income where
the best would be 10 + 0 + 100 + 10 = 120.
\subsection{Part B}
Algorithm:

%\newpage
\section{Book Problem 6.7}
A quick look here and we will note that we should be solving for the subsection of what is the optimal days
to buy and sell within interval (1,i) where $i \leq N$.  Therefore we must calculate for the best day to buy
before selling on a particular day.  i.e. min(i) then using this we can create an optimal sell value list for
what value can you get the most money if you sell on this day. Then searching through the sell list we find that
of the greatest value and trace back to its best buy date. All of these are linear passes.

\section{hw05}
Completed Maximum sum, can discuss this.

%%%%%%% end Solution %%%%%%%%%%%%%%%%%%%%%%%%%%%%%%%%%%%%%%%%%%%%%%

\end{document}