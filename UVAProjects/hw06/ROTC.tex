% Use this template to write your solutions to COS 423 problem sets

\documentclass[12pt]{article}
\usepackage[utf8]{inputenc}
\usepackage{amsmath, amsfonts, amsthm, amssymb, algorithm, graphicx, mathtools, xfrac}
\usepackage[noend]{algpseudocode}
\usepackage{fancyhdr, lastpage}
\usepackage{booktabs}
\usepackage{multirow}
\usepackage{graphicx}
\usepackage{pgfplots}
\usepackage[vmargin=1.20in,hmargin=1.25in,centering,letterpaper]{geometry}
\setlength{\headsep}{.50in}
\setlength{\headheight}{15pt}

% Landau notation
\DeclareMathOperator{\BigOm}{\mathcal{O}}
\newcommand{\BigOh}[1]{\BigOm\left({#1}\right)}
\DeclareMathOperator{\BigTm}{\Theta}
\newcommand{\BigTheta}[1]{\BigTm\left({#1}\right)}
\DeclareMathOperator{\BigWm}{\Omega}
\newcommand{\BigOmega}[1]{\BigWm\left({#1}\right)}
\DeclareMathOperator{\LittleOm}{\mathrm{o}}
\newcommand{\LittleOh}[1]{\LittleOm\left({#1}\right)}
\DeclareMathOperator{\LittleWm}{\omega}
\newcommand{\LittleOmega}[1]{\LittleWm\left({#1}\right)}

% argmin and argmax
\newcommand{\argmin}{\operatornamewithlimits{argmin}}
\newcommand{\argmax}{\operatornamewithlimits{argmax}}

\newcommand{\calP}{\mathcal{P}}
\newcommand{\Z}{\mathbb{Z}}
\newcommand{\R}{\mathbb{R}}
\newcommand{\Exp}{\mathbb{E}}
\newcommand{\Q}{\mathbb{Q}}
\newcommand{\sign}{\mathrm{sign\ }}
\newcommand{\abs}{\mathrm{abs\ }}
\newcommand{\eps}{\varepsilon}
\newcommand{\zo}{\{0, 1\}}
\newcommand{\SAT}{\mathit{SAT}}
\renewcommand{\P}{\mathbf{P}}
\newcommand{\NP}{\mathbf{NP}}
\newcommand{\coNP}{\co{NP}}
\newcommand{\co}[1]{\mathbf{co#1}}
\renewcommand{\Pr}{\mathop{\mathrm{Pr}}}

% theorems, lemmas, invariants, etc.
\newtheorem{theorem}{Theorem}
\newtheorem{lemma}[theorem]{Lemma}
\newtheorem{invariant}[theorem]{Invariant}
\newtheorem{corollary}[theorem]{Corollary}
\newtheorem{definition}[theorem]{Definition}
\newtheorem{property}[theorem]{Property}
\newtheorem{proposition}[theorem]{Proposition}

% piecewise functions
\newenvironment{piecewise}{\left \{\begin{array}{l@{,\ }l}}
{\end{array}\right.}

% paired delimiters
\DeclarePairedDelimiter{\ceil}{\lceil}{\rceil}
\DeclarePairedDelimiter{\floor}{\lfloor}{\rfloor}
\DeclarePairedDelimiter{\len}{|}{|}
\DeclarePairedDelimiter{\set}{\{}{\}}

\makeatletter
\@addtoreset{equation}{section}
\makeatother
\renewcommand{\theequation}{\arabic{section}.\arabic{equation}}

% algorithms
\algnewcommand\algorithmicinput{\textbf{INPUT:}}
\algnewcommand\INPUT{\item[\algorithmicinput]}
\algnewcommand\algorithmicoutput{\textbf{OUTPUT:}}
\algnewcommand\OUTPUT{\item[\algorithmicoutput]}


% Formating Macros

\pagestyle{fancy}
\lhead{\sc \hmwkClass\ $\; \;\cdot \; \;$ \hmwkSemester\ $\; \;\cdot \; \;$
Problem \hmwkAssignmentNum.\hmwkProblemNum}
%\chead{\sc Problem \hmwkAssignmentNum.\hmwkProblemNum}
%\chead{}
\rhead{\em \hmwkAuthorName\ $($\hmwkAuthorID$)$\/}
\cfoot{}
\lfoot{}
\rfoot{\sc Page\ \thepage\ of\ \protect\pageref{LastPage}}
\renewcommand\headrulewidth{0.4pt}
\renewcommand\footrulewidth{0.4pt}

\fancypagestyle{fancycollab}
{
    \lfoot{\textit{Collaborators: \hmwkCollaborators}}
}

\fancypagestyle{problemstatement}
{
    \rhead{}
    \lfoot{}
}

%%%%%% Begin document with header and title %%%%%%%%%%%%%%%%%%%%%%%%%

\begin{document}

%%%%%% Header Information %%%%%%%%%%%%%%%%%%%%%%%%%%%%%%%%%%%%%%%%%%%

%%% Shouldn't need to change these
\newcommand{\hmwkClass}{COS 255}
\newcommand{\hmwkSemester}{Spring 2016}

%%% Your name, in standard First Last format
\newcommand{\hmwkAuthorName}{Lukas Leung}
%%% Your NetID
\newcommand{\hmwkAuthorID}{lleung}

%%% The problem set number (just the number)
\newcommand{\hmwkAssignmentNum}{6}

%%% The problem number (just the number)
\newcommand{\hmwkProblemNum}{3}

%%% A list of your collaborators' NetIDs, separated by ", ".
%%% You can use a new line ("\\") in the middle to prevent a long
%%% list from overflowing.
\newcommand{\hmwkCollaborators}{}
%%% Sets the collaborator list to appear on the first page
\thispagestyle{fancycollab}

%%%%%%% begin Solution %%%%%%%%%%%%%%%%%%%%%%%%%%%%%%%%%%%%%%%%%%%%
%%%%%%% start ROTC %%%%%%%%%%%%%%%%%%%%%%%%%%%%%%%%%%%%%%%%%%

\section{Book 6.16: ROTC}
\textbf{Background} \\
~\indent There are many sunny days in Ithaca, New York; but this year, as it happens, the
spring ROTC picnic at Cornell has fallen on a rainy day. The ranking officer decides
to postpone the picnic and must notify everyone by phone. Here is the mechanism
she uses to do this. \\
~\indent Each ROTC person on campus except the ranking officer reports to a unique
$superior\ officer$. Thus the reporting hierarchy can be described by a tree $T$, rooted
at the ranking officer, in which each other node $v$ has a parent node $u$ equal to his or
her superior officer. Conversely, we will call $v$ a $direct\ subordinate$ of $u$. \\
~\indent To notify everyone of the postponement, the ranking officer first calls each of her
direct subordinates, one at a time. As soon as each subordinate gets the phone call,
he or she must notify each of his or her direct subordinates, one at a time. The
process continues this way until everyone has been notified. Note that each person in
this process can only call direct subordinates on the phone. \\
~\indent We can picture this process as being divided into rounds. In one round, each
person who has already learned of the postponement can call one of his or her direct
subordinates on the phone. The number of rounds it takes for everyone to be notified
depends on the sequence in which each person calls their direct subordinates. \\
~\indent Give an efficient algorithm that determines the minimum number of rounds needed
for everyone to be notified, and outputs a sequence of phone calls that achieves this
minimum number of rounds.

%%%%%%% end Problem %%%%%%%%%%%%%%%%%%%%%%%%%%%%%%%%%%%%%%%%%%%%%%%

\newpage

%%%%%%% Mathematical Formulation %%%%%%%%%%%%%%%%%%%%%%%%%%%%%%%%%%
\subsection{Mathematical Formulation}
Given an input of the tree T with $n$ nodes in total and depth of $d$. Let the most
number of direct subordinates (children) any one officer (parent) has be expressed as
$c$. We will determine least number of calls needed to reach every node of the tree as
well as the order in which to acheive this.

%%%%%%% Algorithm %%%%%%%%%%%%%%%%%%%%%%%%%%%%%%%%%%%%%%%%%%%%%%%%%

\subsection{Solution}
This algorithm begin by doing a by-level bottom up traversal. As it traverses, it will
assign values to each node based off its children. The way it does this is first sorting
all of its children from highest value to lowest and such that $v_0 \geq v_1 \geq ... \geq v_{n-1}$ for
a node with $n$ children, then saying:
\[ value =
\begin{cases}
    0\ if\ no\ children \\
    max_{i \in 0..(n-1)}\ (child_{i}value + 1 + i)\ else
\end{cases}
\]
With this we will develope our values and when we reach the top level ($n$), its value
will be the least number of calls needed to be made. Our path to take then is to call
the open value node from each seen node.

\begin{algorithm}[H]
\caption{Fill out Tree and get num Calls}
\begin{algorithmic}
    \Procedure{getNumCalls}{Tree T}
        \For{$level \in 1..d$}
            \For{$node \in T.level(level)$}
                \State int value = 0
                \If{node.hasChildren}
                    \State children.inverseSort
                    \State int i = 0
                    \For{$child \in children$}
                        \State childV = child.value + 1 + (i++)
                        \State value = max(value, childV)
                    \EndFor
                \EndIf
                \State node.\Call{setValue}{value}
            \EndFor
        \EndFor
        \State print(root.value)  // least num calls
        \State \Call{traverse}{T} // T has all values filled out and inverse sorted
    \EndProcedure
\end{algorithmic}
\end{algorithm}


%%%%%%% Correctness %%%%%%%%%%%%%%%%%%%%%%%%%%%%%%%%%%%%%%%%%%%%%%%
\newpage
\subsection{Correctness}
%%%%%%% PROPOSITION 1 %%%%%%%%%%%%%%%
\begin{proposition}
~ \\ \indent This algorithm will determine the correct minimum number of calls.
\end{proposition}

\begin{proof}
~ \\ \indent We do this by construction. We observe that doing a bottom up traversal we can
see that as long as the children have the correct value, then the parent will be able to
calculate the correct value. When I say value of a node I am referring to the number
of calls required to fully reach the subtree rooted at this node, assuming the node has
already been reached. $\implies$ if the node is a leaf, its value = 0. Therefore when we have
a node with a single leaf as a child, $\implies$ will take the number of calls that the
leaf will make, 0, + 1 because it has to call the leaf. Now if there are multiple children,
k of them, and all of them are leaves we will note that we cannot call multiple leaves at
the same time, therefore the number of calls that this node will have to make is $n$ and
its value will also be n since each of the nodes must make 0 calls. \\
\indent Now lets change it such that each of the children have thier own value $v_1,...,v_n$
we note that it will always be most efficient to call the children with more calls to make
first, therefore let us inverse-sort them based off of their values s.t. $v_1\geq...\geq v_n$
Now we can create a $value$ variable and set it equal to cost of calling $c_1$ which is
$v_1 + 1$ beacause we have to call the node itself. We will then check to see if the cost of calling
$c_2$ exceeds that of $c_1$ as in is cost of calling $c_2$ second $\textgreater$ cost of calling
$c_1$ first? i.e. $value = max(value, v_2 + 1 + 1)$ since it cost 0 to call first, 1
to call second , 2 to call third, etc. We will then continue with this letting
$value = max_{i \in 1..(n)}\ (child_{i}value + 1 + (i-1))$ \\
\indent After we have done this for every node (building from bottom $\rightarrow$ up), we
just ask for the value of the root which will give us the minimum number of calls we
have to make to reach everyone.
\end{proof}


%%%%%%% Analysis %%%%%%%%%%%%%%%%%%%%%%%%%%%%%%%%%%%%%%%%%%%%%%%%%%
\subsection{Analysis}

%%%%%%% PROPOSITION 1 %%%%%%%%%%%%%%%
\begin{proposition}
\label{numq}
The \underline{space complexity} of this algorithm is \textbf{O(size(T))}
\end{proposition}

\begin{proof}
~ \\ \indent This is due to the fact that all of our data is stored in the tree itself:
\end{proof}

%%%%%%% PROPOSITION 2 %%%%%%%%%%%%%%%
\begin{proposition}
\label{numq}
The \underline{time complexity} of this algorithm is \textbf{O(N$\cdot$n$^{\ast}\cdot$log(n$^{\ast}$))}
\end{proposition}

\begin{proof}
~ \\ \indent Let us define the average number of children every node has as $n^{\ast} \implies$ the
cost to sort will be $n^{\ast}\cdot log(n^{\ast})$ now since we have to sort for each node, N times
$\implies N\cdot n^{\ast}\cdot log(n^{\ast})$ Note here though that $n^{\ast}$ is likely to be a low
number when compared to $N$.
\end{proof}


%%%%%%% end ROTC %%%%%%%%%%%%%%%%%%%%%%%%%%%%%%%%%%%%%%%%%%%


%%%%%%% end Solution %%%%%%%%%%%%%%%%%%%%%%%%%%%%%%%%%%%%%%%%%%%%%%

\end{document}